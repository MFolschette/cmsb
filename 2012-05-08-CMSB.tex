\documentclass[fleqn]{llncs}

\usepackage[english]{babel}
\usepackage[utf8]{inputenc}
\usepackage[T1]{fontenc}
%\usepackage[top=2cm, bottom=2cm, left=2cm, right=2cm]{geometry} % Marges

\usepackage{amsmath}  % Maths
\usepackage{amsfonts} % Maths
\usepackage{amssymb}  % Maths
\usepackage{stmaryrd} % Maths (crochets doubles)

\usepackage{theorem} % Styles supplémentaires pour théorèmes
\usepackage{url}     % Mise en forme + liens pour URLs
\usepackage{array}   % Tableaux évolués

% Police
%\usepackage{lmodern}
%\usepackage{libertine}


%%%%%%%%%%%%%%%%%%%%%%%%%%%%%%%%%%%%%%
\usepackage{tikz}
\newdimen\pgfex
\newdimen\pgfem
\usetikzlibrary{arrows,shapes,shadows,scopes}
\usetikzlibrary{positioning}
\usetikzlibrary{matrix}
\usetikzlibrary{decorations.text}
\usetikzlibrary{decorations.pathmorphing}

% Macros relatives à la traduction de PH avec arcs neutralisants vers PH à k-priorités fixes

% Macros générales
\newcommand{\ie}{\textit{i.e.} }

\def\Pint{\textsc{PINT}}

% Notations générales pour PH
\newcommand{\PH}{\mathcal{PH}}
\newcommand{\PHs}{\mathcal{S}}
%\newcommand{\PHp}{\mathcal{P}}
\newcommand{\PHp}{\textcolor{red}{\mathcal{P}}}
\newcommand{\PHproc}{\mathcal{P}}
\newcommand{\PHa}{\mathcal{A}}
\newcommand{\PHl}{\mathcal{L}}
\newcommand{\PHn}{\mathcal{N}}

\newcommand{\PHfrappeur}{\mathsf{frappeur}}
\newcommand{\PHcible}{\mathsf{cible}}
\newcommand{\PHbond}{\mathsf{bond}}
\newcommand{\PHsorte}{\mathsf{sorte}}
\newcommand{\PHbloquant}{\mathsf{bloquante}}
\newcommand{\PHbloque}{\mathsf{bloquee}}

\newcommand{\PHfrappeR}{\textcolor{red}{\rightarrow}}
\newcommand{\PHmonte}{\textcolor{red}{\Rsh}}

\newcommand{\PHfrappeA}{\rightarrow}
\newcommand{\PHfrappeB}{\Rsh}
%\newcommand{\PHfrappe}[3]{\mbox{$#1\PHfrappeA#2\PHfrappeB#3$}}
%\newcommand{\PHfrappebond}[2]{\mbox{$#1\PHfrappeB#2$}}
\newcommand{\PHfrappe}[3]{#1\PHfrappeA#2\PHfrappeB#3}
\newcommand{\PHfrappebond}[2]{#1\PHfrappeB#2}
\newcommand{\PHobjectif}[2]{\mbox{$#1\PHfrappeB^*\!#2$}}
\newcommand{\PHconcat}{::}
\newcommand{\PHneutralise}{\rtimes}

\def\PHget#1#2{{#1[#2]}}
%\newcommand{\PHchange}[2]{#1\langle #2 \rangle}
\newcommand{\PHchange}[2]{(#1 \Lleftarrow #2)}
\newcommand{\PHarcn}[2]{\mbox{$#1\PHneutralise#2$}}
\newcommand{\PHjoue}{\cdot}

\newcommand{\PHetat}[1]{\mbox{$\langle #1 \rangle$}}

% Notations spécifiques aux graphes d'états
\newcommand{\PHge}{\textcolor{red}{\mathcal{GE}}}
\newcommand{\PHt}{\mathcal{T}}
\newcommand{\GE}{\mathcal{GE}}
\newcommand{\GEt}{\mathcal{T}}
\newcommand{\GEl}{\PHl}
\newcommand{\GEa}{\PHa}
\newcommand{\GEva}[3]{#1 \stackrel{#2}{\longrightarrow} #3}
\newcommand{\GEval}[3]{#1 \stackrel{#2}{\Longrightarrow} #3}
\newcommand{\GEget}[2]{\PHget{#1}{#2}}

\input{macros/macros-ph}
% Macros spécifiques au Modèle de Thomas / aux RRB

% Notations pour le modèle de Thomas (depuis thèse)
\newcommand{\GRN}{\mathcal{GRN}}
\newcommand{\IG}{\mathcal{G}}
%\def\IG{\mathrm{IG}}
\def\levels{\mathsf{levels}}
\def\levelsA#1#2{\levels_+(#1\rightarrow #2)}
\def\levelsI#1#2{\levels_-(#1\rightarrow #2)}
\newcommand{\PHres}{\mathsf{Res}}

\newcommand{\Kinconnu}{\emptyset}
\newcommand{\RRGva}[3]{#1 \stackrel{#2}{\longrightarrow} #3}
\newcommand{\RRGgi}{\mathcal{G}}
\newcommand{\RRGreg}[1]{\RRGgi_{#1}}
\newcommand{\RRGres}[2]{\PHres_{#1}(#2)}



%\definecolor{darkred}{rgb}{0.5,0,0}
%\definecolor{lightred}{rgb}{1,0.8,0.8}
%\definecolor{lightgreen}{rgb}{0.7,1,0.7}
\definecolor{darkgreen}{rgb}{0,0.5,0}
%\definecolor{darkyellow}{rgb}{0.5,0.5,0}
%\definecolor{lightyellow}{rgb}{1,1,0.6}
%\definecolor{darkcyan}{rgb}{0,0.6,0.6}
%\definecolor{darkorange}{rgb}{0.8,0.2,0}

%\definecolor{notsodarkgreen}{rgb}{0,0.7,0}

%\definecolor{coloract}{rgb}{0,1,0}
%\definecolor{colorinh}{rgb}{1,0,0}
\colorlet{coloract}{darkgreen}
\colorlet{colorinh}{red}
%\colorlet{coloractgray}{lightgreen}
%\colorlet{colorinhgray}{lightred}
%\colorlet{colorinf}{darkgray}
%\colorlet{coloractgray}{lightgreen}
%\colorlet{colorinhgray}{lightred}

%\colorlet{colorgray}{lightgray}


\tikzstyle{grn}=[every node/.style={circle,draw=black,outer sep=2pt,minimum
                size=15pt,text=black}, node distance=1.5cm]
\tikzstyle{inh}=[>=|,-|,draw=colorinh,thick, text=black,label]
\tikzstyle{act}=[->,>=triangle 60,draw=coloract,thick,color=coloract]
%\tikzstyle{inhgray}=[>=|,-|,draw=colorinhgray,thick, text=black,label]
%\tikzstyle{actgray}=[->,>=triangle 60,draw=coloractgray,thick,color=coloractgray]
\tikzstyle{inf}=[->,draw=colorinf,thick,color=colorinf]
%\tikzstyle{elabel}=[fill=none, above=-1pt, sloped,text=black, minimum size=10pt, outer sep=0, font=\scriptsize,draw=none]
\tikzstyle{elabel}=[fill=none,text=black, above=-2pt,%sloped,
minimum size=10pt, outer sep=0, font=\scriptsize, draw=none]
%\tikzstyle{elabel}=[]





%\tikzstyle{plot}=[every path/.style={-}]
%\tikzstyle{axe}=[gray,->,>=stealth']
%\tikzstyle{ticks}=[font=\scriptsize,every node/.style={gray}]
%\tikzstyle{mean}=[thick]
%\tikzstyle{interval}=[line width=5pt,red,draw opacity=0.7]
%\definecolor{lightred}{rgb}{1,0.3,0.3}

%\tikzstyle{hl}=[yellow]
%\tikzstyle{hl2}=[orange]

%\tikzstyle{every matrix}=[ampersand replacement=\&]
%\tikzstyle{shorthandoff}=[]
%\tikzstyle{shorthandon}=[]
%%%%%%%%%%%%%%%%%%%%%%%%%%%%%%%%%%%%%%%%



% Commande À FAIRE
\usepackage{color} % Couleurs du texte
\newcommand{\afaire}[1]{\textcolor{red}{[À FAIRE : #1]}}



% Un vrai symbole pour l'ensemble vide
\renewcommand{\emptyset}{\varnothing}


%\title{}
%\author{}
%\date{2012/04/18}




\begin{document}

The Process Hitting is a framework that allows to study dynamic models. It allows to model systems from Thomas' modeling using the translation given in [XXX]. We present in this paper a new translation from a Process Hitting model to a Biological Regulatory Network. This translation allows to infer an IG from the given Process Hitting, and using this first result, infer a possibly partial parametrization.

\section{Introduction}

\section{Thomas' modeling}
The Thomas' modeling is the historical and widely-used model for the study of dynamical gene systems, and takes the form of a Biological Regulatory Network. In this section, we present this tool by defining the Interaction Graph and the Parametrization. We also justify our extension of the latter to use interval parameters instead of integers.

\section{The Process Hitting framework}
The Process Hitting is a new framework that allows to model dynamic systems with an atomistic point of view. In this section, we present the definitions of this modeling and mention how static analysis makes it efficient to study large systems. We finally remind the way to translate a BRN to a PH using cooperative sorts.

\section{Interaction Graph inference}
The Process Hitting can be used to finely model a system by defining cooperations or studying its behavior using the available tools. One can then decide to translate this model back to Thomas' modeling, and the Interaction Graph inference is the first step to perform this. It relies on an exhaustive approach that looks for possible influences of a gene on the others given all possible configurations. This part gives good results.

\section{Parametrization inference}
The second part of the translation process is to infer the Parametrization. It relies on an exhaustive enumeration of all predecessors of each gene in order to fin attractor processes. The implementation returns a possibly incomplete of all necessary parameters, given the exhaustiveness of the cooperations. \afaire{The last step consists of the enumeration of all possible configurations, given this set of inferred parameters, and some constraints on the Parametrization.}

\section{Example}
In this section we illustrate the functioning of this method with a simple example, and we give some figures about its execution on more substantial examples \afaire{TCRSIG40? EGFR20?}.

\section{Future work}
In order to gain some performance in the inference, we may consider several leads. Model reduction by cooperative sorts removal may allow to obtain good results without exhaustive search and with possibly lower complexity. Using the multiplexes semantics would allow to reduce the possible parametrizations and make the cooperations appear in the Interaction Graph. Finally, Other ways of coping with the incomplete cooperations could be found.

\section{Conclusion}


\end{document}
