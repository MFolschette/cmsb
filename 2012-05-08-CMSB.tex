\documentclass{llncs}

\usepackage[english]{babel}
\usepackage[utf8]{inputenc}
\usepackage[T1]{fontenc}
%\usepackage[top=2cm, bottom=2cm, left=2cm, right=2cm]{geometry} % Marges

\usepackage{amsmath}  % Maths
\usepackage{amsfonts} % Maths
\usepackage{amssymb}  % Maths
\usepackage{stmaryrd} % Maths (crochets doubles)

\usepackage{theorem} % Styles supplémentaires pour théorèmes
\usepackage{url}     % Mise en forme + liens pour URLs
\usepackage{array}   % Tableaux évolués

% Police
%\usepackage{lmodern}
%\usepackage{libertine}


%%%%%%%%%%%%%%%%%%%%%%%%%%%%%%%%%%%%%%
\usepackage{tikz}
\newdimen\pgfex
\newdimen\pgfem
\usetikzlibrary{arrows,shapes,shadows,scopes}
\usetikzlibrary{positioning}
\usetikzlibrary{matrix}
\usetikzlibrary{decorations.text}
\usetikzlibrary{decorations.pathmorphing}

% Macros relatives à la traduction de PH avec arcs neutralisants vers PH à k-priorités fixes

% Macros générales
\newcommand{\ie}{\textit{i.e.} }

\def\Pint{\textsc{PINT}}

% Notations générales pour PH
\newcommand{\PH}{\mathcal{PH}}
\newcommand{\PHs}{\mathcal{S}}
%\newcommand{\PHp}{\mathcal{P}}
\newcommand{\PHp}{\textcolor{red}{\mathcal{P}}}
\newcommand{\PHproc}{\mathcal{P}}
\newcommand{\PHa}{\mathcal{A}}
\newcommand{\PHl}{\mathcal{L}}
\newcommand{\PHn}{\mathcal{N}}

\newcommand{\PHfrappeur}{\mathsf{frappeur}}
\newcommand{\PHcible}{\mathsf{cible}}
\newcommand{\PHbond}{\mathsf{bond}}
\newcommand{\PHsorte}{\mathsf{sorte}}
\newcommand{\PHbloquant}{\mathsf{bloquante}}
\newcommand{\PHbloque}{\mathsf{bloquee}}

\newcommand{\PHfrappeR}{\textcolor{red}{\rightarrow}}
\newcommand{\PHmonte}{\textcolor{red}{\Rsh}}

\newcommand{\PHfrappeA}{\rightarrow}
\newcommand{\PHfrappeB}{\Rsh}
%\newcommand{\PHfrappe}[3]{\mbox{$#1\PHfrappeA#2\PHfrappeB#3$}}
%\newcommand{\PHfrappebond}[2]{\mbox{$#1\PHfrappeB#2$}}
\newcommand{\PHfrappe}[3]{#1\PHfrappeA#2\PHfrappeB#3}
\newcommand{\PHfrappebond}[2]{#1\PHfrappeB#2}
\newcommand{\PHobjectif}[2]{\mbox{$#1\PHfrappeB^*\!#2$}}
\newcommand{\PHconcat}{::}
\newcommand{\PHneutralise}{\rtimes}

\def\PHget#1#2{{#1[#2]}}
%\newcommand{\PHchange}[2]{#1\langle #2 \rangle}
\newcommand{\PHchange}[2]{(#1 \Lleftarrow #2)}
\newcommand{\PHarcn}[2]{\mbox{$#1\PHneutralise#2$}}
\newcommand{\PHjoue}{\cdot}

\newcommand{\PHetat}[1]{\mbox{$\langle #1 \rangle$}}

% Notations spécifiques aux graphes d'états
\newcommand{\PHge}{\textcolor{red}{\mathcal{GE}}}
\newcommand{\PHt}{\mathcal{T}}
\newcommand{\GE}{\mathcal{GE}}
\newcommand{\GEt}{\mathcal{T}}
\newcommand{\GEl}{\PHl}
\newcommand{\GEa}{\PHa}
\newcommand{\GEva}[3]{#1 \stackrel{#2}{\longrightarrow} #3}
\newcommand{\GEval}[3]{#1 \stackrel{#2}{\Longrightarrow} #3}
\newcommand{\GEget}[2]{\PHget{#1}{#2}}

\input{macros/macros-ph}
% Macros spécifiques au Modèle de Thomas / aux RRB

% Notations pour le modèle de Thomas (depuis thèse)
\newcommand{\GRN}{\mathcal{GRN}}
\newcommand{\IG}{\mathcal{G}}
%\def\IG{\mathrm{IG}}
\def\levels{\mathsf{levels}}
\def\levelsA#1#2{\levels_+(#1\rightarrow #2)}
\def\levelsI#1#2{\levels_-(#1\rightarrow #2)}
\newcommand{\PHres}{\mathsf{Res}}

\newcommand{\Kinconnu}{\emptyset}
\newcommand{\RRGva}[3]{#1 \stackrel{#2}{\longrightarrow} #3}
\newcommand{\RRGgi}{\mathcal{G}}
\newcommand{\RRGreg}[1]{\RRGgi_{#1}}
\newcommand{\RRGres}[2]{\PHres_{#1}(#2)}



%\definecolor{darkred}{rgb}{0.5,0,0}
%\definecolor{lightred}{rgb}{1,0.8,0.8}
%\definecolor{lightgreen}{rgb}{0.7,1,0.7}
\definecolor{darkgreen}{rgb}{0,0.5,0}
%\definecolor{darkyellow}{rgb}{0.5,0.5,0}
%\definecolor{lightyellow}{rgb}{1,1,0.6}
%\definecolor{darkcyan}{rgb}{0,0.6,0.6}
%\definecolor{darkorange}{rgb}{0.8,0.2,0}

%\definecolor{notsodarkgreen}{rgb}{0,0.7,0}

%\definecolor{coloract}{rgb}{0,1,0}
%\definecolor{colorinh}{rgb}{1,0,0}
\colorlet{coloract}{darkgreen}
\colorlet{colorinh}{red}
%\colorlet{coloractgray}{lightgreen}
%\colorlet{colorinhgray}{lightred}
%\colorlet{colorinf}{darkgray}
%\colorlet{coloractgray}{lightgreen}
%\colorlet{colorinhgray}{lightred}

%\colorlet{colorgray}{lightgray}


\tikzstyle{grn}=[every node/.style={circle,draw=black,outer sep=2pt,minimum
                size=15pt,text=black}, node distance=1.5cm]
\tikzstyle{inh}=[>=|,-|,draw=colorinh,thick, text=black,label]
\tikzstyle{act}=[->,>=triangle 60,draw=coloract,thick,color=coloract]
%\tikzstyle{inhgray}=[>=|,-|,draw=colorinhgray,thick, text=black,label]
%\tikzstyle{actgray}=[->,>=triangle 60,draw=coloractgray,thick,color=coloractgray]
\tikzstyle{inf}=[->,draw=colorinf,thick,color=colorinf]
%\tikzstyle{elabel}=[fill=none, above=-1pt, sloped,text=black, minimum size=10pt, outer sep=0, font=\scriptsize,draw=none]
\tikzstyle{elabel}=[fill=none,text=black, above=-2pt,%sloped,
minimum size=10pt, outer sep=0, font=\scriptsize, draw=none]
%\tikzstyle{elabel}=[]





%\tikzstyle{plot}=[every path/.style={-}]
%\tikzstyle{axe}=[gray,->,>=stealth']
%\tikzstyle{ticks}=[font=\scriptsize,every node/.style={gray}]
%\tikzstyle{mean}=[thick]
%\tikzstyle{interval}=[line width=5pt,red,draw opacity=0.7]
%\definecolor{lightred}{rgb}{1,0.3,0.3}

%\tikzstyle{hl}=[yellow]
%\tikzstyle{hl2}=[orange]

%\tikzstyle{every matrix}=[ampersand replacement=\&]
%\tikzstyle{shorthandoff}=[]
%\tikzstyle{shorthandon}=[]
%%%%%%%%%%%%%%%%%%%%%%%%%%%%%%%%%%%%%%%%



% Commandes À FAIRE
\usepackage{color} % Couleurs du texte
\newcommand{\afaire}[1]{\textcolor{red}{[À FAIRE~: #1]}}
\newcommand{\resume}[1]{\textcolor{blue}{#1}}
\newcommand{\todo}[1]{\textcolor{darkgreen}{[#1]}}



% Un vrai symbole pour l'ensemble vide
\renewcommand{\emptyset}{\varnothing}

% Id est
\newcommand{\ie}{\textit{i.e.} }


\title{xxxx}

\author{Maxime Folschette\inst{1,2}, Lo\"ic Paulev\'e\inst{3}, Katsumi Inoue\inst{2}, Morgan Magnin\inst{1}, Olivier Roux\inst{1}}
\authorrunning{M. Folschette, et al.}

\institute{
LUNAM Universit\'e, \'Ecole Centrale de Nantes, IRCCyN UMR CNRS 6597\\
(Institut de Recherche en Communications et Cybern\'etique de Nantes)\\
1 rue de la No\"e - B.P. 92101 - 44321 Nantes Cedex 3, France.\\
\email{Maxime.Folschette@irccyn.ec-nantes.fr}
\and
\todo{NII}
\and
LIX, \'Ecole Polytechnique, 91128 Palaiseau Cedex, France.
}


\begin{document}

\maketitle

\begin{abstract}
The Process Hitting is a framework that allows to study dynamic models. It allows to model systems from Thomas' modeling using the translation given in [XXX]. We present in this paper a new translation from a Process Hitting model to a Biological Regulatory Network. This translation allows to infer an IG from the given Process Hitting, and using this first result, infer a possibly partial parametrization.
\end{abstract}



\section{Introduction}

\cite{PMR10-TCSB}
\cite{PMR12-MSCS}
\cite{BernotSemBRN}
\cite{BernotMultiplexes}




\section{Frameworks}

\subsection{Thomas' modeling}
We concisely present the Thomas' modeling of BRNs dynamics, merely inspired by
\cite{Richard06,BernotSemBRN}.
We extend the classical formalism by setting parameters to interval of values instead of a single
value, and briefly discuss this addition.

Thomas' formalism lies on two complementary descriptions of the system. First, the
\emph{Interaction Graph} (IG) models the structure of the system by defining the components'
mutual influences.
The \emph{Parametrization} then specifies the levels to which tends a component when a given
configuration of its regulators applies.

The IG is composed of nodes that represent components, and edges labeled with a threshold that stand
for either positive or negative interactions (\pref{def:ig}).
For such an interaction to take place, the expression level of its head component has to be higher than its threshold; otherwise, the opposite influence is expressed.
%Therefore, for any component $b$, a predecessor $a$ of $b$ such that we have $a \xrightarrow{t} b$
%can be either an activator or an inhibitor of $a$, according to the sign of the interaction and if the expression level of $b$ if above or below the threshold $t$.
We call $\levelsA{a}{b}$ (resp. $\levelsI{a}{b}$) the levels of $a$ where it is an
activator (resp. inhibitor) of $b$ (\pref{def:levels});
$l_a$ denotes the maximum level of $a$.
%For the sake of simplicity and to establish a parallel with PH, if $a$ represents a component, we call $a_i$ its $i^\text{th}$ expression level.

\begin{definition}[Interaction Graph]
\label{def:ig}
An \emph{Interaction Graph} (IG) is a triple $(\Gamma, E_+, E_-)$ where $\Gamma$ is a finite number of \emph{components},
and $E_+$ (resp. $E_-$) $\subset \{a \xrightarrow{t} b \mid a, b \in \Gamma \wedge t \in \mathbb{N}\}$
is the set of positive (resp. negative) \emph{regulations} between two nodes, labeled with a \emph{threshold}.

A regulation from $a$ to $b$ is uniquely referenced:
if $a \xrightarrow{t} b \in E_+$ (resp. $E_-$),
$\nexists a \xrightarrow{t'} b \in E_+ \text{ (resp. $E_-$)}, t \neq t'$
and $\nexists t', a \xrightarrow{t'} b \in E_-$ (resp. $E_+$).
\end{definition}

\begin{definition}[Effective levels ($\levels$)]\label{def:levels}
Let $(\Gamma,E_+,E_-)$ be an IG and $a, b \in \Gamma$ two of its components:
\begin{itemize}
  \item if $a \xrightarrow{t} b \in E_+$, $\levelsA{a}{b} = [t; l_a]$ and
    $\levelsI{a}{b} = [0; t-1]$;
  \item if $a \xrightarrow{t} b \in E_-$, $\levelsA{a}{b} = [0; t-1]$ and
    $\levelsI{a}{b} = [t; l_a]$;
  \item otherwise, $\levelsA{a}{b} = \levelsI{a}{b} = \emptyset$.
\end{itemize}
\end{definition}

For all component $a \in \Gamma$, we denote
$\GRNreg{a} = \{ b\in\Gamma\mid \exists b\xrightarrow t a\in E_+\cup E_- \}$
the set of its regulators.
%$E_+(a) = \{b \in \Gamma \mid \exists b \xrightarrow{t} a \in
%E_+\}$ (resp. $E_-(a) = \{b \in \Gamma \mid \exists b \xrightarrow{t} a \in E_-\}$) denotes the set
%of its positive (resp. negative) regulators; and
%$\GRNreg{a} = E_+(a)\cup E_-(a)$ the set of its regulators.

\begin{example*}
\pref{fig:runningBRN}(left) represents an Interaction Graph $(\Gamma,E_+,E_-)$ with
$\Gamma = \{a, b, c\}$,
$E_+ = \{b \xrightarrow{1} a, c \xrightarrow{1} a\}$ and
$E_- = \{a \xrightarrow{2} b\}$;
hence $\GRNreg{a} = \{b, c\}$.
%This IG can represent the same behavior as the PH given in \pref{fig:runningPH-2}.
\end{example*}

\begin{figure}[t]
\begin{minipage}{0.4\linewidth}
\centering
\scalebox{1.2}{
\begin{tikzpicture}[grn]
\path[use as bounding box] (-0.3,-0.75) rectangle (2.5,1.5);
\node[inner sep=0] (a) at (2,0) {a};
\node[inner sep=0] (b) at (0,0) {b};
\node[inner sep=0] (c) at (2,1.2) {c};
%\path
%  node[elabel, below=-1em of a] {$0..2$}
%  node[elabel, below=-1em of b] {$0..1$}
%  node[elabel, below=-1em of c] {$0..1$};
\path[->]
  (b) edge[bend right] node[elabel, below=-2pt] {$+1$} (a)
  (c) edge node[elabel, right=-2pt] {$+1$} (a)
  (a) edge[bend right] node[elabel, above=-5pt] {$-2$} (b);
\end{tikzpicture}
}
\end{minipage}
\begin{minipage}{0.6\linewidth}
\centering
\begin{align*}
K_{a,\{b,c\},\emptyset} &= [2 ; 2] & K_{b,\{a\},\emptyset} &= [0 ; 1] \\
K_{a,\{b\},\{c\}} &= [1 ; 1] & K_{b,\emptyset,\{a\}} &= [0 ; 0] \\
K_{a,\{c\},\{b\}} &= [1 ; 1] &&\\
K_{a,\emptyset,\{b,c\}} &= [0 ; 0] & K_{c,\emptyset,\emptyset} &= [0 ; 1]
\end{align*}
\end{minipage}
\caption{\label{fig:runningBRN}
(left)
IG example.
%Components are represented by nodes labeled with a name and possible expression levels.
Regulations are represented by the edges labeled with their sign and threshold.
For instance, the edge from $b$ to $a$ is labeled $+1$, which stands for: $b \xrightarrow{1} a \in
E_+$.
%and means that if the level of expression of $b$ is equal to (i.e. above) 1, then $b$ activates $a$,
%otherwise, $b$ inhibits $a$.
(right)
Example Parametrization of the left IG.
}
\end{figure}

A \emph{state} $s$ of an IG $(\Gamma, E_+, E_-)$ is an element in $\prod_{a \in \Gamma} [0;l_a]$.
$\GRNget{s}{a}$ refers to the level of component $a$ in $s$.
The specificity of Thomas' approach lies in the use of discrete \emph{parameters} to represent the
focal level interval towards which the component will evolve in each configuration of its regulators
(\pref{def:param}).
Indeed, for each possible state of a BRN, all regulators of a component $a$ are be divided into
\emph{activators} and \emph{inhibitors}, given their type of interaction and expression level,
referred to as the \emph{resources} of $a$ in $s$ (\pref{def:resources}).

While the extension of parameters to interval of values does not add expressivity for boolean
networks, it allows to specify a larger range of dynamics in the general case (w.r.t. to a fixed
IG).
Indeed, assuming a parameter assigned to an interval of three values, it is impossible to
differentiate the three cases with the boolean definition of resources.


\begin{definition}[Discrete parameter $K_{a,A,B}$ and Parametrization $K$]\label{def:param}
For a given component $a \in \Gamma$ and $A$ (resp. $B$) $\subset \GRNreg{a}$ a set of activators (resp. inhibitors) of $a$ such that
$A \cup B = \GRNreg{a}$ and $A \cap B = \emptyset$,
we define the discrete \emph{parameter} $K_{a,A,B} = [i_1; i_2]$ as a non-empty interval towards which the component $a$ will tend
in the states where its activators (resp. inhibitors) are the regulators in set $A$ (resp. $B$).
A complete map $K$ of discrete parameters on an IG $\IG$ is called a \emph{parametrization} of $\IG$.
\end{definition}
%A consequence of this definition is that $0 \leq i_1 \leq i_2 \leq l_a$.
%We also denote: $j < K_{a,A,B} \Leftrightarrow j < i_1$ and $j > K_{a,A,B} \Leftrightarrow j> i_2$.

\begin{definition}[Resources $\GRNres{a}{s}$]\label{def:resources}
For a given state $s$ of a BRN, we define the \emph{activators} and \emph{inhibitors} of $a$ in $s$ as $\GRNres{a}{s} = A,B$, where:
\begin{align*}
  A &= \{b \in \Gamma \mid \GRNget{s}{b} \in \levelsA{b}{a}\} \\
  B &= \{b \in \Gamma \mid \GRNget{s}{b} \in \levelsI{b}{a}\}
\end{align*}
We also denote $\GRNallres{a}$ the set of all possible configurations of resources of $a$:
\[\GRNallres{a} = \{A,B \mid \exists s \in \prod_{a \in \Gamma} [0;l_a], \GRNres{a}{s} = A,B\}\]
\end{definition}

%\begin{example*}
%\pref{fig:runningBRN}(right) gives a Parametrization of the IG of \pref{fig:runningBRN}(left).
%\end{example*}

At last, \pref{def:dynamics} gives the asynchronous dynamics of a BRN using Thomas' parameters.
From a given state $s$, a transition to another state $s'$ is possible provided that only one component $a$ will evolve of one level towards $K_{a,\GRNres{a}{s}}$.

\begin{definition}[Asynchronous dynamics]\label{def:dynamics}
Let $s$ be a state of a BRN using Thomas' parameters $(\IG, K)$ where $\IG = (\Gamma, E_+, E_-)$.
The state that succeeds to $s$ is given by the indeterministic function $f(s)$:
\begin{align*}
  & f(s) = s' \Leftrightarrow \exists a \in \Gamma,
    \GRNget{s'}{a} = f^a(s) \wedge
    \forall b \in \Gamma, b \neq a, \GRNget{s}{b} = \GRNget{s'}{b}
    \quad\text{, with}\\
  & f^a(s) =
  \begin{cases}
    \GRNget{s}{a} + 1 & \text{if } \GRNget{s}{a} < K_{a,A,B} \\% \GRNres{a}{s}} \\
    \GRNget{s}{a} & \text{if } \GRNget{s}{a} \in K_{a,A,B} \\ %\GRNres{a}{s}}\\
    \GRNget{s}{a} - 1 & \text{if } \GRNget{s}{a} > K_{a,A,B} % \GRNres{a}{s}}
  \end{cases}
\end{align*}
where $A,B=\GRNres{a}{s}$.
\end{definition}

\begin{example*}
In the BRN that consists of the IG and Parametrization of \pref{fig:runningBRN}, the following transitions are possible given the dynamics defined in \pref{def:dynamics}:
$\GRNetat{a_0, b_1, c_1} \rightarrow \GRNetat{a_1, b_1, c_1} \rightarrow \GRNetat{a_2, b_1, c_1} \rightarrow
\GRNetat{a_2, b_0, c_1} \rightarrow \GRNetat{a_1, b_0, c_1}$,
where $a_i$ is the component $a$ at level $i$.
\end{example*}



\subsection{The Process Hitting framework}
\resume{The Process Hitting is a new framework that allows to model dynamic systems with an atomistic point of view. In this subsection, we present the definitions of this modeling and mention how static analysis makes it efficient to study large systems. We finally remind the way to translate a BRN to a PH using cooperative sorts.}

\todo{Definition: Process Hitting + dynamics + formalism used after in the article}

\todo{Give a simple example (similar to GRN subsection?)}

\todo{Mention static analysis (fixed points \& reachability)}

\todo{Definition: Generalized dynamics}

\todo{Refining with cooperative sorts}



\section{Interaction Graph Inference}\label{sec:infer-IG}

In order to infer a complete BRN, one has to find the Interaction Graph (IG) first, as some
constraints on the Parametrization rely on it.
Inferring the IG is an abstraction step which consists in determining the global influence of
components on each of its successors.

This section introduces first the notion of focal processes within a Process Hitting
(\pref{ssec:focal}) which is used to characterize well-formed Process Hittings for IG inference
in \pref{ssec:wf}, and as well used by the parametrization inference presented in \pref{sec:infer-K}.
Finally, the rules for infering the interactions between components from a Process Hitting are
described in \pref{ssec:infer-IG}.

We assume hereafter a global Process Hitting $(\PHs,\PHl,\PHa)$ on which the IG inference is to be
performed.

\subsection{Focal Processes}\label{ssec:focal}

Many of the inferences defined in the rest of this paper rely on the knowledge of \emph{focal
processes} w.r.t. a given context (a set of processes that are potentially present).
When such a context applies, we expect to (always) reach one focal process in a bound number of
actions.

Let us first define as $\PHa(S_a,\ctx)$ the set of actions on the sort $a$ having their hitter in
$\ctx$ and target in $S_a$ (\pref{eq:PHa-ctx}).
\begin{equation}
\PHa(S_a,\ctx) \DEF \{ \PHfrappe{b_i}{a_j}{a_k}\in\PHa \mid b_i\in\ctx \wedge a_j\in S_a \}
\enspace.
\label{eq:PHa-ctx}
\end{equation}

We denote $\focals(a,S_a,\ctx)$ the set of focal processes of sort $a$ in the context
$\ctx\cup S_a$
(\pref{def:focals}).

\begin{definition}[$\focals(a,S_a,\ctx)$]\label{def:focals}
Let us define the digraph $G = (V, E)$ where arcs are the bounces within the sort $a$
triggered by actions having their hitter in $\ctx$ and target in $S_a\subseteq L_a$:
\begin{align*}
E  & \DEF \{(a_j,a_k)\in (S_a \times \PHl_a) \mid 
			\exists\PHfrappe{b_i}{a_j}{a_k}\in \PHa(S_a,\ctx) \}
\\
V & \DEF S_a \cup \{ a_k\in L_a\mid \exists (a_j,a_k)\in E\}
\end{align*}
$\focals(a,S_a,\ctx)$ is the set of focal processes of sort $a$ in the context $\ctx$:
\[
\focals(a,S_a,\ctx) \DEF
\begin{cases}
\{ a_i \in V \mid \nexists (a_i,a_j)\in E\} & \text{if $G$ is acyclic},\\
\emptyset & \text{otherwise.}\\
\end{cases}
\]
\end{definition}

We say that a state $s\in\PHl$ \emph{matches} a context $\ctx$ if and only if
$\forall a\in\PHsort(\ctx), \PHget{s}{a}\in\ctx$, where $\PHsort(\ctx)$ is the set of sorts of
processes in $\ctx$.
From \pref{def:focals}, it derives that:
\begin{enumerate}
\item if $\focals(a,S_a,\ctx)=\emptyset$, there exists a 
state $s\in \PHl$ matching $\ctx\cup S_a$ such that $\forall n\in\mathbb N$ there
exists a sequence of $n+1$ actions in $\PHa(S_a,\ctx)$ successively playable in $s$.
\item if $\focals(a,S_a,\ctx)\neq\emptyset$, for all
state $s\in \PHl$ matching $\ctx\cup S_a$,
for any sequence of actions $h^1,\dots,h^k$ in $\PHa(S_a,\ctx)$ successively playable in $s$,
either
\begin{itemize}
\item $(s\play h^1\cdots h^k)[a] \in \focals(a,S_a,\ctx)$;
\item or, $\exists h^{k+1}\in \PHa(a,\ctx)$ playable in $s\play h^1\cdots h^k$.
\end{itemize}
Moreover $k\leq|\PHa(S_a,\ctx)|$ (no cycle of actions possible).
\end{enumerate}

In other words, if $\focals(a,S_a,\ctx)$ is empty, there exists a set of actions in
$\PHa(S_a,\ctx)$ that may be played as many number of times as we want (cycle);
if it is non-empty, all possible succession of actions in $\PHa(S_a,\ctx)$ have a bound length and
lead $a$ either to a process in $S_a$ that is not hit by processes in $\ctx$, or to a process in
$L_a\setminus S_a$.

\begin{example*}
In the Process Hitting of \pref{fig:runningPH-1}, we obtain:
\begin{align*}
\focals(a,L_a,\{b_0,c_0\}) &= \{ a_0 \}
&
\focals(a,L_a,\{b_1,c_1\}) &= \{ a_2 \}\\
\focals(a,L_a,\{b_1,c_0\}) &= \emptyset\enspace.
\end{align*}
\end{example*}

\subsection{Well-formed Process Hitting for Interaction Graph Inference}\label{ssec:wf}

The inference of an IG from a Process Hitting assumes that the Process Hitting defines two types of
sorts:
the sorts corresponding to BRN components, and the cooperative sorts.
This leads to the characterization of a \emph{well-formed} Process Hitting for IG inference.

The identification of sorts modelling components rely on the observation that their processes
represent (ordered) qualitative levels.
Hence an action on such a sort cannot make it bounce to process at a distance more than one.
The set of sorts satisfying such a condition is referred to as $\Gamma$
(\pref{eq:PH-components}).
Therefore, in the rest of this paper, $\Gamma$ denotes the set of components of the BRN to infer.

\begin{equation}
\Gamma \DEF \{a \in \PHs \mid \nexists \PHfrappe{b_i}{a_j}{a_k} \in \PHa, |j - k| > 1\} \\
\label{eq:PH-components}
\end{equation}

Any sort that does not act as a component should then be treated as a cooperative sort.
As explained in \pref{ssec:PH}, the role of a cooperative sort $\upsilon$ is to compute the current
state of set of cooperating processes.
Hence, for each sub-state $\sigma$ formed by the sorts hitting $\upsilon$, $\upsilon$ should
converge to a focal process.
This is expressed by \pref{pro:wf-cooperative-sort}, where
the set of sorts having an action on a given sort $a$ is given by 
$\PHdirectpredec{a}$ (\pref{eq:ph_direct_predec})
and $\PHproc(\sigma)$ is the set of processes that compose the sub-state $\sigma$.

\begin{equation}
\forall a \in \PHs, \PHdirectpredec{a} \DEF \{b \in \PHs \mid \exists \PHfrappe{b_i}{a_j}{a_k}\in\PHa \}
\label{eq:ph_direct_predec}
\end{equation}

\begin{property}[Well-formed cooperative sort]\label{pro:wf-cooperative-sort}
A sort $\upsilon\in\PHs$ is a well-formed cooperative sort if and only if
each configuration $\sigma$ of its predecessors leads $\upsilon$ to a unique focal process,
denoted by $\upsilon(\sigma)$:
\[
\forall \sigma \in {\textstyle\prod_{
a\in\PHdirectpredec{\upsilon} \wedge a\neq \upsilon}}
\PHl_{a},
\focals(\upsilon,\PHl_\upsilon,\PHproc(\sigma)\cup \PHl_\upsilon) = \{ \upsilon(\sigma) \}\]
\end{property}

Such a property allows a large variety of definition of a cooperative sort, but
for the sake of simplicity, does not allow the existence of multiple focal processes.
While this may be easily extended to (the condition becomes 
$\focals(\upsilon,\PHl_\upsilon, \PHproc(\sigma)\cup \PHl_\upsilon)\neq\emptyset$), it makes some
hereafter equations a bit more complex to read as they should handle set of focal processes instead
of a unique focal process.


Finally, \pref{pro:wf-ph} sums up the conditions for a Process Hitting to be suitable for IG
inference.
In addition of having either component sorts or well-formed cooperative sorts, we also impose that
there is no cycle between cooperative sorts, and that
sorts being not hit (\ie{}, serving as invariant environment) are components.

\begin{property}[Well-formed Process Hitting for IG inference]\label{pro:wf-ph}
A Process Hitting is well-formed for IG inference if and only if the following conditions are
verified:
\begin{itemize}
\item 
each sort $a\in\PHs$ either belongs to $\Gamma$, or is a well-formed cooperative sort;
\item 
there is no cycle between cooperative sorts
(the digraph $(\Sigma,\{(a,b)\in(\Sigma\times\Sigma)\mid \exists \PHfrappe{a_i}{b_j}{b_k}\in\PHa
\wedge a\neq b\wedge \{a,b\}\cap\Gamma=\emptyset \})$ is
acyclic);
\item 
sorts having no action hitting them belong to $\Gamma$
($\{ a \in \Sigma\mid \nexists \PHfrappe{b_i}{a_j}{a_k}\in\PHa\} \subset \Gamma$).
\end{itemize}
\end{property}

\begin{example*}
In the model of \pref{fig:runningPH-2}, $bc$ is a well-formed cooperative sort as defined in \pref{pro:wf-cooperative-sort}, because:
\begin{align*}
\focals(bc, \PHl_{bc}, \{b_0, c_0\} \cup \PHl_{bc}) = \{bc_{00}\} && \focals(bc, \PHl_{bc}, \{b_0, c_1\} \cup \PHl_{bc}) = \{bc_{01}\} \\
\focals(bc, \PHl_{bc}, \{b_1, c_0\} \cup \PHl_{bc}) = \{bc_{10}\} && \focals(bc, \PHl_{bc}, \{b_1, c_1\} \cup \PHl_{bc}) = \{bc_{11}\}
\end{align*}
Furthermore, $\PHs = \Gamma \cup \{bc\}$, with $\Gamma = \{a, b, c\}$ as defined in \pref{eq:PH-components}.
Therefore, as $bc$ is the only cooperative sort and does not hit itself,
\pref{pro:wf-ph} holds for the PH in \pref{fig:runningPH-2}, which is well-formed for IG inference.

The model in \pref{fig:runningPH-1} is also well-formed for IG inference as it contains no cooperative sort, and: $\PHs = \Gamma = \{a, b, c\}$.
\end{example*}


\begin{comment}
\begin{center}
\begin{tikzpicture}
% Sortes
\TSetSortLabel{a}{b}
\TSetSortLabel{b}{c}
\TSetSortLabel{z}{a}
\TSetSortLabel{ab}{bc}
\TSort{(0,3)}{a}{2}{l}
\TSort{(0,0)}{b}{2}{l}
\TSetTick{ab}{0}{00}
\TSetTick{ab}{1}{01}
\TSetTick{ab}{2}{10}
\TSetTick{ab}{3}{11}
\TSort{(3,0.5)}{ab}{4}{r}
\TSort{(6,1)}{z}{3}{r}

% Actions de màj de ab
\THit{a_1}{}{ab_0}{.west}{ab_2}
\THit{a_1}{}{ab_1}{.west}{ab_3}
\path[bounce,bend left] \TBounce{ab_0}{}{ab_2}{.south} \TBounce{ab_1}{}{ab_3}{.south};

\THit{a_0}{}{ab_2}{.west}{ab_0}
\THit{a_0}{}{ab_3}{.west}{ab_1}
\path[bounce,bend right] \TBounce{ab_2}{}{ab_0}{.north} \TBounce{ab_3}{}{ab_1}{.north};

\THit{b_0}{}{ab_3}{.west}{ab_2}
\THit{b_0}{}{ab_1}{.west}{ab_0}
\THit{b_1}{}{ab_0}{.west}{ab_1}
\THit{b_1}{}{ab_2}{.west}{ab_3}
\path[bounce,bend right] \TBounce{ab_1}{}{ab_0}{.north} \TBounce{ab_3}{}{ab_2}{.north};
\path[bounce,bend left] \TBounce{ab_0}{}{ab_1}{.south} \TBounce{ab_2}{}{ab_3}{.south};

% Arcs sortant de ab
\THit{ab_2}{}{z_1}{.north west}{z_2}
\THit{ab_2}{}{z_0}{.north west}{z_1}
\path[bounce,bend left] \TBounce{z_1}{}{z_2}{.south} \TBounce{z_0}{}{z_1}{.south};
\THit{ab_3}{}{z_2}{.south west}{z_1}
\THit{ab_3}{}{z_0}{.north west}{z_1}
\path[bounce,bend left] \TBounce{z_2}{bend right}{z_1}{.north} \TBounce{z_0}{}{z_1}{.south};
\THit{ab_1}{}{z_2}{.south west}{z_1}
\THit{ab_1}{}{z_1}{.south west}{z_0}
\path[bounce,bend right] \TBounce{z_2}{}{z_1}{.north} \TBounce{z_1}{}{z_0}{.north};
\THit{ab_0}{}{z_2}{.south west}{z_1}
\THit{ab_0}{}{z_1}{.south west}{z_0}
\path[bounce,bend right] \TBounce{z_2}{}{z_1}{.north} \TBounce{z_1}{}{z_0}{.north};
\end{tikzpicture}
\end{center}
\end{comment}

\subsection{Interaction Inference}\label{ssec:infer-IG}

At this point we can divide the set of sorts $\PHs$ into components ($\Gamma$) and cooperative sorts
($\PHs \setminus \Gamma$) that will not appear in the IG. 
We define in \pref{eq:ph_predec} the set of predecessors of a sort $a$, that is, the sorts influencing $a$
by considering direct actions and possible intermediate cooperative sorts.
The predecessors of $a$ that are components are the regulators of $a$, noted $\PHpredecgene{a}$
(\pref{eq:regulators}).
\begin{align}
\begin{split}
\forall a \in \PHs, \PHpredec{a} &\DEF \{b \in \PHs \mid \exists n \in \mathbb{N}^*, \exists
(c^k)_{k \in [0;n]} \in \PHs^{n+1}, \\
                                   & \quad \quad c^0 = b \wedge c^n = a \\
                                   & \quad \quad \wedge \forall k \in \llbracket 0 ; n-1 \rrbracket,
								   c^k \in \PHdirectpredec{c^{k+1}} \cap (\PHs\setminus\Gamma)\}
\end{split}
\label{eq:ph_predec}
\\
\forall a\in \PHs, \PHpredecgene{a} & \DEF \PHpredec{a} \cap \Gamma
\label{eq:regulators}
\end{align}

We now aim to determine what kind of influence gets each component from its component predecessors.
Let $a \in \Gamma$ be a component and $b \in \PHpredecgene{b}$ a component that may influence it.
\pref{eq:cooperating-with-b} defines 
$\gamma(b\rightarrow a)$ as the set of components cooperating with $b$ to hit $a$, including $b$ and
$a$.
We note $\configs b$ the set of configurations of such components (\pref{eq:configurations}).
\begin{align}
\begin{split}
\gamma(b\rightarrow a)  &\DEF \{ a, b \} \cup \{ c \in \Gamma \mid 
			\exists \upsilon\in\PHs\setminus\Gamma,
				\upsilon\in\PHpredec{a} \wedge \{b,c\}\subset\PHpredec{\upsilon} \}
\end{split}
\label{eq:cooperating-with-b}
\\
\configs b & \DEF \textstyle\prod_{c\in\gamma(b\rightarrow a)} L_c
\label{eq:configurations}
\end{align}

Given a configuration $\sigma\in\configs b$, $\ctx(\sigma)$ refers to the set of focal processes
regulated by $b$ that can hit the sort $a$ (\pref{eq:ctx-sigma}).
This set is composed of the active process of sort $b$ ($\sigma[b]$), and the focal process (assumed
unique) of the cooperative sorts $\upsilon$ hitting $a$ that have $b$ as a predecessor.
The evaluation of the focal process of $\upsilon$ in context $\sigma$, denoted $\upsilon(\sigma)$,
rely on \pref{pro:wf-cooperative-sort}, which gives its value when all the direct predecessors of
$\upsilon$ are defined in $\sigma$.
When a predecessor $\upsilon'$ is not in $\sigma$, we extend the evaluation by recursively computing
the focal value of $\upsilon'$ is $\sigma$, as stated in \pref{eq:cooperative-eval}.
Because there is no cycle between cooperative sorts, this recursive evaluation of $\upsilon(\sigma)$
always terminates.

\begin{align}
\ctx(\sigma) & \DEF \{ \sigma[b] \} \cup \{ \upsilon(\sigma) \mid
\upsilon\in\PHdirectpredec{a} \wedge  \upsilon \notin\Gamma \wedge b\in \PHpredecgene{\upsilon} \}
\label{eq:ctx-sigma}
\\
\upsilon(\sigma) & \DEF
\upsilon(\sigma \uplus \state{\upsilon'(\sigma) \mid 
	\upsilon'\in\PHdirectpredec{\upsilon} \wedge
	\upsilon'\in\PHs\setminus\Gamma })
\label{eq:cooperative-eval}
\end{align}

We aim at inferring that $b$ activates (inhibits) $a$ if there exists a configuration where increasing
the level of $b$ makes possible the increase (decrease) of the level of $a$.
This is analogous to standard IG inferences from boolean and discrete maps 
\cite{RiCo07}.

This reasoning can be straightforwardly applied to PH when inferring the influence of $b$ for $a$ when
$b\neq a$ (\pref{eq:edges-inference-b}).
Given a configuration $\sigma\in\configs b$, 
$\focals(a,\{a_i\},\ctx(\sigma))$ gives the bounces that a given process $a_i$ can make in the
context $\ctx(\sigma)$.
We note $\sigma\{b_i\}$ the configuration $\sigma$ where the process of sort $b$ has been replaced
by $b_i$.
If there exists $b_i,b_{i+1}\in L_b$ such that one bounce in 
$\focals(a, \{\sigma[a]\}, \ctx(\sigma\{b_i\}))$
has a lower (resp. higher) level that one bounce in
$\focals(a, \{\sigma[a]\}, \ctx(\sigma\{b_{i+1}\}))$, then
$b$ as positive (resp. negative) influence on $a$ with a maximum threshold $l=j$.

Then, we infer that $a$ has a self-influence if its current level has an impact on its own evolution
driven by a process $b_h$ hitting $a$ ($b\in\PHdirectpredec{a}$).
Given $a_i,a_{i+1}\in L_a$, we pick $a_j\in\focals(a,\{a_i\},\{a_i,b_h\})$ and
$a_k\in\focals(a,\{a_{i+1}\},\{a_{i+1},b_h\})$.
If $k=j+1$, we can not conclude as there is no difference in the evolution of both levels.
If $k\neq j+1$ and $k-j\neq 0$, then $a_i$ and $a_{i+1}$ have divergent evolutions: we infer an
influence of sign of $k-j$ at threshold $i+1$.
We note that some aspects of this inference are arbitrary and may have impact on the number of
parameters to infer in the next section.
In particular, in some cases, the use of intervals for Thomas' parameters drop the requirement of
inferring a self-activation.
Future work may propose alternative definitions of self-influences inference in order to range over
different parametrization configurations.

\begin{comment}
a global overview of the evolution of each level
of $a$ w.r.t. a configuration $\sigma\in\configs a$.
The evolution of $a_i$ in the context $\ctx(\sigma\{a_i\})$ is given by the function
$\epsilon(a_i,\sigma)$ (\pref{eq:epsilon}) which returns $\varnothing$ if there is no action
possible, $+$ (resp. $-$) if all actions makes $a_i$ bounce to a higher (resp. lower) level,
and $\pm$ if both evolutions are possible.
\begin{equation}
\epsilon(a_i, \sigma) \DEF
\begin{cases}
\varnothing & \text{if }\focals(a,\{a_i\},\ctx(\sigma\{a_i\}))=\{ a_i \} \\
+ &  \text{if }\focals(a, \{a_i\},\ctx(\sigma\{a_i\})) = \{ a_{i+1} \}\\
- &  \text{if }\focals(a, \{a_i\},\ctx(\sigma\{a_i\})) = \{ a_{i-1} \}\\
\pm & \text{otherwise.}
\end{cases}
\label{eq:epsilon}
\end{equation}
We infer the self-influence of $a$ by checking one of the following three cases.
First, if there exists $a_i,a_j\in L_a, i <j$ such that $\epsilon(a_i,\sigma)$ and
$\epsilon(a_j,\sigma)$ are of opposite sign, then $a$ has a self-influence of the sign of the latter
$\epsilon$ (with a maximum threshold $k=j$).
Second, we look at the evolution at the limit levels of $a$:
if $\epsilon(a_0,\sigma)=\bar s$ or $\epsilon(a_{l_a},\sigma)=s$, we infer an influence of sign $s$
with a threshold $k=l_a$.
We note that this case can only apply for negative interactions (as $\epsilon(a_0,\sigma)$ (resp.
$\epsilon(a_{l_a},\sigma)$ can never be negative (resp. positive)).
Third, if $\forall a_i\in L_a$, $\epsilon(a_i,\sigma)$ is either $\varnothing$ or of sign $s$, we
ensure there exists $a_j\in L_a$ such that $\epsilon(a_j,\sigma)=s$ and we infer a self-influence of
sign $s$ and threshold $k=j$.
\end{comment}


\pref{def:inference-edges} details the inference of all existing influences between genes occurring
with a threshold $l$.
These influences are split into positive and negative ones, and represent possible edges in the final IG.
\begin{definition}[Edges inference]\label{def:inference-edges}
We define the set of positive (resp. negative) influences $\hat{E}_+$ (resp. $\hat{E}_-$) for any
$a\in\Gamma$ by:
\begin{align}
\begin{split}
\forall b\in\PHpredecgene{a}, b\neq a, \\
b\xrightarrow l a \in \hat{E}_s & \Longleftrightarrow
 \exists \sigma\in\configs b, \exists b_i,b_{i+1}\in \PHl_b,\\
& \qquad\qquad
        \exists a_{j}\in\focals(a, \{\sigma[a]\}, \varsigma(\sigma\{b_i\})), \\
& \qquad\qquad
        \exists a_{k}\in\focals(a, \{\sigma[a]\}, \varsigma(\sigma\{b_{i+1}\})), \\
& \qquad\qquad\qquad
                        s = \f{sign}(k - j) \wedge l = i+1
\end{split}
\label{eq:edges-inference-b}
\\
\begin{split}
a \xrightarrow l a \in \hat{E}_s & \Longleftrightarrow
\exists b\in \PHdirectpredec{a}, \exists b_h \in L_b,
       \exists a_i,a_{i+1}\in \PHl_a, \\
& \qquad\qquad
        \exists a_{j}\in\focals(a, \{a_i\}, \{a_i,b_h\}), \\
& \qquad\qquad
        \exists a_{k}\in\focals(a, \{a_{i+1}\}, \{a_{i+1},b_h\}), \\
& \qquad\qquad\quad
			k \neq j+1
				\wedge s = \f{sign}(k - j) \wedge l = i+1
\end{split}
\label{eq:edges-inference-a}
\end{align}
where $s \in \{ +, - \}$, $\bar s = + \EQDEF s = -$, $\bar s = - \EQDEF s = +$,
$\f{sign}(n) = + \EQDEF n > 0$,
$\f{sign}(n) = - \EQDEF n < 0$,
and $\f{sign}(0) \EQDEF 0$.
\end{definition}

\begin{comment}
\begin{align*}
\begin{split}
a \xrightarrow l a \in \hat{E}_s & \Longleftrightarrow
\neg(\PHpredecgene{a} = \{a\} \wedge \focals(a, \PHl_a, \PHl_a) = [a_i;a_j], i\leq j) \\
& \wedge \exists\sigma\in L_a\times\textstyle\prod_{c\in \PHpredecgene{a}} L_c, ( 
       (\exists a_i,a_{i+1}\in \PHl_b, \\
& \qquad\qquad
        \exists a_{j}\in\focals(a, \{a_i\}, \ctx'(\sigma\{a_i\})), \\
& \qquad\qquad
        \exists a_{k}\in\focals(a, \{a_{i+1}\}, \ctx'(\sigma\{a_{i+1}\})), \\
& \qquad\qquad\quad
			k \neq j+1
				\wedge s = \f{sign}(k - j) \wedge l = i+1\\
& \qquad\vee
	(\forall a_i\in L_a, \focals(a,\{a_i\},\ctx'(\sigma\{a_i\})) = \{a_i\} \\
& \qquad\qquad 
		\wedge s=+ \wedge l=1)
)
\end{split}
\end{align*}
\end{comment}

\begin{comment}
\begin{align*}
\begin{split}
a \xrightarrow k a \in \hat{E}_s & \Longleftrightarrow
\neg(\PHpredecgene{a} = \{a\} \wedge \focals(a, \PHl_a, \PHl_a) = [a_i;a_j], i\leq j) \\
& \wedge \exists\sigma\in\configs a, (
        (\exists a_i,a_j\in L_a, i < j, \\
& \qquad\qquad
                \epsilon(a_i,\sigma) \in \{\bar s,\pm\}
					\wedge \epsilon(a_j,\sigma) \in \{ s, \pm \}  \wedge k = j) \\
& \qquad \vee
        ((\epsilon(a_0, \sigma) = \bar s \vee \epsilon(a_{l_a}, \sigma) = s) \wedge k = l_a) \\
& \qquad \vee
        (\epsilon(a_0, \sigma) = \epsilon(a_{l_a}, \sigma) = \varnothing \\
& \qquad\qquad       \wedge \forall a_i\in L_a, \epsilon(a_i, \sigma)\in \{\varnothing, s\} \\
& \qquad\qquad       \wedge \exists a_j\in L_a, \epsilon(a_j, \sigma) = s \wedge k = j)
                )
\end{split}
\end{align*}
\end{comment}

We are now able to infer the edges of the final IG by considering positive and negative influences. We infer a positive (resp. negative) edge if there exists a corresponding influence with the same sign. If an influence is both positive and negative, we infer an unsigned edge. In the end, the threshold of each edge is the minimum threshold for which the influence has been found. As unsigned edges represent ambiguous interactions, no threshold is inferred.
\begin{definition}[Interaction Graph inference]\label{def:inference-IG}
\todo{We deduce IG from the Edges Inference:}
$\IG = (\Gamma,E_+,E_-,E_\pm)$, where
\begin{align*}
E_+ &= \{a \xrightarrow{t} b \mid \nexists a \xrightarrow{t'} b \in \hat{E}_-
  \wedge t = \min \{ l \mid a \xrightarrow{l} b \in \hat{E}_+\}\} \\
E_- &= \{a \xrightarrow{t} b \mid \nexists a \xrightarrow{t'} b \in \hat{E}_+
  \wedge t = \min \{l \mid a \xrightarrow{l} b \in \hat{E}_-\}\} \\
E_\pm &= \{a \rightarrow b \mid \exists a \xrightarrow{t} b \in \hat{E}_+ \wedge \exists a \xrightarrow{t'} b \in \hat{E}_-\} \\
\end{align*}
\end{definition}


\begin{example*}
The PH of \pref{fig:runningPH-2} gives the following influences:
\begin{align*}
  \hat{E}_+ = \{b \xrightarrow{1} a, c \xrightarrow{1} a\} && \hat{E}_- = \{a \xrightarrow{2} b\}
\end{align*}
As no influence is found in both $\hat{E}_+$ and $\hat{E}_-$, IG inference therefore leads to a regular IG which is identical to the one given in \pref{fig:runningBRN-ig}.

The IG inference performed on the PH of \pref{fig:runningPH-1} gives the same results.
\end{example*}


\section{Parametrization inference}\label{sec:infer-K}

Given the IG inference results from a PH, as presented in the previous section, one can find the discrete parameters that model the behavior of the studied PH using the method presented in the following.
It relies on an exhaustive enumeration of all predecessors of each component in order to find attractor processes and returns a possibly incomplete Parametrization, given the exhaustiveness of the cooperations.
The last step consists of the enumeration of all possible Parametrizations, given this set of
inferred parameters, \todo{the PH structure}, and some biological constraints on parameters.

\subsection{Parameters inference}

This subsection presents some results related to the inference of independent discrete parameters from a given PH. These results are equivalent to those presented in \cite{PMR10-TCSB}, with notation adapted to be shared with the previous section.
We consider here a global PH $(\PHs,\PHl,\PHa)$ well-formed for IG inference, on which the parametrization inference has now to be performed. We suppose that the extended IG $(\Gamma, E_+, E_-, E_\pm)$ inferred from this PH contains no unsigned edge, that is: $E_\pm = \emptyset$, and thus can be seen as the regular IG $(\Gamma, E_+, E_-)$.
Let $K_{a,A,B}$ be the parameter we want to infer, for a given component $a \in \Gamma$,
%and $A,B \in \GRNallres{a}$ a configuration of resources of $a$ (activators and inhibitors).
and $A \subset \GRNreg{a}$ (resp. $B \subset \GRNreg{a}$) a set of activators (resp. inhibitors) in the regulators of $a$.
This inference, as for the Interaction Graph inference, relies on the search of focal processes of the component for the given configuration of its regulators.

For each sort $b \in \GRNreg{a}$, we define a context $C^b_{a,A,B}$ in \pref{eq:param_context} that contains all processes representing the influence of the regulators in the configuration $A,B$.
The context of a cooperative sort $\upsilon$ that regulates $a$ is given in \pref{eq:param_context_coop} as the set of processes that represent the given configuration.
$C_{a,A,B}$ refers to the union of all these contexts (\pref{eq:K-ctx}).
\begin{align}
\label{eq:param_context}
\forall b\in\Gamma,~
C_{a,A,B}^b & \DEF \begin{cases}
\levelsA{b}{a} & \text{if $b \in A$,}\\
\levelsI{b}{a} & \text{if $b \in B$,}\\
L_a		& \text{otherwise;}\\
\end{cases}
\\
\label{eq:param_context_coop}
\forall \upsilon \in \PHpredec{a}\setminus\Gamma,~
C_{a,A,B}^\upsilon & \DEF \{
\upsilon(\sigma) \mid \sigma \in \textstyle\prod_{c\in\PHdirectpredec{\upsilon}}C_{a,A,B}^c \}
\\
C_{a,A,B} & \DEF \textstyle\bigcup_{b\in\PHpredec{a}} C^b_{a,A,B}
\label{eq:K-ctx}
\end{align}

The parameter $K_{a,A,B}$ specifies to which values $a$ eventually evolves as long as the context
$C_{a,A,B}$ holds, which is precisely the definition of the $\focals$ function
(\pref{def:focals} in \pref{ssec:focal}).
Hence $K_{a,A,B} = \focals(a,C^a_{a,A,B},C_{a,A,B})$ if this latter is a non-empty interval
(\pref{pps:param_K}).

\begin{proposition}[Parameter inference]
\label{pps:param_K}
Let $(\PHs, \PHl, \PHh)$ be a Process Hitting well-defined for IG inference, and $\IG = (\Gamma,
E_+, E_-)$ the inferred IG.
Let $A$ (resp. $B$) $\subseteq \Gamma$ be the set of regulators that activate (resp. inhibit) a sort
$a$.
%If $\focals(a,C_{a,A,B})$ is a non-empty interval, then $K_{a,A,B} = \focals(a, C_{a,A,B})$.
If $\focals(a,C^a_{a,A,B},C_{a,A,B})=[a_i;a_j]$ is a non-empty interval, 
	then $K_{a,A,B} = [i;j]$.
\end{proposition}

\begin{example*}
\todo{parler uniquement de $K_b,\{a\},\emptyset$, $K_a,\{b,c\},\emptyset$ et $k_a,\{b\},\{c\}$ dans
le cas de \pref{fig:runningPH-1}.}
Parameters inference performed on the PH in \pref{fig:runningPH-1} gives the following partial Parametrization:
\begin{align*}
K_{b,\{a\},\emptyset} &= [0 ; 1] & K_{a,\{b,c\},\emptyset} &= [2 ; 2] \\
K_{b,\emptyset,\{a\}} &= [0 ; 0] & K_{a,\emptyset,\{b,c\}} &= [0 ; 0] \\
K_{c,\emptyset,\emptyset} &= [0 ; 1]
\end{align*}
No result is found for parameters $K_{a,\{b\},\{c\}}$ and $K_{a,\{c\},\{b\}}$ as the opposite influences of $b$ and $c$ on $a$ prevent any inference.

When performed on the PH in \pref{fig:runningPH-2}, parameters inference returns the complete Parametrization given in \pref{fig:runningBRN}.
\end{example*}

Given the \pref{pps:param_K}, we see that in some cases, the inference of the targeted parameter is impossible.
This can be due to a lack of cooperation between regulators: when two regulators independently hit a component, their actions can have opposite effects, leading to either an indeterministic evolution or to oscillations.
Such an indeterminism is not possible in a GRN as in a given configuration of regulators, a component can only have an interval attractor, and eventually reaches a steady-state.
In order to avoid such inconclusive cases, one has to ensure that no such behavior is allowed by either removing undesired actions or using cooperative sorts to avoid opposite influences between regulators of a component.

\todo{conclure sur la correction de l'inférence d'un paramètre}

\subsection{Admissible parametrizations enumeration}\label{ssec:admissible-K}

When building a BRN, one has to find the parametrization that best describes the desired behavior of the studied system.
Complexity is inherent to this process as the number of possible parametrizations for a given IG is exponential w.r.t. the number of components.
However, the method of parameters inference presented in this section gives some information about necessary parameters given a certain dynamics described by a PH.
This information thus drops the number of possible parametrizations, allowing to find the desired behavior more easily.

\todo{s'assurer qu'un paramètre ne dépasse pas la dynamique imposée par le PH \pref{pro:K-valid};
donner un argument pour la correction}

\begin{property}[Parameter validity]\label{pro:K-valid}
A parameter $K_{a,A,B}$ is valid w.r.t. the PH iff the following equation is verified:
\begin{align*}
\forall \sigma\in \textstyle\prod_{c\in\PHdirectpredec{a} \cup \{a\}} C^c_{a,A,B},
	\sigma[a] = a_i,  & \\
		a_i \notin K_{a,A,B} \Longrightarrow (
  \exists \PHfrappe{c_k}{a_i}{a_j}\in\PHa, \sigma[c] = c_k, & \\
 \qquad a_i < K_{a,A,B} & \Rightarrow j > i  \\
 \qquad \wedge a_i > K_{a,A,B} & \Rightarrow j <i )
\end{align*}
\end{property}
		

%The last step of our method is to enumerate all possible parametrizations regarding the results of
%the parameters inference and
\todo{+ 
some biological constraints given in \cite{BernotSemBRN}, that we sum
up in the following three properties:}

\begin{property}[Extreme values assumption]
Let $\IG = (\Gamma, E_+, E_-)$ be an IG. A parametrization $K$ on $\IG$ satisfies the \emph{extreme values assumption} iff:
\label{prop:param_enum_extreme}
\[
  \forall b \in \Gamma, \GRNreg{b} \neq \emptyset \Rightarrow 0 \in K_{b,\emptyset,\GRNreg{b}} \wedge l_b \in K_{b,\GRNreg{b},\emptyset}
\]
\end{property}

\begin{property}[Activity assumption]
\label{prop:param_enum_activity}
Let $\IG = (\Gamma, E_+, E_-)$ be an IG. A parametrization $K$ on $\IG$ satisfies the \emph{activity assumption} iff:
\begin{align*}
  \forall b \in \Gamma, \forall a \in \GRNreg{b}, \exists A,B \in \GRNallres{a}, K_{b,A,B} <_{[]} K_{b,A \cup \{b\},B \setminus \{b\}}
\\
  \forall b \in \Gamma, \forall a \in \GRNreg{b}, \exists A,B \in \GRNallres{a}, K_{b,A \setminus \{b\},B \cup \{b\}} <_{[]} K_{b,A,B}
\end{align*}
\end{property}

\begin{property}[Monotonicity assumption]
\label{prop:param_enum_monotonicity}
Let $\IG = (\Gamma, E_+, E_-)$ be an IG. A parametrization $K$ on $\IG$ satisfies the \emph{monotonicity assumption} iff:
\[
  \forall b \in \Gamma, \forall A,B \in \GRNallres{b}, \forall A',B' \in \GRNallres{b},
  A \subset A' \wedge B' \subset B \Rightarrow K_{b,A,B} \leq_{[]} K_{b,A',B'}
\]
\end{property}

\begin{comment}
\begin{definition}[Admissible parametrization \& Admissible parametrization with respect to inferred parameters]
\label{def:param_enum_inf}
Let $\PH = (\PHs, \PHl, \PHh)$ be a PH so that IG inference is possible, and $\IG = (\Gamma, E_+,
E_-)$ the inferred IG.
A parametrization $K$ on $\IG$ is said to be \emph{admissible} iff it respects
the extreme values assumption, the activity assumption and the monotonicity assumption.
A parametrization $K$ on $\IG$ is said to be \emph{admissible with respect to the
inferred parameters} iff it is admissible and that all parameters that can be inferred regarding
\pref{pps:param_K} are equal to their inferred value.
\end{definition}

\todo{utilité de “Admissible parametrization” seul ?}
\end{comment}


\subsection{Answer Set Programming implementation concepts}

\newcommand{\ti}[1]{\texttt{\textit{#1}}}
\newcommand{\aspil}[1]{\texttt{#1}}
\newcommand{\asp}[1]{\begin{itemize} \item[] \aspil{#1} \end{itemize}}

It is then essential to get an efficient method to enumerate all the admissible parametrizations. We choose to focus on Answer Set Programming (ASP) \cite{Baral03} to address this issue. The motivations are following: 
\begin{itemize}
\item ASP efficiently tackles the inherent complexity of the models we use, thus allows an efficient execution of the formal tools defined in this paper.
\item It allows to easily constrain the answers according to properties or cardinalities.
\end{itemize}
We now synthesize some key points to make the reader better understand our ASP implementation with the enumeration example.

When the step of admissible Parametrizations enumeration is reached, a lot of information has been gathered about the studied system: in addition to the starting PH model, corresponding data have been inferred about a complete IG and a possibly partial parametrization.
All this information describing the model can be expressed in ASP using facts.
For functional purposes, we assign a unique label to all discrete parameters of a given component, which allows to refer to a parameter using two variables (in other terms, we define a unique label for each couple $A,B$ of activators and inhibitors).
For example, if we want to express that a parameter of component \ti{a} has the label \ti{i}, we can use an atom named \aspil{param\_label} in the following fact:
\asp{param\_label(\ti{a}, \ti{i}).}

Defining a set in ASP is equivalent to defining the rule for belonging to this set. For example, we can define an atom \aspil{param\_act} that describes the set of all active regulators for a parameter of gene \ti{a} and label \ti{i} (\ie the set $A$ of a parameter $K_{\ti{a},A,B}$). For example, the assignation of label \ti{i} to $K_{\ti{a},\{\ti{b},\ti{c}\},\{\ti{d}\}}$, gives:
\asp{param\_act(\ti{a}, \ti{i}, \ti{b}). \item[] param\_act(\ti{a}, \ti{i}, \ti{c}).}
The absence of such a rule involving \ti{d} with label \ti{i} indicates that \ti{d} is not an activator in the configuration of regulators related to this parameter.

Rules allow more detailed declarations than facts as they have a body containing constraints. Variables can be used to produce interesting rules.
For instance, in order to define the set of expression levels of a component, we can declare:
\asp{component\_levels(A, 0..L) :- component(A, L).}
where the \aspil{component(A, L)} atom stands for the existence of a component \aspil{A} with a maximum level \aspil{L}.
Considering this declaration, Clingo will ground any possible answer for the atom \aspil{component\_levels} by binding all possible values of its terms regarding all existing facts: an answer \aspil{component\_levels(\ti{a}, \ti{k})} will depend on the existence of a fact like given in the body, and the second term will also depend on the constraint $0 \leq \ti{k} \leq \aspil{L}$.

The enumeration of all possible parametrizations can be performed using a cardinality, which constrains the number of answers for some variables.
Such a cardinality gives the set of atoms to enumerate in curly brackets, and a lower and upper bounds constraining the number of allowed answers.
For example,
\asp{1 \{enum\_param(A,P,V) : component\_levels(A,V)\} :- \item[] ~~~~~is\_component(A), param\_label(A,P), not infered\_param(A,P).}
means that any parameter of label \aspil{P} and gene \aspil{A} must contain at least one level value (\aspil{V}) in the possible expression levels of \aspil{A}.
Indeed, the lower bound is 1, forcing at least one element in the parameter, but no upper bound is specified, allowing up to any number of answers.
The body (right-hand side) of the rule also checks for the existence of component \aspil{A} and parameter label \aspil{P}, and constrains that the parametrization inference was not conclusive for the considered parameter (otherwise we would consider only the value of the inferred parameter).
Such a constraint gives multiple results as any set of atoms satisfying the cardinality will lead to a new global set of answers.
In this way, we can enumerate all possible parametrization which respects the results of parameters
inference, but completely disregarding the notion of admissible parametrizations given in
\pref{ssec:admissible-K}.

In order to filter only admissible parametrizations regarding the previous properties from all obtained answers, we rely on integrity constraints.
Such constraint is a rule with no head, that makes an answer set unsatisfiable if its body turns out to be true.
For the sake of clarity, we note $K^i_{a,A,B}$ the parameter of component $a$ whose regulators $A,B$ are assigned to the label $i$. Hence, supposing that:
\begin{itemize}
  \item the \aspil{less\_active(\ti{a}, \ti{i}, \ti{j})} atom means that $K^\ti{i}_{\ti{a},A,B}$ stands for a configuration with less activating regulators than $K^\ti{j}_{\ti{a},A',B'}$ (\ie $A \subset A'$),
  \item the \aspil{param\_inf(\ti{a}, \ti{i}, \ti{j})} atom means that $K^\ti{i}_{\ti{a},A,B} \leq K^\ti{j}_{\ti{a},A',B'}$,
\end{itemize}
then the monotonicity assumption can be formulated as the following integrity constraint:
\asp{:- less\_active(A,P1,P2), not param\_inf(A,P1,P2).}
where the \aspil{not} keyword stands for a negation: the whole body becomes false if the atom in the negation is true. This integrity constraint indeed removes all parametrization results where a couple of parameters $K_{a,A,B}$ and $K_{a,A',B'}$ exists such that $A \subset A'$ and $K_{a,A,B} > K_{a,A',B'}$, which would violate the monotonicity assumption. Of course, other assumptions can be formulated in the same way.

In conclusion, the approach succinctly described above allows to describe all given information about a BRN (PH model and informations about IG and Thomas' parameters) in order to allow ASP programs execution.
Such logic programming have been used to solve all steps of Thomas' modeling inference, but it finds a particularly interesting application in the enumeration of parameters: all possible parametrizations are generated in separate answer sets, and integrity constraints are formulated to remove answer sets that do not fit the assumptions of admissible parametrizations.
This method allows to efficiently use the parameter inference results to cut the number of possible parametrization, and reduce the number of interesting parametrizations to consider in the end.


\section{Examples}\label{sec:examples}
\resume{In this section we illustrate the functioning of this method with a simple example, and we give some figures about its execution on more substantial examples.}

In this section, we present an example of IG an Parametrization inference, and give some results about the implementation of the method described in this paper.

\subsection{Running example}
The inference of BRN using René Thomas's parameters presented in the previous sections is illustrated by an example in the following.
Consider the PH given in \pref{fig:runningPH-1}.
It contains no cooperative sort as all sorts behave like components; given \pref{eq:PH-components}, we have: $\Gamma = \{a, b, c\}$.
this PH is therefore well-formed for IG inference.
We wish to infer the influence of $b$ towards $a$, that is, infer the sign and threshold of the edge $b \rightarrow a$.
We have: $\PHpredecgene{a} = \{b, c\}$, but as $b$ and $c$ do not cooperate, it comes: $\gamma(b \rightarrow a) = \{a, b\}$.
In order to understand the influence of $b$ on $a$, we will study the evolutions of $a$ in the following set of configurations: $\configs{b} = \{\PHetat{a_0, b_0}, \PHetat{a_0, b_1}, \PHetat{a_1, b_0}, \PHetat{a_1, b_1}\}$.

Let us consider the state: $\sigma = \PHetat{a_1, b_0}$. We thus have:
\begin{align*}
\f{bounces}(\PHget{\sigma}{a},\varsigma(\sigma\{b_0\})) &= \f{bounces}(a_1,\{a_1, b_0\})) = \{a_0\} \\
\f{bounces}(\PHget{\sigma}{a},\varsigma(\sigma\{b_1\})) &= \f{bounces}(a_1,\{a_1, b_1\})) = \{a_2\}
\end{align*}
More generally, we can only find positive influences of $b$ on $a$, as for all $\sigma \in \configs{b}$,
\begin{align*}
&\f{bounces}(\PHget{\sigma}{a},\varsigma(\sigma\{b_0\})) = \{a_k\} \\
&\wedge \f{bounces}(\PHget{\sigma}{a},\varsigma(\sigma\{b_1\})) = \{a_l\} \\
&\wedge k < l
\end{align*}
Given these results, as $k < l \Rightarrow \f{sign}(l - k) = +$, we deduce that $b$ has only a positive influence on $a$:
$b \xrightarrow{1} a \in \hat{E}_+$ and $\nexists b \xrightarrow{t'} a \in \hat{E}_-$.
We find the same results for the influence of $c$ on $a$ as $b$ and $c$ have identical actions on $a$.

Then, we can infer the influence of $a$ on $b$ --- that consists in the only action $\PHfrappe{a_2}{b_1}{b_0}$.
Following the same kind of reasoning, and considering for example the state $\sigma' = \PHetat{a_2, b_1}$, we find that:
\begin{align*}
\f{bounces}(\PHget{\sigma}{b},\varsigma(\sigma\{a_1\})) &= \f{bounces}(b_1,\{a_1, b_1\})) = \{b_1\} \\
\f{bounces}(\PHget{\sigma}{b},\varsigma(\sigma\{a_2\})) &= \f{bounces}(b_1,\{a_2, b_1\})) = \{b_0\}
\end{align*}
which gives: $a \xrightarrow{2} b \in \hat{E}_-$, but no influence of $a$ on $b$.

In the end, we have: $\hat{E}_+ = \{b \xrightarrow{1} a, c \xrightarrow{1} a\}$ and $\hat{E}_- = \{a \xrightarrow{2} b\}$.
In this example, the sets of influences do not contain several influences of the same gene with different thresholds. Furthermore, no influence turns out to be both positive and negative. Hence, the IG inference gives the following extended IG:
$(\{a, b, c\}, \hat{E}_+, \hat{E}_-, \emptyset)$,
which can also be seen as the IG in \pref{fig:runningBRN-ig}, as its fourth component is an empty set.

As a second step, we can now perform the inference of Thomas' parameters, using the previously inferred IG which is well-formed for this task.
For example, in order to determine the parameter $K_{b,\emptyset,\{a\}}$, that is, the focal point of $b$ when $a$ has a negative action on it, we have to determine the value of: $\focals(b,C^b_{b,A,B},C_{b,A,B})$. We have:
\begin{align*}
C^a_{b,A,B} &= \levelsI{a}{b} = \{a_2\} \\
C^b_{b,A,B} &= \PHl_b = \{b_0, b_1\} \\
C^c_{b,A,B} &= \PHl_c = \{c_0, c_1\}
\end{align*}
Therefore: $C_{b,A,B} = \{a_2, b_0, b_1\}$.
As there is no cycle in the bounces of $b$, and given \pref{def:param_K}, it hence comes:
$$K_{b,\emptyset,\{a\}} = \focals(b,C^b_{b,A,B},C_{b,A,B}) = \{b_0\}.$$

\subsection{Implementation}
The inference method described in this paper has been implemented as a tool named \texttt{ph2thomas}. It can be used as a part of \textsc{Pint}\footnote{Available at \url{http://process.hitting.free.fr}}, a library developed to deal with Process Hitting models and perform several operation on them.
Like \textsc{Pint}, \texttt{ph2thomas} uses OCaml to handle data and create ASP programs that are solved using Clingo\footnote{Available at \url{http://potassco.sourceforge.net/index.html}, as part of the Potassco collection}.
The whole process is divided into four parts: cooperative sorts differentiation, IG inference, parameters inference and admissible Parametrizations enumeration; all of them involve an ASP part for the resolution.
Once the \textsc{Pint} library downloaded and compiled, one can perform IG and parameters inference by using the \texttt{ph2thomas} tool with the command:
\begin{itemize}
  \item[] \texttt{ph2thomas -i model.ph -{}-dot ig.dot}
\end{itemize}
where \texttt{model.ph} is the PH model in \textsc{Pint} format and \texttt{ig.dot} is an output of the inferred IG in DOT format.
The (possibly partial) inferred Parametrization will be returned on the standard output.
The admissible Parametrizations enumeration can also been performed by adding the \texttt{-{}-enumerate} parameter at the end of the command line.

The current implementation can successfully solve the examples given in \pref{fig:runningPH-1} (with an incomplete Parametrization) and \pref{fig:runningPH-2}. It can also handle bigger models such as an ERBB receptor-regulated G1/S transition model \todo{ref? \cite{PMR10-TCSB}?} which contains 20 components, and a T-cells receptor model \todo{name? ref?} which contains 40 components.
The following table gives the execution time of the \texttt{ph2thomas} tool for these models, on a personal laptop (Intel Core 2 Duo CPU T5550 @ $1.83\text{GHz} \times 2$ with 3.9 Gib memory, on an Ubuntu 11.10 64-bits OS).
The first column shows the execution time on the complete models without performing admissible Parametrizations enumeration (as all cooperations are defined, only one admissible Parametrization is found), and the second column shows the execution time with admissible Parametrizations enumeration on the same examples without the cooperations (all cooperative sorts are removed and only independent actions are left, thus leading to numerous admissible Parametrizations).
\begin{center}
\begin{tabular}{c|c||c|c|}
  & Components & Without enumeration & With enumeration
\\ \hline
ERBB transition & 20 & 950 ms & ?? ms
\\ \hline
T-cells receptor & 40 & 700 ms & ?? ms
\\ \hline
\end{tabular}
\end{center}
We note that despite its smaller size in term of components, the ERBB transmission model takes more time to be computed because the biggest cooperative sorts contain more processes (up to 32 processes) than in the T-cells receptor model (up to 8 processes).

\begin{comment}
\begin{figure}[t]
\centering
\includegraphics[height=0.90\textheight]{figs/tcrsig40.png}
\caption{\label{fig:tcrsig40-ig}
The IG returned by \texttt{ph2thomas} when used on the T-cell receptor PH model.
}
\end{figure}
In the case of the T-cell receptor model, the inferred IG obtained using the implementation is given in figure \pref{fig:tcrsig40-ig}.
\todo{Give the whole model in annex?}
\end{comment}



\section{Future work}
\resume{In order to gain some performance in the inference, we may consider several leads. Model reduction by cooperative sorts removal may allow to obtain good results without exhaustive search and with possibly lower complexity. Using the multiplexes semantics would allow to reduce the possible parametrizations and make the cooperations appear in the Interaction Graph. Finally, Other ways of coping with the incomplete cooperations could be found.}

\todo{Cope with cases where cooperations are incomplete}

\todo{Use the multiplexes semantics and infer them for complete cooperations to reduce the number of parameters}

\todo{Other approach: model reduction using projections}


\input{parts/ccl}






\end{document}
