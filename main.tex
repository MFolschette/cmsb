\documentclass{llncs}

\usepackage[english]{babel}
\usepackage[utf8]{inputenc}
\usepackage[T1]{fontenc}

\usepackage{amsmath}  % Maths
\usepackage{amsfonts} % Maths
\usepackage{amssymb}  % Maths
\usepackage{stmaryrd} % Maths (crochets doubles)

\usepackage{url}     % Mise en forme + liens pour URLs
\usepackage{array}   % Tableaux évolués

\usepackage{comment}

\usepackage{prettyref}
\newrefformat{def}{Def.~\ref{#1}}
\newrefformat{fig}{Fig.~\ref{#1}}
\newrefformat{pro}{Property~\ref{#1}}
\newrefformat{lem}{Lemma~\ref{#1}}
\newrefformat{thm}{Theorem~\ref{#1}}
\newrefformat{sec}{Sect.~\ref{#1}}
\newrefformat{ssec}{Subsect.~\ref{#1}}
\newrefformat{suppl}{Appendix~\ref{#1}}
\newrefformat{eq}{Eq.~\eqref{#1}}
\def\pref{\prettyref}

\spnewtheorem*{example*}{Example}{\itshape}{}

\usepackage{tikz}
\newdimen\pgfex
\newdimen\pgfem
\usetikzlibrary{arrows,shapes,shadows,scopes}
\usetikzlibrary{positioning}
\usetikzlibrary{matrix}
\usetikzlibrary{decorations.text}
\usetikzlibrary{decorations.pathmorphing}

% Macros relatives à la traduction de PH avec arcs neutralisants vers PH à k-priorités fixes

% Macros générales
\def\Pint{\textsc{PINT}}

% Notations générales pour PH
\newcommand{\PH}{\mathcal{PH}}
\newcommand{\PHs}{\Sigma}
\newcommand{\PHl}{L}
%\newcommand{\PHp}{\mathcal{P}}
\newcommand{\PHp}{\textcolor{red}{\mathcal{P}}}
\newcommand{\PHproc}{\mathcal{P}}
\newcommand{\PHa}{\mathcal{H}}
\newcommand{\PHn}{\mathcal{N}}

\newcommand{\PHhitter}{\mathsf{hitter}}
\newcommand{\PHtarget}{\mathsf{target}}
\newcommand{\PHbounce}{\mathsf{bounce}}
\newcommand{\PHsort}{\Sigma}

\def\f#1{\mathsf{#1}}
\def\focals{\f{focals}}
\def\play{\cdot}

%\newcommand{\PHfrappeR}{\textcolor{red}{\rightarrow}}
%\newcommand{\PHmonte}{\textcolor{red}{\Rsh}}

\newcommand{\PHfrappeA}{\rightarrow}
\newcommand{\PHfrappeB}{\Rsh}
%\newcommand{\PHfrappe}[3]{\mbox{$#1\PHfrappeA#2\PHfrappeB#3$}}
%\newcommand{\PHfrappebond}[2]{\mbox{$#1\PHfrappeB#2$}}
\newcommand{\PHfrappe}[3]{#1\PHfrappeA#2\PHfrappeB#3}
\newcommand{\PHfrappebond}[2]{#1\PHfrappeB#2}
\newcommand{\PHobjectif}[2]{\mbox{$#1\PHfrappeB^*\!#2$}}
\newcommand{\PHconcat}{::}
\newcommand{\PHneutralise}{\rtimes}

\def\PHget#1#2{{#1[#2]}}
%\newcommand{\PHchange}[2]{#1\langle #2 \rangle}
\newcommand{\PHchange}[2]{(#1 \Lleftarrow #2)}
\newcommand{\PHarcn}[2]{\mbox{$#1\PHneutralise#2$}}
\newcommand{\PHjoue}{\cdot}

\newcommand{\PHetat}[1]{\mbox{$\langle #1 \rangle$}}


% Notations spécifiques à ce papier
\newcommand{\PHdirectpredec}[1]{\PHs^{-1}(#1)}
\newcommand{\PHpredec}[1]{\f{pred}(#1)}
\newcommand{\PHpredecgene}[1]{\f{reg}({#1})}
\newcommand{\PHpredeccs}[1]{\PHpredec{#1} \setminus \Gamma}

\newcommand{\PHincl}[2]{#1 :: #2}

\def\ctx{\varsigma}
\def\ctxOverride{\Cap}
\def\state#1{\langle #1 \rangle}

% Notations spécifiques aux graphes d'états
%\newcommand{\PHge}{\textcolor{red}{\mathcal{GE}}}
%\newcommand{\PHt}{\mathcal{T}}
%\newcommand{\GE}{\mathcal{GE}}
%\newcommand{\GEt}{\mathcal{T}}
%\newcommand{\GEl}{\PHl}
%\newcommand{\GEa}{\PHa}
%\newcommand{\GEva}[3]{#1 \stackrel{#2}{\longrightarrow} #3}
%\newcommand{\GEval}[3]{#1 \stackrel{#2}{\Longrightarrow} #3}
%\newcommand{\GEget}[2]{\PHget{#1}{#2}}


\def\DEF{\stackrel{\Delta}=}
\def\EQDEF{\stackrel{\Delta}\Leftrightarrow}
\def\IG{\mathrm{IG}}


\usepackage{ifthen}
\usepackage{tikz}
\usetikzlibrary{arrows,shapes}

\definecolor{lightgray}{rgb}{0.8,0.8,0.8}
\definecolor{lightgrey}{rgb}{0.8,0.8,0.8}

\tikzstyle{boxed ph}=[]
\tikzstyle{sort}=[fill=lightgray,rounded corners]
\tikzstyle{process}=[circle,draw,minimum size=15pt,fill=white,
font=\footnotesize,inner sep=1pt]
\tikzstyle{black process}=[process, fill=black,text=white, font=\bfseries]
\tikzstyle{gray process}=[process, draw=black, fill=lightgray]
\tikzstyle{current process}=[process, draw=black, fill=lightgray]
\tikzstyle{process box}=[white,draw=black,rounded corners]
\tikzstyle{tick label}=[font=\footnotesize]
\tikzstyle{tick}=[black,-]%,densely dotted]
\tikzstyle{hit}=[->,>=angle 45]
\tikzstyle{selfhit}=[min distance=30pt,curve to]
\tikzstyle{bounce}=[densely dotted,->,>=latex]
\tikzstyle{hl}=[font=\bfseries,very thick]
\tikzstyle{hl2}=[hl]
\tikzstyle{nohl}=[font=\normalfont,thin]

\newcommand{\currentScope}{}
\newcommand{\currentSort}{}
\newcommand{\currentSortLabel}{}
\newcommand{\currentAlign}{}
\newcommand{\currentSize}{}

\newcounter{la}
\newcommand{\TSetSortLabel}[2]{
  \expandafter\repcommand\expandafter{\csname TUserSort@#1\endcsname}{#2}
}
\newcommand{\TSort}[4]{
  \renewcommand{\currentScope}{#1}
  \renewcommand{\currentSort}{#2}
  \renewcommand{\currentSize}{#3}
  \renewcommand{\currentAlign}{#4}
  \ifcsname TUserSort@\currentSort\endcsname
    \renewcommand{\currentSortLabel}{\csname TUserSort@\currentSort\endcsname}
  \else
    \renewcommand{\currentSortLabel}{\currentSort}
  \fi
  \begin{scope}[shift={\currentScope}]
  \ifthenelse{\equal{\currentAlign}{l}}{
    \filldraw[process box] (-0.5,-0.5) rectangle (0.5,\currentSize-0.5);
    \node[sort] at (-0.2,\currentSize-0.4) {\currentSortLabel};
   }{\ifthenelse{\equal{\currentAlign}{r}}{
     \filldraw[process box] (-0.5,-0.5) rectangle (0.5,\currentSize-0.5);
     \node[sort] at (0.2,\currentSize-0.4) {\currentSortLabel};
   }{
    \filldraw[process box] (-0.5,-0.5) rectangle (\currentSize-0.5,0.5);
    \ifthenelse{\equal{\currentAlign}{t}}{
      \node[sort,anchor=east] at (-0.3,0.2) {\currentSortLabel};
    }{
      \node[sort] at (-0.6,-0.2) {\currentSortLabel};
    }
   }}
  \setcounter{la}{\currentSize}
  \addtocounter{la}{-1}
  \foreach \i in {0,...,\value{la}} {
    \TProc{\i}
  }
  \end{scope}
}

\newcommand{\TTickProc}[2]{ % pos, label
  \ifthenelse{\equal{\currentAlign}{l}}{
    \draw[tick] (-0.6,#1) -- (-0.4,#1);
    \node[tick label, anchor=east] at (-0.55,#1) {#2};
   }{\ifthenelse{\equal{\currentAlign}{r}}{
    \draw[tick] (0.6,#1) -- (0.4,#1);
    \node[tick label, anchor=west] at (0.55,#1) {#2};
   }{
    \ifthenelse{\equal{\currentAlign}{t}}{
      \draw[tick] (#1,0.6) -- (#1,0.4);
      \node[tick label, anchor=south] at (#1,0.55) {#2};
    }{
      \draw[tick] (#1,-0.6) -- (#1,-0.4);
      \node[tick label, anchor=north] at (#1,-0.55) {#2};
    }
   }}
}
\newcommand{\TSetTick}[3]{
  \expandafter\repcommand\expandafter{\csname TUserTick@#1_#2\endcsname}{#3}
}

\newcommand{\myProc}[3]{
  \ifcsname TUserTick@\currentSort_#1\endcsname
    \TTickProc{#1}{\csname TUserTick@\currentSort_#1\endcsname}
  \else
    \TTickProc{#1}{#1}
  \fi
  \ifthenelse{\equal{\currentAlign}{l}\or\equal{\currentAlign}{r}}{
    \node[#2] (\currentSort_#1) at (0,#1) {#3};
  }{
    \node[#2] (\currentSort_#1) at (#1,0) {#3};
  }
}
\newcommand{\TSetProcStyle}[2]{
  \expandafter\repcommand\expandafter{\csname TUserProcStyle@#1\endcsname}{#2}
}
\newcommand{\TProc}[1]{
  \ifcsname TUserProcStyle@\currentSort_#1\endcsname
    \myProc{#1}{\csname TUserProcStyle@\currentSort_#1\endcsname}{}
  \else
    \myProc{#1}{process}{}
  \fi
}

\newcommand{\repcommand}[2]{
  \providecommand{#1}{#2}
  \renewcommand{#1}{#2}
}
\newcommand{\THit}[5]{
  \path[hit] (#1) edge[#2] (#3#4);
  \expandafter\repcommand\expandafter{\csname TBounce@#3@#5\endcsname}{#4}
}
\newcommand{\TBounce}[4]{
  (#1\csname TBounce@#1@#3\endcsname) edge[#2] (#3#4)
}

\newcommand{\TState}[1]{
  \foreach \proc in {#1} {
    \node[current process] (\proc) at (\proc.center) {};
  }
}

% Macros spécifiques au Modèle de Thomas / aux RRB

% Notations pour le modèle de Thomas (depuis thèse)
\newcommand{\GRN}{\mathcal{GRN}}
\newcommand{\IG}{\mathcal{G}}
%\def\IG{\mathrm{IG}}
\newcommand{\GRNreg}[1]{\PHpredecgene{#1}}
\newcommand{\GRNres}[2]{\mathsf{Res}_{#1}(#2)}
\newcommand{\GRNget}[2]{\PHget{#1}{#2}}
\newcommand{\GRNetat}[1]{\PHetat{#1}}

\def\levels{\mathsf{levels}}
\def\levelsA#1#2{\levels_+(#1\rightarrow #2)}
\def\levelsI#1#2{\levels_-(#1\rightarrow #2)}
%\newcommand{\PHres}{\mathsf{Res}}

\newcommand{\Kinconnu}{\emptyset}
\newcommand{\RRGva}[3]{#1 \stackrel{#2}{\longrightarrow} #3}
\newcommand{\RRGgi}{\mathcal{G}}
\newcommand{\RRGreg}[1]{\RRGgi_{#1}}



%\definecolor{darkred}{rgb}{0.5,0,0}
%\definecolor{lightred}{rgb}{1,0.8,0.8}
%\definecolor{lightgreen}{rgb}{0.7,1,0.7}
\definecolor{darkgreen}{rgb}{0,0.5,0}
%\definecolor{darkyellow}{rgb}{0.5,0.5,0}
%\definecolor{lightyellow}{rgb}{1,1,0.6}
%\definecolor{darkcyan}{rgb}{0,0.6,0.6}
%\definecolor{darkorange}{rgb}{0.8,0.2,0}

%\definecolor{notsodarkgreen}{rgb}{0,0.7,0}

%\definecolor{coloract}{rgb}{0,1,0}
%\definecolor{colorinh}{rgb}{1,0,0}
\colorlet{coloract}{darkgreen}
\colorlet{colorinh}{red}
%\colorlet{coloractgray}{lightgreen}
%\colorlet{colorinhgray}{lightred}
%\colorlet{colorinf}{darkgray}
%\colorlet{coloractgray}{lightgreen}
%\colorlet{colorinhgray}{lightred}

%\colorlet{colorgray}{lightgray}


\tikzstyle{grn}=[every node/.style={circle,draw=black,outer sep=2pt,minimum
                size=15pt,text=black}, node distance=1.5cm]
\tikzstyle{act}=[->,draw=black,thick,color=black]
\tikzstyle{inh}=[>=|,-|,draw=black,thick, text=black,label]
%\tikzstyle{inh}=[>=|,-|,draw=colorinh,thick, text=black,label]
%\tikzstyle{act}=[->,>=triangle 60,draw=coloract,thick,color=coloract]
%\tikzstyle{inhgray}=[>=|,-|,draw=colorinhgray,thick, text=black,label]
%\tikzstyle{actgray}=[->,>=triangle 60,draw=coloractgray,thick,color=coloractgray]
\tikzstyle{inf}=[->,draw=colorinf,thick,color=colorinf]
%\tikzstyle{elabel}=[fill=none, above=-1pt, sloped,text=black, minimum size=10pt, outer sep=0, font=\scriptsize,draw=none]
\tikzstyle{elabel}=[fill=none,text=black, above=-2pt,%sloped,
minimum size=10pt, outer sep=0, font=\scriptsize, draw=none]
%\tikzstyle{elabel}=[]
\tikzstyle{sg}=[every node/.style={outer sep=2pt,minimum
                size=15pt,text=black}, node distance=2cm]



% Commandes À FAIRE
\usepackage{color} % Couleurs du texte
\newcommand{\afaire}[1]{\textcolor{red}{[À FAIRE~: #1]}}
\newcommand{\resume}[1]{\textcolor{blue}{#1}}
\newcommand{\todo}[1]{\textcolor{darkgreen}{[#1]}}


% Id est
\newcommand{\ie}{\textit{i.e.} }


\title{Concretizing the Process Hitting into Biological Regulatory Networks}

\author{Maxime Folschette\inst{1,2}, Lo\"ic Paulev\'e\inst{3}, Katsumi Inoue\inst{2}, Morgan Magnin\inst{1}, Olivier Roux\inst{1}}

\institute{
LUNAM Universit\'e, \'Ecole Centrale de Nantes, IRCCyN UMR CNRS 6597\\
(Institut de Recherche en Communications et Cybern\'etique de Nantes)\\
1 rue de la No\"e - B.P. 92101 - 44321 Nantes Cedex 3, France.\\
\email{Maxime.Folschette@irccyn.ec-nantes.fr}
\and
National Institute of Informatics,\\
2-1-2, Hitotsubashi, Chiyoda-ku, Tokyo 101-8430, Japan.
\and
LIX, \'Ecole Polytechnique, 91128 Palaiseau Cedex, France.
}


\begin{document}

\maketitle

\begin{abstract}
The Process Hitting (PH) is a recently introduced framework to model concurrent processes.
Its major originality lies in a specific restriction on the causality of actions, which  makes
tractable the formal analysis of very large systems.
The PH is suitable to model Biological Regulatory Networks (BRNs) with complete or partial
knowledge of cooperations between regulators by defining the most permissive dynamics
w.r.t. to these constraints.

On the other hand, the qualitative modeling of BRNs has been widely addressed using Ren\'e Thomas'
formalism, leading to numerous theoretical work and practical tools to understand emerging behaviours.

Given a PH model of a BRN, we first tackle the inference of the underlying Interaction Graph
between components.
Then the inference of corresponding Thomas' models is provided using Answer-Set Programming,
which allows notably an efficient enumeration of (possibly numerous) compatible parametrizations.

In addition to giving a formal link between different approaches for qualitative BRNs modeling, 
this work emphasises the ability of PH to deal with large BRNs with incomplete knowledge on
cooperations, where Thomas' approach fails because of the combinatoric of parameters.
\end{abstract}


\section{Introduction}
As regulatory phenomena play a crucial role in biological systems and they need to be studied accurately.
Biological Regulatory Networks (BRNs) consist in sets of either positive or negative mutual effects between the components.
With the purpose of analyzing these systems, they are often modeled as graphs which make it possible to determine the possible evolutions of all the interacting components of the system.
Indeed, besides continuous models of physicians, often designed through systems of ordinary
differential equations, a discrete modeling approach was initiated by René Thomas in 1973
\cite{Thomas73}.

In this approach, the different levels of a component (concentration, expression level, \ldots) are abstractly represented by (positive) integer values and transitions between these levels may be considered as instantaneous.
Hence, qualitative state graphs may be derived from which we are able to formally find out all the possible behaviors expressed as sequences of transitions between these states.
Nevertheless, these dynamics can be precisely established only with regard to some discrete parameters which stand for a kind of "focal points", i.e. the evolutionary tendency from each state and depending of the set of resources in this very state, that is, the set of the other currently interacting components.
Hereafter, we refer to these discrete parameters as Thomas' parameters.

Thomas' modeling has motivated numerous work around the link between the Interaction Graph (IG)
(summarizing the global influences between components) and the possible dynamics (e.g.,
\cite{RiCo07,RRT08}),
model reduction (e.g., \cite{Naldi09}), formal checking of dynamics (e.g., \cite{Richard06,Naldi07}), 
and the incorporation of time (e.g., \cite{Siebert06,Ahmad08}) and probability
(e.g., \cite{Twardziok10-CMSB}) dimensions, to name but a few.
While the formal checking of dynamical properties is often limited to small networks because of the
state graph explosion, the main drawback of this framework is the difficulty to specify the Thomas'
parameters, especially for large networks.

In order to address the formal checking of dynamical properties within very large BRNs, we recently
introduced in \cite{PMR10-TCSB} a new formalism, named the \emph{``Process Hitting''} (PH), to model
concurrent systems having components with a few qualitative levels.
A PH describes, in an atomic manner, the possible evolutions of a process (representing one
component at one level) triggered by the hit of at most one other process in the system.
This framework can be seen as a special class of formalisms like Petri Nets or Communicating Finite
State Machines, where the causality between actions is restricted.
Thanks to the particular structure of interactions within a PH, very efficient static analysis
methods have been developed to over- and under-approximate reachability properties making tractable
the formal analysis of BRNs with hundreds of components \cite{PMR12-MSCS}.

The PH is suitable to model BRNs with different levels of abstraction in the specification of
cooperations (associated influences) between components.
This allows to model BRNs with a partial knowledge on precise evolution functions of components
by capturing the largest (the more general) dynamics.

As a matter of fact, the objectives of the work presented in this paper are the following.
First, we show that starting from one Process Hitting (PH), it is possible to find
back the underlying IG.
We perform an exhaustive search for the possible interactions on one component from all the
others, consistently with the knowledge of the dynamics that these interactions lead to and that are
expressed in PH.
The second phase of our work concerns the Thomas' parameters inference.
It consists in determining the nesting set (possibly too large) of the parameters which necessarily lead to the satisfaction of the known cooperating constraints.
These enumerative searches are made tractable thanks to the use of the Answer-Set Programming (ASP) method \cite{Baral03}.

The outcome of this work is twofold.
The first benefit is that such an approach makes it possible to refine the construction of
BRNs with a partial and progressively brought knowledge in PH, while being able to export such
models in the Thomas' framework.
The second feature of our method is that it can be applied on very large BRNs.

Finally, it must be noticed that we are not interested in this paper in the derivation of one
PH from a BRN (which was previously described in \cite{PMR10-TCSB}) but, on the contrary, to finding out
a set of BRNs from one PH.

Our work is related to the approach of \cite{20646302,DBLP:conf/ipcat/CorblinFTCT12} which also uses constraint programming to determine a class of models which are consistent with available partial data on the regulatory structure and dynamical properties.
These classes are built in order to infer properties common to all some studied models.
In our approach, we intend to focus on the Thomas' parameters inference and we claim we are able to deal with larger biological networks.

\paragraph{Outline.}
\pref{sec:frameworks} recalls the PH and Thomas frameworks;
\pref{sec:infer-IG} defines the IG inference from PH;
\pref{sec:infer-K} details the enumeration of Thomas parametrizations compatible with a PH
and discuss its implementation in ASP.
\pref{sec:examples} illustrates the applicability of our method on simple examples
and large biological models.

\paragraph{Notations.}
$[i;j]$ is the set of integers $\{ i, i+1, \dots, j \}$;
we note $[i_1;j_1] \leq_{[]} [i_2;j_2] \EQDEF (i_1\leq j_1\wedge i_2\leq j_2)$
and $[i_1;j_1] <_{[]} [i_2;j_2] \EQDEF 
%	[i_1;j_1] \leq_{[]} [i_2;j_2] \wedge (i_1\neq i_2 \vee j_1\neq j_2)$.
(i_1<i_2 \wedge j_1\leq j_2) \vee (i_1\leq i_2\wedge j_1 < j_2)$.


\section{Frameworks}
\subsection{The Process Hitting framework}
\resume{The Process Hitting is a new framework that allows to model dynamic systems with an atomistic point of view. In this subsection, we present the definitions of this modeling and mention how static analysis makes it efficient to study large systems. We finally remind the way to translate a BRN to a PH using cooperative sorts.}

\todo{Definition: Process Hitting + dynamics + formalism used after in the article}

\todo{Give a simple example (similar to GRN subsection?)}

\todo{Mention static analysis (fixed points \& reachability)}

\todo{Definition: Generalized dynamics}

\todo{Refining with cooperative sorts}

\subsection{Thomas' modeling}
\resume{The Thomas' modeling is the historical and widely-used model for the study of dynamical gene systems, and takes the form of a Biological Regulatory Network. In this subsection, we present this tool by defining the Interaction Graph and the Parametrization. We also justify our extension of the latter to use interval parameters instead of integers.}

\todo{dire que l'on s'inspire de \cite{Richard06,BernotSemBRN}}
The modeling of a BRN using René Thomas' formalism lies on two complementary sets of information about the system. First, the \emph{Interaction Graph} (IG) models the structure of the system by defining the components' properties and their mutual influences. The \emph{Parametrization} then allows to restrict the dynamic behavior of the system by allowing specific strengths to the influences.

As for PH, IG represents information about a finite number of \emph{components} allowed to take a value amongst a finite number of possible \emph{expression levels}.
For the sake of simplicity and to establish a parallel with PH, if $a$ represents a component, we call $a_i$ its $i^\text{th}$ expression level.
The IG is therefore composed of nodes that represent components, and edges labeled with a threshold that stand for interactions and can be either positive or negative.
For such an interaction to take place, the expression level of its head component has to be higher than its threshold; otherwise, the opposite influence is expressed.
Therefore, for any component $b$, a predecessor $a$ of $b$ such that we have $a \xrightarrow{t} b$ can be either an activator or an inhibitor of $a$, according to the type of interaction involved and if the expression level of $b$ if above of below the related threshold $t$.
We call $\levelsA{a}{b}$ (resp. $\levelsI{a}{b}$) the expression levels of $a$ where it is an activator (resp. inhibitor) of $b$.

\begin{definition}[Interaction Graph]
\label{def:ig}
An \emph{Interaction Graph} (IG) is a triple $(\Gamma, E_+, E_-)$ where $\Gamma$ is a finite number of \emph{components},
and $E_+$ (resp. $E_-$) $\subset \{a \xrightarrow{t} b \mid a, b \in \Gamma \wedge t \in \mathbb{N}\}$
is the set of positive (resp. negative) \emph{regulations} between two nodes, labeled with a \emph{threshold}.

A regulation from $a$ to $b$ is uniquely referenced:
if $a \xrightarrow{t} b \in E_+$ (resp. $E_-$),
$\nexists a \xrightarrow{t'} b \in E_+ \text{ (resp. $E_-$)}, t \neq t'$
and $\nexists t', a \xrightarrow{t'} b \in E_-$ (resp. $E_+$).
\end{definition}

\begin{definition}[Effective levels ($\levels$)]\label{def:levels}
Let $(\Gamma,E_+,E_-)$ be an IG and $a, b \in \Gamma$ two of its components:
\begin{itemize}
  \item if $a \xrightarrow{t} b \in E_+$, $\levelsA{a}{b} = [t; l_a]$ and
    $\levelsI{a}{b} = [0; t-1]$;
  \item if $a \xrightarrow{t} b \in E_-$, $\levelsA{a}{b} = [0; t-1]$ and
    $\levelsI{a}{b} = [t; l_a]$;
  \item otherwise, $\levelsA{a}{b} = \levelsI{a}{b} = \emptyset$.
\end{itemize}
\end{definition}

For all component $a \in \Gamma$, we also denote $\GRNreg{a} = \{b \in \Gamma \mid \exists b \xrightarrow{t} a \in E_+ \cup E_-\}$ the set of its regulators,
and $E_+(a) = \{b \in \Gamma \mid \exists b \xrightarrow{t} a \in E_+\}$ (resp. $E_-(a) = \{b \in \Gamma \mid \exists b \xrightarrow{t} a \in E_-\}$) the set of its positive (resp. negative) regulators.
\todo{Check if it is a good idea to use the same symbol $\f{Reg}$ for both BRN and PH.}

\begin{example*}
\pref{fig:runningBRN-ig} represents an Interaction Graph $(\Gamma,E_+,E_-)$ with
$\Gamma = \{a, b, c\}$,
$E_+ = \{b \xrightarrow{1} a, c \xrightarrow{1} a\}$ and
$E_- = \{a \xrightarrow{2} b\}$.
Furthermore, we have: $E_+(a) = \PHpredecgene{a} = \{b, c\}$ and $E_-(a) = \emptyset$.
This IG can represent the same behavior as the PH given in \pref{fig:runningPH-2}.

\begin{figure}[t]
\centering
\scalebox{1.5}{
\begin{tikzpicture}[grn]
\path[use as bounding box] (-0.5,-0.75) rectangle (4.5,0.7);
\node[inner sep=0] (a) at (2,0) {a};
\node[inner sep=0] (b) at (0,0) {b};
\node[inner sep=0] (c) at (4,0) {c};
\path
  node[elabel, below=-1em of a] {$0..2$}
  node[elabel, below=-1em of b] {$0..1$}
  node[elabel, below=-1em of c] {$0..1$};
\path
  (b) edge[act, bend right] node[elabel, below=-2pt] {$+1$} (a)
  (c) edge[act] node[elabel, above=-2pt] {$+1$} (a)
  (a) edge[inh, bend right] node[elabel, above=-5pt] {$-2$} (b);
\end{tikzpicture}
}
\caption{\label{fig:runningBRN-ig}
An Interaction Graph example.
Components are represented by nodes labeled with a name and possible expression levels.
Regulations are represented by the edges, labeled with a sign that stands for their type ($+$ for positive and $-$ for negative) and a threshold.
For instance, the edge from $b$ to $a$ is labeled $+1$, which stands for: $b \xrightarrow{1} a \in E_+$,
and means that if the level of expression of $b$ is equal to (i.e. above) 1, then $b$ activates $a$,
otherwise, $b$ inhibits $a$.
}
\end{figure}
\end{example*}

A \emph{state} of an IG $(\Gamma, E_+, E_-)$ is an element in $\prod_{a \in \Gamma} \{a_0, \dots, a_{l_a}\}$.
The specificity of René Thomas' approach lies in the use of discrete \emph{parameters} to represent the focal point towards which the expression level of a component will evolve in each configuration of its regulators.
Indeed, for each possible state of a BRN, all regulators of a gene are be divided into \emph{activators} and \emph{inhibitors}, given their type of interaction and expression level.
The direction of evolution of a gene thus depends on these \emph{resources} in the considered state.
In this paper, we extend the classical definition focal point from a unique integer to an interval of integers.
\todo{The interval semantic is more expressive: justify!}
The association of an IG and a related Parametrization corresponds to the definition a \emph{BRN using René Thomas' parameters}, which entirely describes the structure and allows to compute the dynamics of the modeled system.
\todo{Find another name for “BRN using René Thomas' parameters”?}

\begin{definition}[Discrete parameter $K_{a,A,B}$ and Parametrization $K$]\label{def:param}
For a given component $a \in \Gamma$ and $A$ (resp. $B$) $\subset \GRNreg{a}$ a set of activators (resp. inhibitors) of $a$ such that
$A \cup B = \GRNreg{a}$ and $A \cap B = \emptyset$,
we define the discrete \emph{parameter} $K_{a,A,B} = [i_1; i_2]$ as a non-empty interval towards which the component $a$ will tend
in the states where its activators (resp. inhibitors) are the regulators in set $A$ (resp. $B$).
A complete map $K$ of discrete parameters on an IG $\IG$ is called a \emph{parametrization} of $\IG$.
\end{definition}
A consequence of this definition is that $0 \leq i_1 \leq i_2 \leq l_a$.
%We also denote: $j < K_{a,A,B} \Leftrightarrow j < i_1$ and $j > K_{a,A,B} \Leftrightarrow j> i_2$.

\begin{definition}[Resources $\GRNres{a}{s}$]\label{def:resources}
For a given state $s$ of a BRN, we define the \emph{activators} and \emph{inhibitors} of $a$ in $s$ as $\GRNres{a}{s} = A,B$, where:
\begin{align*}
  A &= \{b \in \Gamma \mid \GRNget{s}{b} \in \levelsA{b}{a}\} \\
  B &= \{b \in \Gamma \mid \GRNget{s}{b} \in \levelsI{b}{a}\}
\end{align*}
\end{definition}

\begin{definition}[Biological Regulatory Network using Thomas' parameters]\label{def:brn}
A \emph{biological regulatory network (BRN) using Thomas' parameters} is a pair $(\IG; K)$ where the first entry is an Interaction Graph and the second is a complete Parametrization.
\end{definition}
We call $s$ a state of a BRN using Thomas' parameters if $s$ is a state of the IG $\IG$.

\begin{example*}
In order to achieve the very same behavior as the PH given in \pref{fig:runningPH-2}, we can use the Parametrization of the IG of \pref{fig:runningBRN-ig} given in \pref{fig:runningBRN-param}.

\begin{figure}[t]
\begin{align*}
K_{a,\{b,c\},\emptyset} &= [2 ; 2] & K_{b,\{a\},\emptyset} &= [0 ; 1] \\
K_{a,\{b\},\{c\}} &= [1 ; 1] & K_{b,\emptyset,\{a\}} &= [0 ; 0] \\
K_{a,\{c\},\{b\}} &= [1 ; 1] &&\\
K_{a,\emptyset,\{b,c\}} &= [0 ; 0] & K_{c,\emptyset,\emptyset} &= [0 ; 1]
\end{align*}
\caption{\label{fig:runningBRN-param}
An possible Parametrization of the IG of \pref{fig:runningBRN-ig}.
This Parametrization ensures the same behavior as the PH given in \pref{fig:runningPH-2}.
}
\end{figure}
\end{example*}

At last, we describe the asynchronous dynamics of a BRN using Thomas' parameters.
From a given state $s$, a transition to another state $s'$ is possible provided that only one component $a$ will evolve of one expression level towards the nearest value of $K_{a,\GRNres{a}{s}}$.

\begin{definition}[Asynchronous dynamics]\label{def:dynamics}
Let $s$ be a state of a BRN using Thomas' parameters $(\IG, K)$ where $\IG = (\Gamma, E_+, E_-)$.
The state that succeeds to $s$ is given by the indeterministic function $f(s)$:
\begin{align*}
  & f(s) = s' \Longrightarrow \exists a \in \Gamma,
    \GRNget{s'}{a} = f^a(s) \wedge
    \forall b \in \Gamma, b \neq a, \GRNget{s}{b} = \GRNget{s'}{b}
    \quad\text{, with}\\
  & f^a(s) =
  \begin{cases}
    \GRNget{s}{a} + 1 & \text{if } \GRNget{s}{a} < K_{a, \GRNres{a}{s}} \\
    \GRNget{s}{a} & \text{if } \GRNget{s}{a} \in K_{a,\GRNres{a}{s}}\\
    \GRNget{s}{a} - 1 & \text{if } \GRNget{s}{a} > K_{a,\GRNres{a}{s}}
  \end{cases}
\end{align*}
\end{definition}

\begin{example*}
In the BRN that consists of the IG in \pref{fig:runningBRN-ig} and the Parametrization in \pref{fig:runningBRN-param}, the following transitions are possible given the dynamics defined in \pref{def:dynamics}:
\[\GRNetat{a_0, b_1, c_1} \rightarrow \GRNetat{a_1, b_1, c_1} \rightarrow \GRNetat{a_2, b_1, c_1} \rightarrow
\GRNetat{a_2, b_0, c_1} \rightarrow \GRNetat{a_1, b_0, c_1}.\]
\end{example*}


\section{Interaction Graph Inference}\label{sec:infer-IG}

In order to infer a complete BRN, one has to find the Interaction Graph (IG) first, as some
constraints on the parametrization rely on it.
Inferring the IG is an abstraction step which consists in determining the global influence of
components on each of its successors.

This section introduces first the notion of focal processes within a PH
(\pref{ssec:focal}) which is used to characterize well-formed PH for IG inference
in \pref{ssec:wf}, and as well used by the parametrization inference presented in \pref{sec:infer-K}.
Finally, the rules for inferring the interactions between components from a PH are
described in \pref{ssec:infer-IG}.
We consider hereafter a global PH $(\PHs,\PHl,\PHa)$ on which the IG inference is to be
performed.

\subsection{Focal Processes}\label{ssec:focal}

Many of the inferences defined in the rest of this paper rely on the knowledge of \emph{focal
processes} w.r.t. a given context (a set of processes that are potentially present).
When such a context applies, we expect to (always) reach one focal process in a bounded number of
actions.

For $S_a\subseteq L_a$ and a context (set of processes) $\sigma$, let us define as $\PHa(S_a,\ctx)$
the set of actions on the sort $a$ having their hitter in $\ctx$ and target in $S_a$
(\pref{eq:PHa-ctx});
and the digraph $(V, E)$ where arcs are the bounces within the sort $a$ triggered by actions
in $\PHa(S_a,\ctx)$ (\pref{eq:bounce-graph}).
$\focals(a,S_a,\ctx)$ denotes the set of focal processes of sort $a$ in the scope of
$\PHa(S_a,\ctx)$ (\pref{def:focals}).
\begin{align}
\PHa(S_a,\ctx) & \DEF \{ \PHfrappe{b_i}{a_j}{a_k}\in\PHa \mid b_i\in\ctx \wedge a_j\in S_a \}
\label{eq:PHa-ctx}
\\
\begin{split}
E  & \DEF \{(a_j,a_k)\in (S_a \times \PHl_a) \mid 
			\exists\PHfrappe{b_i}{a_j}{a_k}\in \PHa(S_a,\ctx) \}
\\
V & \DEF S_a \cup \{ a_k\in L_a\mid \exists (a_j,a_k)\in E\}
\end{split}
\label{eq:bounce-graph}
\end{align}

\begin{definition}[$\focals(a,S_a,\ctx)$]\label{def:focals}
The set of processes that are focal for processes in $S_a$ in the scope of $\PHa(S_a,\ctx)$
are given by:
%$\focals(a,S_a,\ctx)$ is the set of focal processes of sort $a$ in the context $\ctx$:
\[
\focals(a,S_a,\ctx) \DEF
\begin{cases}
\{ a_i \in V \mid \nexists (a_i,a_j)\in E\} & \text{if the digraph $(V,E)$ is acyclic},\\
\emptyset & \text{otherwise.}\\
\end{cases}
\]
\end{definition}

We note $\PHl(\ctx)$ the set of states $s\in L$ such that $\forall a\in\PHsort(\ctx), \PHget{s}{a}\in\ctx$,
where $\PHsort(\ctx)$ is the set of sorts with processes in $\ctx$.
We say a sequence of actions $h^1,\dots,h^n$ is \emph{bounce-wise} if and only if
$\forall m\in[1;n-1], \PHbounce(h^m)=\PHtarget(h^{m+1})$.
From \pref{def:focals}, it derives that:
\begin{enumerate}
\item if $\focals(a,S_a,\ctx)=\emptyset$, there exists a 
state $s\in \PHl(\ctx\cup S_a)$ such that $\forall n\in\mathbb N$ there
exists a bounce-wise sequence of actions $h^1,\dots,h^{n+1}$ in $\PHa(S_a,\ctx)$ 
with $\PHtarget(h^1)\in s$.
\item if $\focals(a,S_a,\ctx)\neq\emptyset$, for all
state $s\in \PHl(\ctx\cup S_a)$,
for any bounce-wise sequence of actions $h^1,\dots,h^n$ in $\PHa(S_a,\ctx)$ where $\PHtarget(h^1)\in
s$,
either
 $\PHbounce(h^n) \in \focals(a,S_a,\ctx)$,
or
$\exists h^{n+1}\in \PHa(a,\ctx)$ such that $\PHbounce(h^n) = \PHtarget(h^{n+1})$.
Moreover $n\leq|\PHa(S_a,\ctx)|$ (i.e. no cycle of actions possible).
\end{enumerate}

It is worth noticing that those bounce-wise sequences of actions may not be successively playable in
a state $s\in L(\ctx\cup S_a)$.
Indeed, nothing impose that the hitters of actions are present in $s$.
In the general case, the playability of those bounce-wise sequence, referred to as \emph{focals
reachability} may be hard to prove.
However, in the scope of this paper, the particular contexts used with $\focals$ ensure this property.
Notably, the rest of this section uses only \emph{strict} contexts (\pref{def:strict-ctx}) which
allow at most one hitter per sort in the bounce-wise sequences (and thus are present in $s$).

\begin{comment}
\begin{property}[$\focals$ reachability]\label{pro:focals-reach}
$\focals(a,S_a,\ctx)$ is reachable if and only if 
$\forall s\in L(\ctx\cup S_a)$,
if $\exists h\in \PHa(S_a,\ctx)$ with $\PHtarget(h)\in s$,
there exists a (possibly empty) sequence of actions
$h^1,\dots,h^n \in \PHa(\ctx\cup S_a)$
	such that $h^1,\dots,h^n,h$ are successively playable in $s$;
where $\PHa(\ctx) \DEF \{ \PHfrappe{b_i}{c_j}{c_k}\in\PHa \mid c\neq a \wedge
			b\in \PHsort(\ctx) \Rightarrow b_i\in \ctx \wedge c\in\PHsort(\ctx) \Rightarrow
			c_i\in\ctx \}$.
\end{property}
\end{comment}

\begin{definition}[Strict context for $S_a$]\label{def:strict-ctx}
A context (set of processes) $\ctx$ is strict for $S_a\subseteq L_a$ if and only if
$\{b_i,b_j\} \subset \ctx \wedge b\neq a \Rightarrow i=j$.
\end{definition}

In other words, assuming focals reachability, if $\focals(a,S_a,\ctx)$ is empty, there exists a
sequence of actions that may be played an unbound number of times (cycle);
if it is non-empty, it is ensured that any state in $\PHl(\ctx\cup S_a)$ converges, in a bounded
number of steps, either to a process in $S_a$ that is not hit by processes in $\ctx$, or to a process in
$L_a\setminus S_a$.

\begin{example*}
In the PH of \pref{fig:runningPH-1}, we obtain:
\begin{align*}
\focals(a,L_a,\{b_0,c_0\}) &= \{ a_0 \}
&
\focals(a,L_a,\{b_1,c_1\}) &= \{ a_2 \}
\\
\focals(a,L_a,\{b_1,c_0\}) &= \emptyset
&
\focals(a,\{a_1\},\{b_1,c_0\}) &= \{ a_0, a_2 \}
\end{align*}
\end{example*}

\subsection{Well-formed Process Hitting for Interaction Graph Inference}\label{ssec:wf}

The inference of an IG from a PH assumes that the PH defines two types of sorts:
the sorts corresponding to BRN components, and the cooperative sorts.
This leads to the characterization of a \emph{well-formed} PH for IG inference.

The identification of sorts modelling components rely on the observation that their processes
represent (ordered) qualitative levels.
Hence an action on such a sort cannot make it bounce to a process at a distance more than one.
The set of sorts satisfying such a condition is referred to as $\Gamma$
(\pref{eq:PH-components}).
Therefore, in the rest of this paper, $\Gamma$ denotes the set of components of the BRN to infer.

\begin{equation}
\Gamma \DEF \{a \in \PHs \mid \nexists \PHfrappe{b_i}{a_j}{a_k} \in \PHa, |j - k| > 1\} \\
\label{eq:PH-components}
\end{equation}

Any sort that does not act as a component should then be treated as a cooperative sort.
As explained in \pref{ssec:PH}, the role of a cooperative sort $\upsilon$ is to compute the current
state of set of cooperating processes.
Hence, for each sub-state $\sigma$ formed by the sorts hitting $\upsilon$, $\upsilon$ should
converge to a focal process.
This is expressed by \pref{pro:wf-cooperative-sort}, where
the set of sorts having an action on a given sort $a$ is given by 
$\PHdirectpredec{a}$ (\pref{eq:ph_direct_predec})
and $\PHproc(\sigma)$ is the set of processes that compose the sub-state $\sigma$.

\begin{equation}
\forall a \in \PHs, \PHdirectpredec{a} \DEF \{b \in \PHs \mid \exists \PHfrappe{b_i}{a_j}{a_k}\in\PHa \}
\label{eq:ph_direct_predec}
\end{equation}

\begin{property}[Well-formed cooperative sort]\label{pro:wf-cooperative-sort}
A sort $\upsilon\in\PHs$ is a well-formed cooperative sort if and only if
each configuration $\sigma$ of its predecessors leads $\upsilon$ to a unique focal process,
denoted by $\upsilon(\sigma)$:
\[
\forall \sigma \in {\textstyle\prod_{
a\in\PHdirectpredec{\upsilon} \wedge a\neq \upsilon}}
\PHl_{a},
\focals(\upsilon,\PHl_\upsilon,\PHproc(\sigma)\cup \PHl_\upsilon) = \{ \upsilon(\sigma) \}\]
\end{property}

Such a property allows a large variety of definition of a cooperative sort, but
for the sake of simplicity, does not allow the existence of multiple focal processes.
While this may be easily extended to (the condition becomes 
$\focals(\upsilon,\PHl_\upsilon, \PHproc(\sigma)\cup \PHl_\upsilon)\neq\emptyset$), it makes some
hereafter equations a bit more complex to read as they should handle set of focal processes instead
of a unique focal process.


Finally, \pref{pro:wf-ph} sums up the conditions for a Process Hitting to be suitable for IG
inference.
In addition of having either component sorts or well-formed cooperative sorts, we also require that
there is no cycle between cooperative sorts, and that
sorts being never hit (\ie{} serving as an invariant environment) are components.

\begin{property}[Well-formed Process Hitting for IG inference]\label{pro:wf-ph}
A PH is well-formed for IG inference if and only if the following conditions are verified:
\begin{itemize}
\item 
each sort $a\in\PHs$ either belongs to $\Gamma$, or is a well-formed cooperative sort;
\item 
there is no cycle between cooperative sorts
(the digraph $(\Sigma,\{(a,b)\in(\Sigma\times\Sigma)\mid \exists \PHfrappe{a_i}{b_j}{b_k}\in\PHa
\wedge a\neq b\wedge \{a,b\}\cap\Gamma=\emptyset \})$ is
acyclic);
\item 
sorts having no action hitting them belong to $\Gamma$
($\{ a \in \Sigma\mid \nexists \PHfrappe{b_i}{a_j}{a_k}\in\PHa\} \subset \Gamma$).
\end{itemize}
\end{property}

\begin{example*}
In the PH of \pref{fig:runningPH-2}, $bc$ is a well-formed cooperative sort as defined in \pref{pro:wf-cooperative-sort}, because:
\begin{align*}
\focals(bc, \PHl_{bc}, \{b_0, c_0\} \cup \PHl_{bc}) = \{bc_{00}\} && \focals(bc, \PHl_{bc}, \{b_0, c_1\} \cup \PHl_{bc}) = \{bc_{01}\} \\
\focals(bc, \PHl_{bc}, \{b_1, c_0\} \cup \PHl_{bc}) = \{bc_{10}\} && \focals(bc, \PHl_{bc}, \{b_1, c_1\} \cup \PHl_{bc}) = \{bc_{11}\}
\end{align*}
Hence, both \pref{fig:runningPH-1} and \pref{fig:runningPH-2} are well-formed PH for IG inference
with $\Gamma = \{a,b,c\}$.
\end{example*}


\subsection{Interaction Inference}\label{ssec:infer-IG}

At this point we can divide the set of sorts $\PHs$ into components ($\Gamma$) and cooperative sorts
($\PHs \setminus \Gamma$) that will not appear in the IG. 
We define in \pref{eq:ph_predec} the set of predecessors of a sort $a$, that is, the sorts influencing $a$
by considering direct actions and possible intermediate cooperative sorts.
The predecessors of $a$ that are components are the regulators of $a$, denoted $\PHpredecgene{a}$
(\pref{eq:regulators}).
\begin{align}
\begin{split}
\forall a \in \PHs, \PHpredec{a} &\DEF \{b \in \PHs \mid \exists n \in \mathbb{N}^*, \exists
(c^k)_{k \in [0;n]} \in \PHs^{n+1}, \\
                                   & \quad \quad c^0 = b \wedge c^n = a \\
                                   & \quad \quad \wedge \forall k \in \llbracket 0 ; n-1 \rrbracket,
								   c^k \in \PHdirectpredec{c^{k+1}} \cap (\PHs\setminus\Gamma)\}
\end{split}
\label{eq:ph_predec}
\\
\forall a\in \PHs, \PHpredecgene{a} & \DEF \PHpredec{a} \cap \Gamma
\label{eq:regulators}
\end{align}

Given a set $g$ of components and a configuration (\ie a sub-state) $\sigma$, $\ctx_g(\sigma)$
refers to the set of focal processes hitting $a$ regulated by any sort in $g$ (\pref{eq:ctx-sigma}).
If $g=\{b\}$, we simple note $\ctx_b(\sigma)$.
This set is composed of the active processes of sort in $g$, and the focal process (assumed
unique) of the cooperative sorts $\upsilon$ hitting $a$ that have a predecessor in $g$.
The evaluation of the focal process of $\upsilon$ in context $\sigma$, denoted $\upsilon(\sigma)$,
relies on \pref{pro:wf-cooperative-sort}, which gives its value when all the direct predecessors of
$\upsilon$ are defined in $\sigma$.
When a predecessor $\upsilon'$ is not in $\sigma$, we extend the evaluation by recursively computing
the focal value of $\upsilon'$ is $\sigma$, as stated in \pref{eq:cooperative-eval}.
Because there is no cycle between cooperative sorts, this recursive evaluation of $\upsilon(\sigma)$
always terminates.
\begin{align}
\forall g\subset \Gamma,
	\ctx_g(\sigma) & \DEF \{ \sigma[b] \mid b\in g \} \cup \{ \upsilon(\sigma) \mid
\upsilon\in\PHdirectpredec{a} \setminus \Gamma \wedge g\cap \PHpredecgene{\upsilon} \neq \emptyset \}
\label{eq:ctx-sigma}
\\
\upsilon(\sigma) & \DEF
\upsilon(\sigma \uplus \state{\upsilon'(\sigma) \mid 
	\upsilon'\in\PHdirectpredec{\upsilon} \wedge
	\upsilon'\in\PHs\setminus\Gamma })
\label{eq:cooperative-eval}
\end{align}

We aim at inferring that $b$ activates (inhibits) $a$ if there exists a configuration where increasing
the level of $b$ makes possible the increase (decrease) of the level of $a$.
This is analogous to standard IG inferences from discrete maps \cite{RiCo07}.

This reasoning can be straightforwardly applied to PH when inferring the influence of $b$ for $a$
when $b\neq a$ (\pref{eq:edges-inference-b}).
Let us define $\gamma(b\rightarrow a)$ as the set of components cooperating with $b$ to hit $a$,
including $b$ and $a$ (\pref{eq:cooperating-with-b}).
Given a configuration $\sigma\in\prod_{c\in\gamma(b\rightarrow a)} L_c$, 
$\focals(a,\{a_i\},\ctx_b(\sigma))$ gives the bounces that a given process $a_i$ can make in the
context $\ctx_b(\sigma)$.
We note $\sigma\{b_i\}$ the configuration $\sigma$ where the process of sort $b$ has been replaced
by $b_i$.
If there exists $b_i,b_{i+1}\in L_b$ such that one bounce in 
$\focals(a, \{\sigma[a]\}, \ctx_b(\sigma\{b_i\}))$
has a lower (resp. higher) level that one bounce in
$\focals(a, \{\sigma[a]\}, \ctx_b(\sigma\{b_{i+1}\}))$, then
$b$ as positive (resp. negative) influence on $a$ with a maximum threshold $l=j$.
\begin{equation}
\gamma(b\rightarrow a)  \DEF \{ a, b \} \cup \{ c \in \Gamma \mid 
			\exists \upsilon\in\PHs\setminus\Gamma,
				\upsilon\in\PHpredec{a} \wedge \{b,c\}\subset\PHpredec{\upsilon} \}
\label{eq:cooperating-with-b}
\end{equation}


Then, we infer that $a$ has a self-influence if its current level can have an impact on its own
evolution at a given configuration $\sigma$.
We consider here a configuration $\sigma$ of a group $g$ of sorts having a cooperation on $a$.
This set of sort groups is given by $X(a)$ (\pref{eq:influence-groups}) which returns the set of
connected components (noted $\mathcal C$) of the graph linking two regulators $b,c$ of $a$ if there
is a cooperative sort hitting $a$ regulated by both of them.
Given $a_i,a_{i+1}\in L_a$, we pick $a_j\in\focals(a,\{a_i\},\ctx_g(\sigma\{a_i\}))$ and
$a_k\in\focals(a,\{a_{i+1}\},\ctx_g(\sigma\{a_{i+1}\}))$.
If $k=j+1$, we can not conclude as there is no difference in the evolution of both levels.
If $k\neq j+1$ and $k-j\neq 0$, then $a_i$ and $a_{i+1}$ have divergent evolutions: we infer an
influence of sign of $k-j$ at threshold $i+1$.
We note that some aspects of this inference are arbitrary and may impact the number of parameters to
infer in the next section.
In particular, in some cases, the use of intervals for Thomas' parameters drops the requirement of
inferring a self-activation.
%Future work may propose alternative definitions of self-influences inference in order to range over
%different parametrization configurations.

\begin{equation}
X(a) = \mathcal C\left( (\PHpredecgene{a}, \{ \{b,c\} \mid
				\exists \upsilon\in \PHdirectpredec{a} \setminus \Gamma,
					\{b,c\} \subset \PHpredecgene{\upsilon} \}) \right)
\label{eq:influence-groups}
\end{equation}

\pref{pps:inference-edges} details the inference of all existing influences between genes occurring
with a threshold $l$.
These influences are split into positive and negative ones, and represent possible edges in the final IG.
\begin{proposition}[Edges inference]\label{pps:inference-edges}
We define the set of positive (resp. negative) influences $\hat{E}_+$ (resp. $\hat{E}_-$) for any
$a\in\Gamma$ by:
\begin{align}
\begin{split}
\forall b\in\PHpredecgene{a}, b\neq a, \\
b\xrightarrow l a \in \hat{E}_s & \Longleftrightarrow
 \exists \sigma\in\textstyle\prod_{c\in\gamma(b\rightarrow a)} L_c, \exists b_i,b_{i+1}\in \PHl_b,\\
& \qquad\qquad
        \exists a_{j}\in\focals(a, \{\sigma[a]\}, \ctx_b(\sigma\{b_i\})), \\
& \qquad\qquad
        \exists a_{k}\in\focals(a, \{\sigma[a]\}, \ctx_b(\sigma\{b_{i+1}\})), \\
& \qquad\qquad\qquad
                        s = \f{sign}(k - j) \wedge l = i+1
\end{split}
\label{eq:edges-inference-b}
\\
\begin{split}
a \xrightarrow l a \in \hat{E}_s & \Longleftrightarrow
\exists g \in X(a), \sigma \in L_a\times \textstyle\prod_{b\in g} L_b,
			\exists a_i,a_{i+1}\in \PHl_a, \\
& \qquad\qquad
        \exists a_{j}\in\focals(a, \{a_i\}, \ctx_g(\sigma\{a_i\})), \\
& \qquad\qquad
        \exists a_{k}\in\focals(a, \{a_{i+1}\},  \ctx_g(\sigma\{a_{i+1}\})), \\
& \qquad\qquad\quad
			k \neq j+1
				\wedge s = \f{sign}(k - j) \wedge l = i+1
\end{split}
\label{eq:edges-inference-a}
\end{align}
where $s \in \{ +, - \}$, $\bar s = + \EQDEF s = -$, $\bar s = - \EQDEF s = +$,
$\f{sign}(n) = + \EQDEF n > 0$,
$\f{sign}(n) = - \EQDEF n < 0$,
and $\f{sign}(0) \EQDEF 0$.
\end{proposition}

We are now able to infer the edges of the final IG by considering positive and negative influences
(\pref{pps:inference-IG}).
We infer a positive (resp. negative) edge if there exists a corresponding influence with the same
sign. If an influence is both positive and negative, we infer an unsigned edge. In the end, the
threshold of each edge is the minimum threshold for which the influence has been found. As unsigned
edges represent ambiguous interactions, no threshold is inferred.
\begin{proposition}[Interaction Graph inference]\label{pps:inference-IG}
We infer $\IG = (\Gamma,E_+,E_-,E_\pm)$ from \pref{pps:inference-edges} as follows:
\begin{align*}
E_+ &= \{a \xrightarrow{t} b \mid \nexists a \xrightarrow{t'} b \in \hat{E}_-
  \wedge t = \min \{ l \mid a \xrightarrow{l} b \in \hat{E}_+\}\} \\
E_- &= \{a \xrightarrow{t} b \mid \nexists a \xrightarrow{t'} b \in \hat{E}_+
  \wedge t = \min \{l \mid a \xrightarrow{l} b \in \hat{E}_-\}\} \\
E_\pm &= \{a \rightarrow b \mid \exists a \xrightarrow{t} b \in \hat{E}_+ \wedge \exists a \xrightarrow{t'} b \in \hat{E}_-\} \\
\end{align*}
\end{proposition}


\begin{example*}
The IG inference from the PH of \pref{fig:runningPH-2} gives
$\hat{E}_+ = \{b \xrightarrow{1} a, c \xrightarrow{1} a\}$ and 
$\hat{E}_- = \{a \xrightarrow{2} b\}$, corresponding to the IG of \pref{fig:runningBRN}.
No self-influence are inferred ($X(a) = \{ \{b,c\} \}$, $X(b)=\{ \{a\}\}$, and $X(c)=\emptyset$).
\end{example*}

\section{Parametrization inference}\label{sec:infer-K}

Given the IG inferred from a PH as presented in the previous section, one can find the discrete parameters that model the behavior of the studied PH using the method presented in the following.
It relies on an exhaustive enumeration of all predecessors of each component in order to find attractor processes and returns a possibly incomplete parametrization, given the exhaustiveness of the cooperations.
The last step consists of the enumeration of all compatible complete parametrizations, given this
set of inferred parameters, the PH dynamics, and some biological constraints on parameters.

\subsection{Parameters inference}

This subsection presents some results related to the inference of independent discrete parameters from a given PH. These results are equivalent to those presented in \cite{PMR10-TCSB}, with notation adapted to be shared with the previous section.
In addition, we introduce the well-formed PH for parameter inference property (\pref{pro:wf-ph-K}),
which imposes that the inferred IG contains no unsigned interactions, and thus can be seen as the
regular IG $(\Gamma, E_+, E_-)$,
and that any processes in $\levelsA{b}{a}$ (resp. $\levelsI{b}{a}$) share the same behaviour
regarding $a$.
\begin{property}[Well-formed PH for parameter inference]\label{pro:wf-ph-K}
A PH is well-formed for parameter inference if and only if
the IG $(\Gamma, E_+, E_-, E_\pm)$ inferred by \pref{pps:inference-IG}
verifies $E_\pm=\emptyset$ and if the following property holds:
\begin{align}
\begin{split}
\forall b\in \GRNreg{a} &,
	\forall (i,j\in\levelsA{b}{a} \vee i,j\in\levelsI{b}{a}), \\
& \qquad	\forall c, ( (b\neq a\wedge c=a) \vee (c\in\PHpredec{a} \wedge b\in\PHdirectpredec{c})), \\
& \qquad \qquad
			\PHfrappe{b_i}{c_k}{c_l}\in\PHa \Leftrightarrow
				\PHfrappe{b_j}{c_k}{c_l}\in\PHa
\end{split}
\label{eq:wf-levels}
\end{align}
\end{property}

Let $K_{a,A,B}$ be the parameter we want to infer for a given component $a \in \Gamma$
%and $A,B \in \GRNallres{a}$ a configuration of resources of $a$ (activators and inhibitors).
and $A \subset \GRNreg{a}$ (resp. $B \subset \GRNreg{a}$) a set of its activators (resp. inhibitors).
This inference, as for the IG inference, relies on the search of focal processes of the component for the given configuration of its regulators.

For each sort $b \in \GRNreg{a}$, we define a context $C^b_{a,A,B}$ in \pref{eq:param_context} that contains all processes representing the influence of the regulators in the configuration $A,B$.
The context of a cooperative sort $\upsilon$ that regulates $a$ is given in
\pref{eq:param_context_coop} as the set of focal processes matching the current configuration.
$C_{a,A,B}$ refers to the union of all these contexts (\pref{eq:K-ctx}).
\begin{align}
\label{eq:param_context}
\forall b\in\Gamma,~
C_{a,A,B}^b & \DEF \begin{cases}
\levelsA{b}{a} & \text{if $b \in A$,}\\
\levelsI{b}{a} & \text{if $b \in B$,}\\
L_b		& \text{otherwise;}\\
\end{cases}
\\
\label{eq:param_context_coop}
\forall \upsilon \in \PHpredec{a}\setminus\Gamma,~
C_{a,A,B}^\upsilon & \DEF \{
\upsilon(\sigma) \mid \sigma \in \textstyle\prod_{c\in\PHdirectpredec{\upsilon}}C_{a,A,B}^c \}
\\
C_{a,A,B} & \DEF \textstyle\bigcup_{b\in\PHpredec{a}} C^b_{a,A,B}
\label{eq:K-ctx}
\end{align}

The parameter $K_{a,A,B}$ specifies to which values $a$ eventually evolves as long as the context
$C_{a,A,B}$ holds, which is precisely the definition of the $\focals$ function
(\pref{def:focals} in \pref{ssec:focal}),
ggere the focals reachability property can by derived from \pref{pro:wf-ph-K} and
\pref{eq:param_context_coop}.
Hence $K_{a,A,B} = \focals(a,C^a_{a,A,B},C_{a,A,B})$ if this latter is a non-empty interval
(\pref{pps:param_K}).

\begin{proposition}[Parameter inference]
\label{pps:param_K}
Let $(\PHs, \PHl, \PHh)$ be a Process Hitting well-defined for IG inference, and $\IG = (\Gamma,
E_+, E_-)$ the inferred IG.
Let $A$ (resp. $B$) $\subseteq \Gamma$ be the set of regulators that activate (resp. inhibit) a sort
$a$.
%If $\focals(a,C_{a,A,B})$ is a non-empty interval, then $K_{a,A,B} = \focals(a, C_{a,A,B})$.
If $\focals(a,C^a_{a,A,B},C_{a,A,B})=[a_i;a_j]$ is a non-empty interval, 
	then $K_{a,A,B} = [i;j]$.
\end{proposition}

\begin{example*}
Applied to the PH in \pref{fig:runningPH-1}, we obtain, for instance, 
$K_{b,\{a\},\emptyset} = [0 ; 1]$,
$K_{a,\{b,c\},\emptyset} = [2 ; 2]$,
while $K_{a,\{b\},\{c\}}$ can not be inferred;
For the PH in \pref{fig:runningPH-2}, this latter is evaluated to $[1;1]$.
\end{example*}

Given the \pref{pps:param_K}, we see that in some cases, the inference of the targeted parameter is impossible.
This can be due to a lack of cooperation between regulators: when two regulators independently hit a component, their actions can have opposite effects, leading to either an indeterministic evolution or to oscillations.
Such an indeterminism is not possible in a GRN as in a given configuration of regulators, a component can only have an interval attractor, and eventually reaches a steady-state.
In order to avoid such inconclusive cases, one has to ensure that no such behavior is allowed by
either removing undesired actions or using cooperative sorts to avoid opposite influences between
regulators.

\subsection{Admissible parametrizations}\label{ssec:admissible-K}

When building a BRN, one has to find the parametrization that best describes the desired behavior of the studied system.
Complexity is inherent to this process as the number of possible parametrizations for a given IG is exponential w.r.t. the number of components.
However, the method of parameters inference presented in this section gives some information about necessary parameters given a certain dynamics described by a PH.
This information thus drops the number of possible parametrizations, allowing to find the desired behavior more easily.

We first delimit the validity of a parameter (\pref{pro:K-valid}) in order to ensure that any
transition in the resulting BRN is allowed by the studied PH.
This is verified by the existence of a hit making the concerned component bounce into the direction
of the value of the parameter in the matching context.
Thus, assuming \pref{pro:wf-ph-K} holds, any transition in the inferred BRN corresponds to at least
one transition in the PH, proving the correctness of our inference.
We remark that any parameter inferred by \pref{pps:param_K} satisfies this property.

\begin{property}[Parameter validity]\label{pro:K-valid}
A parameter $K_{a,A,B}$ is valid w.r.t. the PH iff the following equation is verified:
\begin{align*}
\forall a_i\in C^a_{a,A,B},
		a_i \notin K_{a,A,B} \Longrightarrow (
  \exists \PHfrappe{c_k}{a_i}{a_j}\in\PHa, & c_k \in C^c_{a,A,B} \\
 \wedge a_i < K_{a,A,B} \Rightarrow j > i 
 & \wedge  a_i > K_{a,A,B} \Rightarrow j <i )
\end{align*}
\end{property}
		

Then, we use some additional biological constraints on Thomas' parameters given in
\cite{BernotSemBRN}, that we sum up in the following three properties:

\begin{property}[Extreme values assumption]
Let $\IG = (\Gamma, E_+, E_-)$ be an IG. A parametrization $K$ on $\IG$ satisfies the \emph{extreme values assumption} iff:
\label{prop:param_enum_extreme}
\[
  \forall b \in \Gamma, \GRNreg{b} \neq \emptyset \Rightarrow 0 \in K_{b,\emptyset,\GRNreg{b}} \wedge l_b \in K_{b,\GRNreg{b},\emptyset}
\]
\end{property}

\begin{property}[Activity assumption]
\label{prop:param_enum_activity}
Let $\IG = (\Gamma, E_+, E_-)$ be an IG. A parametrization $K$ on $\IG$ satisfies the \emph{activity assumption} iff:
\begin{align*}
  \forall b \in \Gamma, \forall a \in \GRNreg{b}, \exists A,B \in \GRNallres{a}, K_{b,A,B} <_{[]} K_{b,A \cup \{b\},B \setminus \{b\}}
\\
  \forall b \in \Gamma, \forall a \in \GRNreg{b}, \exists A,B \in \GRNallres{a}, K_{b,A \setminus \{b\},B \cup \{b\}} <_{[]} K_{b,A,B}
\end{align*}
\end{property}

\begin{property}[Monotonicity assumption]
\label{prop:param_enum_monotonicity}
Let $\IG = (\Gamma, E_+, E_-)$ be an IG. A parametrization $K$ on $\IG$ satisfies the \emph{monotonicity assumption} iff:
\[
  \forall b \in \Gamma, \forall A,B \in \GRNallres{b}, \forall A',B' \in \GRNallres{b},
  A \subset A' \wedge B' \subset B \Rightarrow K_{b,A,B} \leq_{[]} K_{b,A',B'}
\]
\end{property}

\begin{comment}
\begin{definition}[Admissible parametrization \& Admissible parametrization with respect to inferred parameters]
\label{def:param_enum_inf}
Let $\PH = (\PHs, \PHl, \PHh)$ be a PH so that IG inference is possible, and $\IG = (\Gamma, E_+,
E_-)$ the inferred IG.
A parametrization $K$ on $\IG$ is said to be \emph{admissible} iff it respects
the extreme values assumption, the activity assumption and the monotonicity assumption.
A parametrization $K$ on $\IG$ is said to be \emph{admissible with respect to the
inferred parameters} iff it is admissible and that all parameters that can be inferred regarding
\pref{pps:param_K} are equal to their inferred value.
\end{definition}

\todo{utilité de “Admissible parametrization” seul ?}
\end{comment}


\subsection{Answer Set Programming implementation concepts}

\newcommand{\ti}[1]{\texttt{\textit{#1}}}
\newcommand{\aspil}[1]{\texttt{#1}}
\newcommand{\asp}[1]{\begin{itemize} \item[] \aspil{#1} \end{itemize}}

It is essential to get an efficient method to enumerate all the admissible parametrizations and
we focus on Answer Set Programming (ASP) \cite{Baral03} to address this issue. The motivations are following:
\begin{itemize}
  \item ASP efficiently tackles the inherent complexity of the models we use, thus allowing an efficient execution of the formal tools defined in this paper,
  \item it is convenient to enumerate a large set of possible answers,
  \item it allows us to easily constrain the answers according to some properties.
\end{itemize}
We now synthesize some key points to make the reader better understand our ASP implementation with the enumeration example.

All information describing the studied model (PH and inferred IG \& parameters) can be expressed in ASP using facts.
For functional purposes, we assign a unique label to each couple $A,B$ of activators and inhibitors of a given component, which allows to refer to the related parameter (in the following, we note $K^p_{a,A,B}$ the parameter of component $a$ whose regulators $A,B$ are assigned to the label $p$).
Consequently, to express that it exists a parameter of component \ti{a} with the label \ti{p}, we use an atom named \aspil{param\_label} in the following fact:
\asp{param\_label(\ti{a}, \ti{p}).}

Defining a set in ASP is equivalent to defining the rule for belonging to this set. For example, we can define an atom \aspil{param\_act} that describes the set of all active regulators for a parameter of gene \ti{a} and label \ti{p} (\ie the set $A$ of a parameter $K^\ti{p}_{\ti{a},A,B}$). For example, describing the activators of $K^\ti{p}_{\ti{a},\{\ti{b},\ti{c}\},\{\ti{d}\}}$ gives:
\asp{param\_act(\ti{a}, \ti{p}, \ti{b}).
\item[] param\_act(\ti{a}, \ti{p}, \ti{c}).}
The absence of such a fact involving \ti{d} with label \ti{p} indicates that \ti{d} is an inhibitor in the configuration of regulators related to this parameter.

Rules allow more detailed declarations than facts as they have a body (right-hand part below) containing constraints and allowing to use variables, while facts only have a head (left-hand part).
For instance, in order to define the set of expression levels of a component, we can declare:
\asp{component\_levels(X, 0..M) :- component(X, M).}
where the \aspil{component(X, M)} atom stands for the existence of a component \aspil{X} with a maximum level \aspil{M}.
Considering this declaration, any possible answer for the atom \aspil{component\_levels} will be found by binding all possible values of its terms with all existing \aspil{component} facts: an answer \aspil{component\_levels(\ti{a}, \ti{k})} will depend on the existence of a term \ti{a}, which is bound with \aspil{X}, and an integer~\ti{k}, constrained by: $0 \leq \ti{k} \leq \aspil{M}$.

Cardinalities are convenient to enumerate all possible parametrizations by creating multiple answer sets.
A cardinality (denoted hereafter with curly brackets) gives any number of possible answers for some atoms between a lower and upper bounds.
For example,
\asp{1 \{ param(X, P, I) : component\_levels(X, I) \} :-
\item[] ~~~~~~param\_label(X, P), not infered\_param(X, P).}
where \aspil{param(X, P, I)} stands for: $\aspil{I} \in K^\aspil{P}_{\aspil{X},A,B}$,
means that any parameter of component \aspil{X} and label \aspil{P} must contain at least one level value (\aspil{I}) in the possible expression levels of \aspil{X}.
Indeed, the lower bound is 1, forcing at least one element in the parameter, but no upper bound is specified, allowing up to any number of answers.
The body (right-hand side) of the rule also checks for the existence of a parameter of \aspil{X} with label \aspil{P}, and constrains that the parametrization inference was not conclusive for the considered parameter (\aspil{not} stands for negation by failure: \aspil{not L} becomes true if \aspil{L} is not true).
Such a constraint gives multiple results as any set of \aspil{param} atoms satisfying the cardinality will lead to a new global set of answers.
In this way, we can enumerate all possible parametrizations which respects the results of parameters
inference, but completely disregarding the notion of admissible parametrizations given in
\pref{ssec:admissible-K}.

We rely on integrity constraints to filter only admissible parametrizations.
An integrity constraint is a rule with no head, that makes an answer set unsatisfiable if its body turns out to be true.
Hence, supposing that:
\begin{itemize}
  \item the \aspil{less\_active(\ti{a}, \ti{p}, \ti{q})} atom means that $K^\ti{p}_{\ti{a},A,B}$ stands for a configuration with less activating regulators than $K^\ti{q}_{\ti{a},A',B'}$ (\ie $A \subset A'$),
  \item the \aspil{param\_inf(\ti{a}, \ti{p}, \ti{q})} atom means: $K^\ti{p}_{\ti{a},A,B} \leq_{[]} K^\ti{q}_{\ti{a},A',B'}$,
\end{itemize}
then the monotonicity assumption can be formulated as the following integrity constraint:
\asp{:- less\_active(X, P, Q), not param\_inf(X, P, Q).}
which removes all parametrization results where parameters $K^\aspil{P}_{\aspil{X},A,B}$ and $K^\aspil{Q}_{\aspil{X},A',B'}$ exist such that $A \subset A'$ and $K^\aspil{P}_{\aspil{X},A',B'} <_{[]} K^\aspil{Q}_{\aspil{X},A,B}$, thus violating the monotonicity assumption.
Of course, other assumptions can be formulated in the same way.

This subsection succinctly described how we write ASP programs to represent a model and solve all steps of Thomas' modeling inference.
It finds a particularly interesting application in the enumeration of parameters: all possible parametrizations are generated in separate answer sets, and integrity constraints are formulated to remove those that do not fit the assumptions of admissible parametrizations,
thus reducing the number of interesting parametrizations to consider in the end.

\section{Examples}\label{sec:examples}

The inference method described in this paper has been implemented as part of
\textsc{Pint}\footnote{Available at \url{http://process.hitting.free.fr}}, which gathers PH related
tools.
Our implementation mainly consists in ASP programs that are solved using Clingo\footnote{Available
at \url{http://potassco.sourceforge.net}}.
The IG and parameters inference can be performed using the command
\texttt{ph2thomas -i model.ph -{}-dot ig.dot}
where \texttt{model.ph} is the PH model in \textsc{Pint} format and \texttt{ig.dot} is an output of the inferred IG in DOT format.
The (possibly partial) inferred parametrization will be returned on the standard output.
The admissible parametrizations enumeration is performed when adding the \texttt{-{}-enumerate}
parameter to the command.

Applied to the example in \pref{fig:runningPH-2} where cooperations have been defined,
our method infers the IG and parametrization given in \pref{fig:runningBRN}.
Regarding the example in \pref{fig:runningPH-1}, the same IG is inferred, as well as for the
parametrization except for the parameters $K_{a,\{b\},\{c\}}$ and $K_{a,\{c\},\{b\}}$ which are
undefined (because of the lack of cooperativity between $b$ and $c$).
In such a case, this partial parametrization allows 36 admissible complete parametrizations, as two
parameters with 3 potential values could not be inferred.
If we constrain these latter parameters so that they contain exactly one element, we obtain only 9
admissible parametrizations.

The current implementation can successfully handle large PH models of BRNs found in the literature
such as an ERBB receptor-regulated G1/S transition model from \cite{Sahin09} which contains 20
components, and a T-cells receptor model from \cite{Klamt06} which contains 40
components\footnote{Both models are available as examples distributed with \textsc{Pint}.}.
For each model, IG and parameters inferences are performed together in less than a second
on a standard desktop computer.
%\footnote{Using a Dell Inspiron 1720 laptop, with an Intel Core 2 Duo CPU T5550 ($2 \times 1.83\text{GHz}$)
%and 3.9 Gib memory, on an Ubuntu 11.10 64-bits OS}
After removing the cooperations from these models (leaving only raw actions), the inferences allow to
determine 40 parameters out of 195 for the 20 components model, and 77 out of 143 for the 40 components model.
As we thus have an order of magnitude of respectively $10^{31}$ and $10^{73}$ admissible parametrizations,
this model would therefore be more efficiently studied as a PH than as a BRN.
We note that the complexity of the method is exponential in the number of regulators of one
component and linear in the number of components.
% We note that despite its smaller size in term of components, the ERBB transmission model takes more time to be computed because the biggest cooperative sorts contain more processes (up to 32 processes) than in the T-cells receptor model (up to 8 processes).

\section{Future work}
\resume{In order to gain some performance in the inference, we may consider several leads. Model reduction by cooperative sorts removal may allow to obtain good results without exhaustive search and with possibly lower complexity. Using the multiplexes semantics would allow to reduce the possible parametrizations and make the cooperations appear in the Interaction Graph. Finally, Other ways of coping with the incomplete cooperations could be found.}

\todo{Cope with cases where cooperations are incomplete}

\todo{Use the multiplexes semantics and infer them for complete cooperations to reduce the number of parameters}

\todo{Other approach: model reduction using projections}

\section{Conclusion}


\bibliographystyle{splncs}
\bibliography{biblio}

\end{document}
