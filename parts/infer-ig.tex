\section{Interaction Graph Inference}\label{sec:infer-IG}

In order to infer a complete BRN, one has to find the Interaction Graph (IG) first, as some
constraints on the Parametrization rely on it.
Inferring the IG is an abstraction step which consists in determining the global influence of
components on each of its successors.

This section introduces first the notion of focal processes within a PH
(\pref{ssec:focal}) which is used to characterize well-formed PH for IG inference
in \pref{ssec:wf}, and as well used by the parametrization inference presented in \pref{sec:infer-K}.
Finally, the rules for inferring the interactions between components from a PH are
described in \pref{ssec:infer-IG}.
We consider hereafter a global PH $(\PHs,\PHl,\PHa)$ on which the IG inference is to be
performed.

\subsection{Focal Processes}\label{ssec:focal}

Many of the inferences defined in the rest of this paper rely on the knowledge of \emph{focal
processes} w.r.t. a given context (a set of processes that are potentially present).
When such a context applies, we expect to (always) reach one focal process in a bounded number of
actions.

Let us define as $\PHa(S_a,\ctx)$ the set of actions on the sort $a$ having their hitter in
$\ctx$ and target in $S_a$ (\pref{eq:PHa-ctx});
and the the digraph $(V, E)$ where arcs are the bounces within the sort $a$ triggered by actions
in $\PHa(S_a,\ctx)$ (\pref{eq:bounce-graph}).
$\focals(a,S_a,\ctx)$ denotes the set of focal processes of sort $a$ in the scope of
$\PHa(S_a,\ctx)$ (\pref{def:focals}).
\begin{align}
\PHa(S_a,\ctx) & \DEF \{ \PHfrappe{b_i}{a_j}{a_k}\in\PHa \mid b_i\in\ctx \wedge a_j\in S_a \}
\label{eq:PHa-ctx}
\\
\begin{split}
E  & \DEF \{(a_j,a_k)\in (S_a \times \PHl_a) \mid 
			\exists\PHfrappe{b_i}{a_j}{a_k}\in \PHa(S_a,\ctx) \}
\\
V & \DEF S_a \cup \{ a_k\in L_a\mid \exists (a_j,a_k)\in E\}
\end{split}
\label{eq:bounce-graph}
\end{align}

\begin{definition}[$\focals(a,S_a,\ctx)$]\label{def:focals}
The set of processes that are focal for processes in $S_a$ in the scope of $\PHa(S_a,\ctx)$
are given by:
%$\focals(a,S_a,\ctx)$ is the set of focal processes of sort $a$ in the context $\ctx$:
\[
\focals(a,S_a,\ctx) \DEF
\begin{cases}
\{ a_i \in V \mid \nexists (a_i,a_j)\in E\} & \text{if the digraph $(V,E)$ is acyclic},\\
\emptyset & \text{otherwise.}\\
\end{cases}
\]
\end{definition}

We say that a state $s\in\PHl$ \emph{matches} a context $\ctx$ if and only if
$\forall a\in\PHsort(\ctx), \PHget{s}{a}\in\ctx$, where $\PHsort(\ctx)$ is the set of sorts with
processes in $\ctx$.
From \pref{def:focals}, it derives that:
\begin{enumerate}
\item if $\focals(a,S_a,\ctx)=\emptyset$, there exists a 
state $s\in \PHl$ matching $\ctx\cup S_a$ such that $\forall n\in\mathbb N$ there
exists a sequence of $n+1$ actions in $\PHa(S_a,\ctx)$ successively playable in $s$.
\item if $\focals(a,S_a,\ctx)\neq\emptyset$, for all
state $s\in \PHl$ matching $\ctx\cup S_a$,
for any sequence of actions $h^1,\dots,h^k$ in $\PHa(S_a,\ctx)$ successively playable in $s$,
either
\begin{itemize}
\item $(s\play h^1\cdots h^k)[a] \in \focals(a,S_a,\ctx)$;
\item or, $\exists h^{k+1}\in \PHa(a,\ctx)$ playable in $s\play h^1\cdots h^k$.
\end{itemize}
Moreover $k\leq|\PHa(S_a,\ctx)|$ (i.e. no cycle of actions possible).
\end{enumerate}

In other words, if $\focals(a,S_a,\ctx)$ is empty, there exists a set of actions in
$\PHa(S_a,\ctx)$ that may be played as many number of times as we want (cycle);
if it is non-empty, all possible succession of actions in $\PHa(S_a,\ctx)$ have a bounded length and
lead $a$ either to a process in $S_a$ that is not hit by processes in $\ctx$, or to a process in
$L_a\setminus S_a$.

\begin{example*}
In the Process Hitting of \pref{fig:runningPH-1}, we obtain:
\begin{align*}
\focals(a,L_a,\{b_0,c_0\}) &= \{ a_0 \}
&
\focals(a,L_a,\{b_1,c_1\}) &= \{ a_2 \}\\
\focals(a,L_a,\{b_1,c_0\}) &= \emptyset\enspace.
\end{align*}
\end{example*}

\subsection{Well-formed Process Hitting for Interaction Graph Inference}\label{ssec:wf}

The inference of an IG from a Process Hitting assumes that the Process Hitting defines two types of
sorts:
the sorts corresponding to BRN components, and the cooperative sorts.
This leads to the characterization of a \emph{well-formed} Process Hitting for IG inference.

The identification of sorts modelling components rely on the observation that their processes
represent (ordered) qualitative levels.
Hence an action on such a sort cannot make it bounce to process at a distance more than one.
The set of sorts satisfying such a condition is referred to as $\Gamma$
(\pref{eq:PH-components}).
Therefore, in the rest of this paper, $\Gamma$ denotes the set of components of the BRN to infer.

\begin{equation}
\Gamma \DEF \{a \in \PHs \mid \nexists \PHfrappe{b_i}{a_j}{a_k} \in \PHa, |j - k| > 1\} \\
\label{eq:PH-components}
\end{equation}

Any sort that does not act as a component should then be treated as a cooperative sort.
As explained in \pref{ssec:PH}, the role of a cooperative sort $\upsilon$ is to compute the current
state of set of cooperating processes.
Hence, for each sub-state $\sigma$ formed by the sorts hitting $\upsilon$, $\upsilon$ should
converge to a focal process.
This is expressed by \pref{pro:wf-cooperative-sort}, where
the set of sorts having an action on a given sort $a$ is given by 
$\PHdirectpredec{a}$ (\pref{eq:ph_direct_predec})
and $\PHproc(\sigma)$ is the set of processes that compose the sub-state $\sigma$.

\begin{equation}
\forall a \in \PHs, \PHdirectpredec{a} \DEF \{b \in \PHs \mid \exists \PHfrappe{b_i}{a_j}{a_k}\in\PHa \}
\label{eq:ph_direct_predec}
\end{equation}

\begin{property}[Well-formed cooperative sort]\label{pro:wf-cooperative-sort}
A sort $\upsilon\in\PHs$ is a well-formed cooperative sort if and only if
each configuration $\sigma$ of its predecessors leads $\upsilon$ to a unique focal process,
denoted by $\upsilon(\sigma)$:
\[
\forall \sigma \in {\textstyle\prod_{
a\in\PHdirectpredec{\upsilon} \wedge a\neq \upsilon}}
\PHl_{a},
\focals(\upsilon,\PHl_\upsilon,\PHproc(\sigma)\cup \PHl_\upsilon) = \{ \upsilon(\sigma) \}\]
\end{property}

Such a property allows a large variety of definition of a cooperative sort, but
for the sake of simplicity, does not allow the existence of multiple focal processes.
While this may be easily extended to (the condition becomes 
$\focals(\upsilon,\PHl_\upsilon, \PHproc(\sigma)\cup \PHl_\upsilon)\neq\emptyset$), it makes some
hereafter equations a bit more complex to read as they should handle set of focal processes instead
of a unique focal process.


Finally, \pref{pro:wf-ph} sums up the conditions for a Process Hitting to be suitable for IG
inference.
In addition of having either component sorts or well-formed cooperative sorts, we also require that
there is no cycle between cooperative sorts, and that
sorts being not hit (\ie{}, serving as an invariant environment) are components.

\begin{property}[Well-formed Process Hitting for IG inference]\label{pro:wf-ph}
A Process Hitting is well-formed for IG inference if and only if the following conditions are
verified:
\begin{itemize}
\item 
each sort $a\in\PHs$ either belongs to $\Gamma$, or is a well-formed cooperative sort;
\item 
there is no cycle between cooperative sorts
(the digraph $(\Sigma,\{(a,b)\in(\Sigma\times\Sigma)\mid \exists \PHfrappe{a_i}{b_j}{b_k}\in\PHa
\wedge a\neq b\wedge \{a,b\}\cap\Gamma=\emptyset \})$ is
acyclic);
\item 
sorts having no action hitting them belong to $\Gamma$
($\{ a \in \Sigma\mid \nexists \PHfrappe{b_i}{a_j}{a_k}\in\PHa\} \subset \Gamma$).
\end{itemize}
\end{property}

\begin{example*}
In the model of \pref{fig:runningPH-2}, $bc$ is a well-formed cooperative sort as defined in \pref{pro:wf-cooperative-sort}, because:
\begin{align*}
\focals(bc, \PHl_{bc}, \{b_0, c_0\} \cup \PHl_{bc}) = \{bc_{00}\} && \focals(bc, \PHl_{bc}, \{b_0, c_1\} \cup \PHl_{bc}) = \{bc_{01}\} \\
\focals(bc, \PHl_{bc}, \{b_1, c_0\} \cup \PHl_{bc}) = \{bc_{10}\} && \focals(bc, \PHl_{bc}, \{b_1, c_1\} \cup \PHl_{bc}) = \{bc_{11}\}
\end{align*}
Furthermore, $\PHs = \Gamma \cup \{bc\}$, with $\Gamma = \{a, b, c\}$ as defined in \pref{eq:PH-components}.
Therefore, as $bc$ is the only cooperative sort and does not hit itself,
\pref{pro:wf-ph} holds for the PH in \pref{fig:runningPH-2}, which is well-formed for IG inference.

The model in \pref{fig:runningPH-1} is also well-formed for IG inference as it contains no cooperative sort, and: $\PHs = \Gamma = \{a, b, c\}$.
\end{example*}


\begin{comment}
\begin{center}
\begin{tikzpicture}
% Sortes
\TSetSortLabel{a}{b}
\TSetSortLabel{b}{c}
\TSetSortLabel{z}{a}
\TSetSortLabel{ab}{bc}
\TSort{(0,3)}{a}{2}{l}
\TSort{(0,0)}{b}{2}{l}
\TSetTick{ab}{0}{00}
\TSetTick{ab}{1}{01}
\TSetTick{ab}{2}{10}
\TSetTick{ab}{3}{11}
\TSort{(3,0.5)}{ab}{4}{r}
\TSort{(6,1)}{z}{3}{r}

% Actions de màj de ab
\THit{a_1}{}{ab_0}{.west}{ab_2}
\THit{a_1}{}{ab_1}{.west}{ab_3}
\path[bounce,bend left] \TBounce{ab_0}{}{ab_2}{.south} \TBounce{ab_1}{}{ab_3}{.south};

\THit{a_0}{}{ab_2}{.west}{ab_0}
\THit{a_0}{}{ab_3}{.west}{ab_1}
\path[bounce,bend right] \TBounce{ab_2}{}{ab_0}{.north} \TBounce{ab_3}{}{ab_1}{.north};

\THit{b_0}{}{ab_3}{.west}{ab_2}
\THit{b_0}{}{ab_1}{.west}{ab_0}
\THit{b_1}{}{ab_0}{.west}{ab_1}
\THit{b_1}{}{ab_2}{.west}{ab_3}
\path[bounce,bend right] \TBounce{ab_1}{}{ab_0}{.north} \TBounce{ab_3}{}{ab_2}{.north};
\path[bounce,bend left] \TBounce{ab_0}{}{ab_1}{.south} \TBounce{ab_2}{}{ab_3}{.south};

% Arcs sortant de ab
\THit{ab_2}{}{z_1}{.north west}{z_2}
\THit{ab_2}{}{z_0}{.north west}{z_1}
\path[bounce,bend left] \TBounce{z_1}{}{z_2}{.south} \TBounce{z_0}{}{z_1}{.south};
\THit{ab_3}{}{z_2}{.south west}{z_1}
\THit{ab_3}{}{z_0}{.north west}{z_1}
\path[bounce,bend left] \TBounce{z_2}{bend right}{z_1}{.north} \TBounce{z_0}{}{z_1}{.south};
\THit{ab_1}{}{z_2}{.south west}{z_1}
\THit{ab_1}{}{z_1}{.south west}{z_0}
\path[bounce,bend right] \TBounce{z_2}{}{z_1}{.north} \TBounce{z_1}{}{z_0}{.north};
\THit{ab_0}{}{z_2}{.south west}{z_1}
\THit{ab_0}{}{z_1}{.south west}{z_0}
\path[bounce,bend right] \TBounce{z_2}{}{z_1}{.north} \TBounce{z_1}{}{z_0}{.north};
\end{tikzpicture}
\end{center}
\end{comment}

\subsection{Interaction Inference}\label{ssec:infer-IG}

At this point we can divide the set of sorts $\PHs$ into components ($\Gamma$) and cooperative sorts
($\PHs \setminus \Gamma$) that will not appear in the IG. 
We define in \pref{eq:ph_predec} the set of predecessors of a sort $a$, that is, the sorts influencing $a$
by considering direct actions and possible intermediate cooperative sorts.
The predecessors of $a$ that are components are the regulators of $a$, denoted $\PHpredecgene{a}$
(\pref{eq:regulators}).
\begin{align}
\begin{split}
\forall a \in \PHs, \PHpredec{a} &\DEF \{b \in \PHs \mid \exists n \in \mathbb{N}^*, \exists
(c^k)_{k \in [0;n]} \in \PHs^{n+1}, \\
                                   & \quad \quad c^0 = b \wedge c^n = a \\
                                   & \quad \quad \wedge \forall k \in \llbracket 0 ; n-1 \rrbracket,
								   c^k \in \PHdirectpredec{c^{k+1}} \cap (\PHs\setminus\Gamma)\}
\end{split}
\label{eq:ph_predec}
\\
\forall a\in \PHs, \PHpredecgene{a} & \DEF \PHpredec{a} \cap \Gamma
\label{eq:regulators}
\end{align}

Given a set $g$ of components and a configuration (\ie a sub-state) $\sigma$, $\ctx_g(\sigma)$
refers to the set of focal processes hitting $a$ regulated by any sort in $g$ (\pref{eq:ctx-sigma}).
If $g=\{b\}$, we simple note $\ctx_b(\sigma)$.
This set is composed of the active processes of sort $g$, and the focal process (assumed
unique) of the cooperative sorts $\upsilon$ hitting $a$ that have a predecessor in $g$.
The evaluation of the focal process of $\upsilon$ in context $\sigma$, denoted $\upsilon(\sigma)$,
relies on \pref{pro:wf-cooperative-sort}, which gives its value when all the direct predecessors of
$\upsilon$ are defined in $\sigma$.
When a predecessor $\upsilon'$ is not in $\sigma$, we extend the evaluation by recursively computing
the focal value of $\upsilon'$ is $\sigma$, as stated in \pref{eq:cooperative-eval}.
Because there is no cycle between cooperative sorts, this recursive evaluation of $\upsilon(\sigma)$
always terminates.
\begin{align}
\forall g\subset \Gamma,
	\ctx_g(\sigma) & \DEF \{ \sigma[b] \mid b\in g \} \cup \{ \upsilon(\sigma) \mid
\upsilon\in\PHdirectpredec{a} \setminus \Gamma \wedge g\cap \PHpredecgene{\upsilon} \neq \emptyset \}
\label{eq:ctx-sigma}
\\
\upsilon(\sigma) & \DEF
\upsilon(\sigma \uplus \state{\upsilon'(\sigma) \mid 
	\upsilon'\in\PHdirectpredec{\upsilon} \wedge
	\upsilon'\in\PHs\setminus\Gamma })
\label{eq:cooperative-eval}
\end{align}

We aim at inferring that $b$ activates (inhibits) $a$ if there exists a configuration where increasing
the level of $b$ makes possible the increase (decrease) of the level of $a$.
This is analogous to standard IG inferences from discrete maps \cite{RiCo07}.

This reasoning can be straightforwardly applied to PH when inferring the influence of $b$ for $a$
when $b\neq a$ (\pref{eq:edges-inference-b}).
Let us define $\gamma(b\rightarrow a)$ as the set of components cooperating with $b$ to hit $a$,
including $b$ and $a$ (\pref{eq:cooperating-with-b}).
Given a configuration $\sigma\in\prod_{c\in\gamma(b\rightarrow a)} L_c$, 
$\focals(a,\{a_i\},\ctx_b(\sigma))$ gives the bounces that a given process $a_i$ can make in the
context $\ctx_b(\sigma)$.
We note $\sigma\{b_i\}$ the configuration $\sigma$ where the process of sort $b$ has been replaced
by $b_i$.
If there exists $b_i,b_{i+1}\in L_b$ such that one bounce in 
$\focals(a, \{\sigma[a]\}, \ctx_b(\sigma\{b_i\}))$
has a lower (resp. higher) level that one bounce in
$\focals(a, \{\sigma[a]\}, \ctx_b(\sigma\{b_{i+1}\}))$, then
$b$ as positive (resp. negative) influence on $a$ with a maximum threshold $l=j$.
\begin{equation}
\gamma(b\rightarrow a)  \DEF \{ a, b \} \cup \{ c \in \Gamma \mid 
			\exists \upsilon\in\PHs\setminus\Gamma,
				\upsilon\in\PHpredec{a} \wedge \{b,c\}\subset\PHpredec{\upsilon} \}
\label{eq:cooperating-with-b}
\end{equation}


Then, we infer that $a$ has a self-influence if its current level can have an impact on its own
evolution at given configuration $\sigma$.
We consider here a configuration $\sigma$ of a group $g$ of sorts having a cooperation on $a$.
This set of sorts is given by $X(a)$ (\pref{eq:influence-groups}) which returns the set of
connected components (noted $\mathcal C$) of the graph linking two regulators $b,c$ of $a$ if there
is a cooperative sort hitting $a$ regulated by them.
Given $a_i,a_{i+1}\in L_a$, we pick $a_j\in\focals(a,\{a_i\},\ctx_g(\sigma\{a_i\}))$ and
$a_k\in\focals(a,\{a_{i+1}\},\ctx_g(\sigma\{a_{i+1}\}))$.
If $k=j+1$, we can not conclude as there is no difference in the evolution of both levels.
If $k\neq j+1$ and $k-j\neq 0$, then $a_i$ and $a_{i+1}$ have divergent evolutions: we infer an
influence of sign of $k-j$ at threshold $i+1$.
We note that some aspects of this inference are arbitrary and may impact the number of parameters to
infer in the next section.
In particular, in some cases, the use of intervals for Thomas' parameters drops the requirement of
inferring a self-activation.
%Future work may propose alternative definitions of self-influences inference in order to range over
%different parametrization configurations.

\begin{equation}
X(a) = \mathcal C\left( (\PHpredecgene{a}, \{ \{b,c\} \mid
				\exists \upsilon\in \PHdirectpredec{a} \setminus \Gamma,
					\{b,c\} \subset \PHpredecgene{\upsilon} \}) \right)
\label{eq:influence-groups}
\end{equation}

\begin{comment}
a global overview of the evolution of each level
of $a$ w.r.t. a configuration $\sigma\in\configs a$.
The evolution of $a_i$ in the context $\ctx(\sigma\{a_i\})$ is given by the function
$\epsilon(a_i,\sigma)$ (\pref{eq:epsilon}) which returns $\varnothing$ if there is no action
possible, $+$ (resp. $-$) if all actions makes $a_i$ bounce to a higher (resp. lower) level,
and $\pm$ if both evolutions are possible.
\begin{equation}
\epsilon(a_i, \sigma) \DEF
\begin{cases}
\varnothing & \text{if }\focals(a,\{a_i\},\ctx(\sigma\{a_i\}))=\{ a_i \} \\
+ &  \text{if }\focals(a, \{a_i\},\ctx(\sigma\{a_i\})) = \{ a_{i+1} \}\\
- &  \text{if }\focals(a, \{a_i\},\ctx(\sigma\{a_i\})) = \{ a_{i-1} \}\\
\pm & \text{otherwise.}
\end{cases}
\label{eq:epsilon}
\end{equation}
We infer the self-influence of $a$ by checking one of the following three cases.
First, if there exists $a_i,a_j\in L_a, i <j$ such that $\epsilon(a_i,\sigma)$ and
$\epsilon(a_j,\sigma)$ are of opposite sign, then $a$ has a self-influence of the sign of the latter
$\epsilon$ (with a maximum threshold $k=j$).
Second, we look at the evolution at the limit levels of $a$:
if $\epsilon(a_0,\sigma)=\bar s$ or $\epsilon(a_{l_a},\sigma)=s$, we infer an influence of sign $s$
with a threshold $k=l_a$.
We note that this case can only apply for negative interactions (as $\epsilon(a_0,\sigma)$ (resp.
$\epsilon(a_{l_a},\sigma)$ can never be negative (resp. positive)).
Third, if $\forall a_i\in L_a$, $\epsilon(a_i,\sigma)$ is either $\varnothing$ or of sign $s$, we
ensure there exists $a_j\in L_a$ such that $\epsilon(a_j,\sigma)=s$ and we infer a self-influence of
sign $s$ and threshold $k=j$.
\end{comment}


\pref{pps:inference-edges} details the inference of all existing influences between genes occurring
with a threshold $l$.
These influences are split into positive and negative ones, and represent possible edges in the final IG.
\begin{proposition}[Edges inference]\label{pps:inference-edges}
We define the set of positive (resp. negative) influences $\hat{E}_+$ (resp. $\hat{E}_-$) for any
$a\in\Gamma$ by:
\begin{align}
\begin{split}
\forall b\in\PHpredecgene{a}, b\neq a, \\
b\xrightarrow l a \in \hat{E}_s & \Longleftrightarrow
 \exists \sigma\in\textstyle\prod_{c\in\gamma(b\rightarrow a)} L_c, \exists b_i,b_{i+1}\in \PHl_b,\\
& \qquad\qquad
        \exists a_{j}\in\focals(a, \{\sigma[a]\}, \ctx_b(\sigma\{b_i\})), \\
& \qquad\qquad
        \exists a_{k}\in\focals(a, \{\sigma[a]\}, \ctx_b(\sigma\{b_{i+1}\})), \\
& \qquad\qquad\qquad
                        s = \f{sign}(k - j) \wedge l = i+1
\end{split}
\label{eq:edges-inference-b}
\\
\begin{split}
a \xrightarrow l a \in \hat{E}_s & \Longleftrightarrow
\exists g \in X(a), \sigma \in L_a\times \textstyle\prod_{b\in g} L_b,
			\exists a_i,a_{i+1}\in \PHl_a, \\
& \qquad\qquad
        \exists a_{j}\in\focals(a, \{a_i\}, \ctx_g(\sigma\{a_i\})), \\
& \qquad\qquad
        \exists a_{k}\in\focals(a, \{a_{i+1}\},  \ctx_g(\sigma\{a_{i+1}\})), \\
& \qquad\qquad\quad
			k \neq j+1
				\wedge s = \f{sign}(k - j) \wedge l = i+1
\end{split}
\label{eq:edges-inference-a}
\end{align}
where $s \in \{ +, - \}$, $\bar s = + \EQDEF s = -$, $\bar s = - \EQDEF s = +$,
$\f{sign}(n) = + \EQDEF n > 0$,
$\f{sign}(n) = - \EQDEF n < 0$,
and $\f{sign}(0) \EQDEF 0$.
\end{proposition}

\begin{comment}
\begin{align*}
\begin{split}
a \xrightarrow l a \in \hat{E}_s & \Longleftrightarrow
\neg(\PHpredecgene{a} = \{a\} \wedge \focals(a, \PHl_a, \PHl_a) = [a_i;a_j], i\leq j) \\
& \wedge \exists\sigma\in L_a\times\textstyle\prod_{c\in \PHpredecgene{a}} L_c, ( 
       (\exists a_i,a_{i+1}\in \PHl_b, \\
& \qquad\qquad
        \exists a_{j}\in\focals(a, \{a_i\}, \ctx'(\sigma\{a_i\})), \\
& \qquad\qquad
        \exists a_{k}\in\focals(a, \{a_{i+1}\}, \ctx'(\sigma\{a_{i+1}\})), \\
& \qquad\qquad\quad
			k \neq j+1
				\wedge s = \f{sign}(k - j) \wedge l = i+1\\
& \qquad\vee
	(\forall a_i\in L_a, \focals(a,\{a_i\},\ctx'(\sigma\{a_i\})) = \{a_i\} \\
& \qquad\qquad 
		\wedge s=+ \wedge l=1)
)
\end{split}
\end{align*}
\end{comment}

\begin{comment}
\begin{align*}
\begin{split}
a \xrightarrow k a \in \hat{E}_s & \Longleftrightarrow
\neg(\PHpredecgene{a} = \{a\} \wedge \focals(a, \PHl_a, \PHl_a) = [a_i;a_j], i\leq j) \\
& \wedge \exists\sigma\in\configs a, (
        (\exists a_i,a_j\in L_a, i < j, \\
& \qquad\qquad
                \epsilon(a_i,\sigma) \in \{\bar s,\pm\}
					\wedge \epsilon(a_j,\sigma) \in \{ s, \pm \}  \wedge k = j) \\
& \qquad \vee
        ((\epsilon(a_0, \sigma) = \bar s \vee \epsilon(a_{l_a}, \sigma) = s) \wedge k = l_a) \\
& \qquad \vee
        (\epsilon(a_0, \sigma) = \epsilon(a_{l_a}, \sigma) = \varnothing \\
& \qquad\qquad       \wedge \forall a_i\in L_a, \epsilon(a_i, \sigma)\in \{\varnothing, s\} \\
& \qquad\qquad       \wedge \exists a_j\in L_a, \epsilon(a_j, \sigma) = s \wedge k = j)
                )
\end{split}
\end{align*}
\end{comment}

We are now able to infer the edges of the final IG by considering positive and negative influences
(\pref{pps:inference-IG}).
We infer a positive (resp. negative) edge if there exists a corresponding influence with the same
sign. If an influence is both positive and negative, we infer an unsigned edge. In the end, the
threshold of each edge is the minimum threshold for which the influence has been found. As unsigned
edges represent ambiguous interactions, no threshold is inferred.
\begin{proposition}[Interaction Graph inference]\label{pps:inference-IG}
We infer $\IG = (\Gamma,E_+,E_-,E_\pm)$ from \pref{pps:inference-edges} as follows:
\begin{align*}
E_+ &= \{a \xrightarrow{t} b \mid \nexists a \xrightarrow{t'} b \in \hat{E}_-
  \wedge t = \min \{ l \mid a \xrightarrow{l} b \in \hat{E}_+\}\} \\
E_- &= \{a \xrightarrow{t} b \mid \nexists a \xrightarrow{t'} b \in \hat{E}_+
  \wedge t = \min \{l \mid a \xrightarrow{l} b \in \hat{E}_-\}\} \\
E_\pm &= \{a \rightarrow b \mid \exists a \xrightarrow{t} b \in \hat{E}_+ \wedge \exists a \xrightarrow{t'} b \in \hat{E}_-\} \\
\end{align*}
\end{proposition}


\begin{example*}
The IG inference from the PH of \pref{fig:runningPH-2} gives
$\hat{E}_+ = \{b \xrightarrow{1} a, c \xrightarrow{1} a\}$ and 
$\hat{E}_- = \{a \xrightarrow{2} b\}$, corresponding the IG of \pref{fig:runningBRN}.
No self-influence are inferred ($X(a) = \{ \{b,c\} \}$ and $X(b)=X(c)=\emptyset$).
\end{example*}
