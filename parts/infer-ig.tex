\section{Interaction Graph inference}
\resume{The Process Hitting can be used to finely model a system by defining cooperations or studying its behavior using the available tools. One can then decide to translate this model back to Thomas' modeling, and the Interaction Graph inference is the first step to perform this. It relies on an exhaustive approach that looks for possible influences of a gene on the others given all possible configurations. This part gives good results.}

In order to infer a complete BRN, one has to find the Interaction Graph first, as some constraints on the Parametrization rely on it. Inferring the Interaction Graph consists in determining the influence of all components on each of its successors.

\todo{Distinguish components and cooperative sorts}

The nodes of the Interaction Graph can be found by analyzing the sorts of the Process Hitting and distinguishing them from cooperative sorts.

\todo{Start a running example here --- maybe re-use the example of previous section}

\todo{Algorithm to infer IG}

\todo{Illustrate with the example}

\todo{Discuss of errors/inconclusive cases}
