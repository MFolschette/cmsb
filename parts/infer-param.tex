\section{Parametrization inference}\label{sec:infer-K}

Given the IG inference results from a PH, as presented in the previous section, one can find the discrete parameters that model the behavior of the studied PH using the method presented in the following.
It relies on an exhaustive enumeration of all predecessors of each component in order to find attractor processes and returns a possibly incomplete Parametrization, given the exhaustiveness of the cooperations.
The last step consists of the enumeration of all possible Parametrizations, given this set of
inferred parameters, \todo{the PH structure}, and some biological constraints on parameters.

\subsection{Parameters inference}

This subsection presents some results related to the inference of independent discrete parameters from a given PH. These results are equivalent to those presented in \cite{PMR10-TCSB}, with notation adapted to be shared with the previous section.
We consider here a global PH $(\PHs,\PHl,\PHa)$ well-formed for IG inference, on which the parametrization inference has now to be performed. We suppose that the extended IG $(\Gamma, E_+, E_-, E_\pm)$ inferred from this PH contains no unsigned edge, that is: $E_\pm = \emptyset$, and thus can be seen as the regular IG $(\Gamma, E_+, E_-)$.
Let $K_{a,A,B}$ be the parameter we want to infer, for a given component $a \in \Gamma$,
%and $A,B \in \GRNallres{a}$ a configuration of resources of $a$ (activators and inhibitors).
and $A \subset \GRNreg{a}$ (resp. $B \subset \GRNreg{a}$) a set of activators (resp. inhibitors) in the regulators of $a$.
This inference, as for the Interaction Graph inference, relies on the search of focal processes of the component for the given configuration of its regulators.

For each sort $b \in \GRNreg{a}$, we define a context $C^b_{a,A,B}$ in \pref{eq:param_context} that contains all processes representing the influence of the regulators in the configuration $A,B$.
The context of a cooperative sort $\upsilon$ that regulates $a$ is given in \pref{eq:param_context_coop} as the set of processes that represent the given configuration.
$C_{a,A,B}$ refers to the union of all these contexts (\pref{eq:K-ctx}).
\begin{align}
\label{eq:param_context}
\forall b\in\Gamma,~
C_{a,A,B}^b & \DEF \begin{cases}
\levelsA{b}{a} & \text{if $b \in A$,}\\
\levelsI{b}{a} & \text{if $b \in B$,}\\
L_a		& \text{otherwise;}\\
\end{cases}
\\
\label{eq:param_context_coop}
\forall \upsilon \in \PHpredec{a}\setminus\Gamma,~
C_{a,A,B}^\upsilon & \DEF \{
\upsilon(\sigma) \mid \sigma \in \textstyle\prod_{c\in\PHdirectpredec{\upsilon}}C_{a,A,B}^c \}
\\
C_{a,A,B} & \DEF \textstyle\bigcup_{b\in\PHpredec{a}} C^b_{a,A,B}
\label{eq:K-ctx}
\end{align}

The parameter $K_{a,A,B}$ specifies to which values $a$ eventually evolves as long as the context
$C_{a,A,B}$ holds, which is precisely the definition of the $\focals$ function
(\pref{def:focals} in \pref{ssec:focal}).
Hence $K_{a,A,B} = \focals(a,C^a_{a,A,B},C_{a,A,B})$ if this latter is a non-empty interval
(\pref{pps:param_K}).

\begin{proposition}[Parameter inference]
\label{pps:param_K}
Let $(\PHs, \PHl, \PHh)$ be a Process Hitting well-defined for IG inference, and $\IG = (\Gamma,
E_+, E_-)$ the inferred IG.
Let $A$ (resp. $B$) $\subseteq \Gamma$ be the set of regulators that activate (resp. inhibit) a sort
$a$.
%If $\focals(a,C_{a,A,B})$ is a non-empty interval, then $K_{a,A,B} = \focals(a, C_{a,A,B})$.
If $\focals(a,C^a_{a,A,B},C_{a,A,B})=[a_i;a_j]$ is a non-empty interval, 
	then $K_{a,A,B} = [i;j]$.
\end{proposition}

\begin{example*}
\todo{parler uniquement de $K_b,\{a\},\emptyset$, $K_a,\{b,c\},\emptyset$ et $k_a,\{b\},\{c\}$ dans
le cas de \pref{fig:runningPH-1}.}
Parameters inference performed on the PH in \pref{fig:runningPH-1} gives the following partial Parametrization:
\begin{align*}
K_{b,\{a\},\emptyset} &= [0 ; 1] & K_{a,\{b,c\},\emptyset} &= [2 ; 2] \\
K_{b,\emptyset,\{a\}} &= [0 ; 0] & K_{a,\emptyset,\{b,c\}} &= [0 ; 0] \\
K_{c,\emptyset,\emptyset} &= [0 ; 1]
\end{align*}
No result is found for parameters $K_{a,\{b\},\{c\}}$ and $K_{a,\{c\},\{b\}}$ as the opposite influences of $b$ and $c$ on $a$ prevent any inference.

When performed on the PH in \pref{fig:runningPH-2}, parameters inference returns the complete Parametrization given in \pref{fig:runningBRN}.
\end{example*}

Given the \pref{pps:param_K}, we see that in some cases, the inference of the targeted parameter is impossible.
This can be due to a lack of cooperation between regulators: when two regulators independently hit a component, their actions can have opposite effects, leading to either an indeterministic evolution or to oscillations.
Such an indeterminism is not possible in a GRN as in a given configuration of regulators, a component can only have an interval attractor, and eventually reaches a steady-state.
In order to avoid such inconclusive cases, one has to ensure that no such behavior is allowed by either removing undesired actions or using cooperative sorts to avoid opposite influences between regulators of a component.

\todo{conclure sur la correction de l'inférence d'un paramètre}

\subsection{Admissible parametrizations enumeration}\label{ssec:admissible-K}

When building a BRN, one has to find the parametrization that best describes the desired behavior of the studied system.
Complexity is inherent to this process as the number of possible parametrizations for a given IG is exponential w.r.t. the number of components.
However, the method of parameters inference presented in this section gives some information about necessary parameters given a certain dynamics described by a PH.
This information thus drops the number of possible parametrizations, allowing to find the desired behavior more easily.

\todo{s'assurer qu'un paramètre ne dépasse pas la dynamique imposée par le PH \pref{pro:K-valid};
donner un argument pour la correction}

\begin{property}[Parameter validity]\label{pro:K-valid}
A parameter $K_{a,A,B}$ is valid w.r.t. the PH iff the following equation is verified:
\begin{align*}
\forall \sigma\in \textstyle\prod_{c\in\PHdirectpredec{a} \cup \{a\}} C^c_{a,A,B},
	\sigma[a] = a_i,  & \\
		a_i \notin K_{a,A,B} \Longrightarrow (
  \exists \PHfrappe{c_k}{a_i}{a_j}\in\PHa, \sigma[c] = c_k, & \\
 \qquad a_i < K_{a,A,B} & \Rightarrow j > i  \\
 \qquad \wedge a_i > K_{a,A,B} & \Rightarrow j <i )
\end{align*}
\end{property}
		

%The last step of our method is to enumerate all possible parametrizations regarding the results of
%the parameters inference and
\todo{+ 
some biological constraints given in \cite{BernotSemBRN}, that we sum
up in the following three properties:}

\begin{property}[Extreme values assumption]
Let $\IG = (\Gamma, E_+, E_-)$ be an IG. A parametrization $K$ on $\IG$ satisfies the \emph{extreme values assumption} iff:
\label{prop:param_enum_extreme}
\[
  \forall b \in \Gamma, \GRNreg{b} \neq \emptyset \Rightarrow 0 \in K_{b,\emptyset,\GRNreg{b}} \wedge l_b \in K_{b,\GRNreg{b},\emptyset}
\]
\end{property}

\begin{property}[Activity assumption]
\label{prop:param_enum_activity}
Let $\IG = (\Gamma, E_+, E_-)$ be an IG. A parametrization $K$ on $\IG$ satisfies the \emph{activity assumption} iff:
\begin{align*}
  \forall b \in \Gamma, \forall a \in \GRNreg{b}, \exists A,B \in \GRNallres{a}, K_{b,A,B} <_{[]} K_{b,A \cup \{b\},B \setminus \{b\}}
\\
  \forall b \in \Gamma, \forall a \in \GRNreg{b}, \exists A,B \in \GRNallres{a}, K_{b,A \setminus \{b\},B \cup \{b\}} <_{[]} K_{b,A,B}
\end{align*}
\end{property}

\begin{property}[Monotonicity assumption]
\label{prop:param_enum_monotonicity}
Let $\IG = (\Gamma, E_+, E_-)$ be an IG. A parametrization $K$ on $\IG$ satisfies the \emph{monotonicity assumption} iff:
\[
  \forall b \in \Gamma, \forall A,B \in \GRNallres{b}, \forall A',B' \in \GRNallres{b},
  A \subset A' \wedge B' \subset B \Rightarrow K_{b,A,B} \leq_{[]} K_{b,A',B'}
\]
\end{property}

\begin{comment}
\begin{definition}[Admissible parametrization \& Admissible parametrization with respect to inferred parameters]
\label{def:param_enum_inf}
Let $\PH = (\PHs, \PHl, \PHh)$ be a PH so that IG inference is possible, and $\IG = (\Gamma, E_+,
E_-)$ the inferred IG.
A parametrization $K$ on $\IG$ is said to be \emph{admissible} iff it respects
the extreme values assumption, the activity assumption and the monotonicity assumption.
A parametrization $K$ on $\IG$ is said to be \emph{admissible with respect to the
inferred parameters} iff it is admissible and that all parameters that can be inferred regarding
\pref{pps:param_K} are equal to their inferred value.
\end{definition}

\todo{utilité de “Admissible parametrization” seul ?}
\end{comment}


\subsection{Answer Set Programming implementation concepts}

\newcommand{\ti}[1]{\texttt{\textit{#1}}}
\newcommand{\aspil}[1]{\texttt{#1}}
\newcommand{\asp}[1]{\begin{itemize} \item[] \aspil{#1} \end{itemize}}

It is then essential to get an efficient method to enumerate all the admissible parametrizations. We choose to focus on Answer Set Programming (ASP) \cite{Baral03} to address this issue. The motivations are following: 
\begin{itemize}
\item ASP efficiently tackles the inherent complexity of the models we use, thus allows an efficient execution of the formal tools defined in this paper.
\item It allows to easily constrain the answers according to properties or cardinalities.
\end{itemize}
We now synthesize some key points to make the reader better understand our ASP implementation with the enumeration example.

When the step of admissible Parametrizations enumeration is reached, a lot of information has been gathered about the studied system: in addition to the starting PH model, corresponding data have been inferred about a complete IG and a possibly partial parametrization.
All this information describing the model can be expressed in ASP using facts.
For functional purposes, we assign a unique label to all discrete parameters of a given component, which allows to refer to a parameter using two variables (in other terms, we define a unique label for each couple $A,B$ of activators and inhibitors).
For example, if we want to express that a parameter of component \ti{a} has the label \ti{i}, we can use an atom named \aspil{param\_label} in the following fact:
\asp{param\_label(\ti{a}, \ti{i}).}

Defining a set in ASP is equivalent to defining the rule for belonging to this set. For example, we can define an atom \aspil{param\_act} that describes the set of all active regulators for a parameter of gene \ti{a} and label \ti{i} (\ie the set $A$ of a parameter $K_{\ti{a},A,B}$). For example, the assignation of label \ti{i} to $K_{\ti{a},\{\ti{b},\ti{c}\},\{\ti{d}\}}$, gives:
\asp{param\_act(\ti{a}, \ti{i}, \ti{b}). \item[] param\_act(\ti{a}, \ti{i}, \ti{c}).}
The absence of such a rule involving \ti{d} with label \ti{i} indicates that \ti{d} is not an activator in the configuration of regulators related to this parameter.

Rules allow more detailed declarations than facts as they have a body containing constraints. Variables can be used to produce interesting rules.
For instance, in order to define the set of expression levels of a component, we can declare:
\asp{component\_levels(A, 0..L) :- component(A, L).}
where the \aspil{component(A, L)} atom stands for the existence of a component \aspil{A} with a maximum level \aspil{L}.
Considering this declaration, Clingo will ground any possible answer for the atom \aspil{component\_levels} by binding all possible values of its terms regarding all existing facts: an answer \aspil{component\_levels(\ti{a}, \ti{k})} will depend on the existence of a fact like given in the body, and the second term will also depend on the constraint $0 \leq \ti{k} \leq \aspil{L}$.

The enumeration of all possible parametrizations can be performed using a cardinality, which constrains the number of answers for some variables.
Such a cardinality gives the set of atoms to enumerate in curly brackets, and a lower and upper bounds constraining the number of allowed answers.
For example,
\asp{1 \{enum\_param(A,P,V) : component\_levels(A,V)\} :- \item[] ~~~~~is\_component(A), param\_label(A,P), not infered\_param(A,P).}
means that any parameter of label \aspil{P} and gene \aspil{A} must contain at least one level value (\aspil{V}) in the possible expression levels of \aspil{A}.
Indeed, the lower bound is 1, forcing at least one element in the parameter, but no upper bound is specified, allowing up to any number of answers.
The body (right-hand side) of the rule also checks for the existence of component \aspil{A} and parameter label \aspil{P}, and constrains that the parametrization inference was not conclusive for the considered parameter (otherwise we would consider only the value of the inferred parameter).
Such a constraint gives multiple results as any set of atoms satisfying the cardinality will lead to a new global set of answers.
In this way, we can enumerate all possible parametrization which respects the results of parameters
inference, but completely disregarding the notion of admissible parametrizations given in
\pref{ssec:admissible-K}.

In order to filter only admissible parametrizations regarding the previous properties from all obtained answers, we rely on integrity constraints.
Such constraint is a rule with no head, that makes an answer set unsatisfiable if its body turns out to be true.
For the sake of clarity, we note $K^i_{a,A,B}$ the parameter of component $a$ whose regulators $A,B$ are assigned to the label $i$. Hence, supposing that:
\begin{itemize}
  \item the \aspil{less\_active(\ti{a}, \ti{i}, \ti{j})} atom means that $K^\ti{i}_{\ti{a},A,B}$ stands for a configuration with less activating regulators than $K^\ti{j}_{\ti{a},A',B'}$ (\ie $A \subset A'$),
  \item the \aspil{param\_inf(\ti{a}, \ti{i}, \ti{j})} atom means that $K^\ti{i}_{\ti{a},A,B} \leq K^\ti{j}_{\ti{a},A',B'}$,
\end{itemize}
then the monotonicity assumption can be formulated as the following integrity constraint:
\asp{:- less\_active(A,P1,P2), not param\_inf(A,P1,P2).}
where the \aspil{not} keyword stands for a negation: the whole body becomes false if the atom in the negation is true. This integrity constraint indeed removes all parametrization results where a couple of parameters $K_{a,A,B}$ and $K_{a,A',B'}$ exists such that $A \subset A'$ and $K_{a,A,B} > K_{a,A',B'}$, which would violate the monotonicity assumption. Of course, other assumptions can be formulated in the same way.

In conclusion, the approach succinctly described above allows to describe all given information about a BRN (PH model and informations about IG and Thomas' parameters) in order to allow ASP programs execution.
Such logic programming have been used to solve all steps of Thomas' modeling inference, but it finds a particularly interesting application in the enumeration of parameters: all possible parametrizations are generated in separate answer sets, and integrity constraints are formulated to remove answer sets that do not fit the assumptions of admissible parametrizations.
This method allows to efficiently use the parameter inference results to cut the number of possible parametrization, and reduce the number of interesting parametrizations to consider in the end.
