\subsection{Thomas' modeling}
\resume{The Thomas' modeling is the historical and widely-used model for the study of dynamical gene systems, and takes the form of a Biological Regulatory Network. In this subsection, we present this tool by defining the Interaction Graph and the Parametrization. We also justify our extension of the latter to use interval parameters instead of integers.}

\todo{dire que l'on s'inspire de \cite{Richard06,BernotSemBRN}}
The modeling of a BRN using René Thomas' formalism lies on two complementary sets of information about the system. First, the \emph{Interaction Graph} (IG) models the structure of the system by defining the components' properties and their mutual influences. The \emph{Parametrization} then allows to restrict the dynamic behavior of the system by allowing specific strengths to the influences.

As for PH, IG represents information about a finite number of \emph{components} allowed to take a value amongst a finite number of possible \emph{expression levels}.
For the sake of simplicity and to establish a parallel with PH, if $a$ represents a component, we call $a_i$ its $i^\text{th}$ expression level.
The IG is therefore composed of nodes that represent components, and edges labeled with a threshold that stand for interactions and can be either positive or negative.
For such an interaction to take place, the expression level of its head component has to be higher than its threshold; otherwise, the opposite influence is expressed.
Therefore, for any component $b$, a predecessor $a$ of $b$ such that we have $a \xrightarrow{t} b$ can be either an activator or an inhibitor of $a$, according to the type of interaction involved and if the expression level of $b$ if above of below the related threshold $t$.
We call $\levelsA{a}{b}$ (resp. $\levelsI{a}{b}$) the expression levels of $a$ where it is an activator (resp. inhibitor) of $b$.

\begin{definition}[Interaction Graph]
\label{def:ig}
An \emph{Interaction Graph} (IG) is a triple $(\Gamma, E_+, E_-)$ where $\Gamma$ is a finite number of \emph{components},
and $E_+$ (resp. $E_-$) $\subset \{a \xrightarrow{t} b \mid a, b \in \Gamma \wedge t \in \mathbb{N}\}$
is the set of positive (resp. negative) \emph{regulations} between two nodes, labeled with a \emph{threshold}.

A regulation from $a$ to $b$ is uniquely referenced:
if $a \xrightarrow{t} b \in E_+$ (resp. $E_-$),
$\nexists a \xrightarrow{t'} b \in E_+ \text{ (resp. $E_-$)}, t \neq t'$
and $\nexists t', a \xrightarrow{t'} b \in E_-$ (resp. $E_+$).
\end{definition}

\begin{definition}[Effective levels ($\levels$)]\label{def:levels}
Let $(\Gamma,E_+,E_-)$ be an IG and $a, b \in \Gamma$ two of its components:
\begin{itemize}
  \item if $a \xrightarrow{t} b \in E_+$, $\levelsA{a}{b} = [t; l_a]$ and
    $\levelsI{a}{b} = [0; t-1]$;
  \item if $a \xrightarrow{t} b \in E_-$, $\levelsA{a}{b} = [0; t-1]$ and
    $\levelsI{a}{b} = [t; l_a]$;
  \item otherwise, $\levelsA{a}{b} = \levelsI{a}{b} = \emptyset$.
\end{itemize}
\end{definition}

For all component $a \in \Gamma$, we also denote $\GRNreg{a} = \{b \in \Gamma \mid \exists b \xrightarrow{t} a \in E_+ \cup E_-\}$ the set of its regulators,
and $E_+(a) = \{b \in \Gamma \mid \exists b \xrightarrow{t} a \in E_+\}$ (resp. $E_-(a) = \{b \in \Gamma \mid \exists b \xrightarrow{t} a \in E_-\}$) the set of its positive (resp. negative) regulators.
\todo{Check if it is a good idea to use the same symbol $\f{Reg}$ for both BRN and PH.}

\begin{example*}
\pref{fig:runningBRN-ig} represents an Interaction Graph $(\Gamma,E_+,E_-)$ with
$\Gamma = \{a, b, c\}$,
$E_+ = \{b \xrightarrow{1} a, c \xrightarrow{1} a\}$ and
$E_- = \{a \xrightarrow{2} b\}$.
Furthermore, we have: $E_+(a) = \PHpredecgene{a} = \{b, c\}$ and $E_-(a) = \emptyset$.
This IG can represent the same behavior as the PH given in \pref{fig:runningPH-2}.

\begin{figure}[t]
\centering
\scalebox{1.5}{
\begin{tikzpicture}[grn]
\path[use as bounding box] (-0.5,-0.75) rectangle (4.5,0.7);
\node[inner sep=0] (a) at (2,0) {a};
\node[inner sep=0] (b) at (0,0) {b};
\node[inner sep=0] (c) at (4,0) {c};
\path
  node[elabel, below=-1em of a] {$0..2$}
  node[elabel, below=-1em of b] {$0..1$}
  node[elabel, below=-1em of c] {$0..1$};
\path
  (b) edge[act, bend right] node[elabel, below=-2pt] {$+1$} (a)
  (c) edge[act] node[elabel, above=-2pt] {$+1$} (a)
  (a) edge[inh, bend right] node[elabel, above=-5pt] {$-2$} (b);
\end{tikzpicture}
}
\caption{\label{fig:runningBRN-ig}
An Interaction Graph example.
Components are represented by nodes labeled with a name and possible expression levels.
Regulations are represented by the edges, labeled with a sign that stands for their type ($+$ for positive and $-$ for negative) and a threshold.
For instance, the edge from $b$ to $a$ is labeled $+1$, which stands for: $b \xrightarrow{1} a \in E_+$,
and means that if the level of expression of $b$ is equal to (i.e. above) 1, then $b$ activates $a$,
otherwise, $b$ inhibits $a$.
}
\end{figure}
\end{example*}

A \emph{state} of an IG $(\Gamma, E_+, E_-)$ is an element in $\prod_{a \in \Gamma} \{a_0, \dots, a_{l_a}\}$.
The specificity of René Thomas' approach lies in the use of discrete \emph{parameters} to represent the focal point towards which the expression level of a component will evolve in each configuration of its regulators.
Indeed, for each possible state of a BRN, all regulators of a gene are be divided into \emph{activators} and \emph{inhibitors}, given their type of interaction and expression level.
The direction of evolution of a gene thus depends on these \emph{resources} in the considered state.
In this paper, we extend the classical definition focal point from a unique integer to an interval of integers.
\todo{The interval semantic is more expressive: justify!}
The association of an IG and a related Parametrization corresponds to the definition a \emph{BRN using René Thomas' parameters}, which entirely describes the structure and allows to compute the dynamics of the modeled system.
\todo{Find another name for “BRN using René Thomas' parameters”?}

\begin{definition}[Discrete parameter $K_{a,A,B}$ and Parametrization $K$]\label{def:param}
For a given component $a \in \Gamma$ and $A$ (resp. $B$) $\subset \GRNreg{a}$ a set of activators (resp. inhibitors) of $a$ such that
$A \cup B = \GRNreg{a}$ and $A \cap B = \emptyset$,
we define the discrete \emph{parameter} $K_{a,A,B} = [i_1; i_2]$ as a non-empty interval towards which the component $a$ will tend
in the states where its activators (resp. inhibitors) are the regulators in set $A$ (resp. $B$).
A complete map $K$ of discrete parameters on an IG $\IG$ is called a \emph{parametrization} of $\IG$.
\end{definition}
A consequence of this definition is that $0 \leq i_1 \leq i_2 \leq l_a$.
%We also denote: $j < K_{a,A,B} \Leftrightarrow j < i_1$ and $j > K_{a,A,B} \Leftrightarrow j> i_2$.

\begin{definition}[Resources $\GRNres{a}{s}$]\label{def:resources}
For a given state $s$ of a BRN, we define the \emph{activators} and \emph{inhibitors} of $a$ in $s$ as $\GRNres{a}{s} = A,B$, where:
\begin{align*}
  A &= \{b \in \Gamma \mid \GRNget{s}{b} \in \levelsA{b}{a}\} \\
  B &= \{b \in \Gamma \mid \GRNget{s}{b} \in \levelsI{b}{a}\}
\end{align*}
\end{definition}

\begin{definition}[Biological Regulatory Network using Thomas' parameters]\label{def:brn}
A \emph{biological regulatory network (BRN) using Thomas' parameters} is a pair $(\IG; K)$ where the first entry is an Interaction Graph and the second is a complete Parametrization.
\end{definition}
We call $s$ a state of a BRN using Thomas' parameters if $s$ is a state of the IG $\IG$.

\begin{example*}
In order to achieve the very same behavior as the PH given in \pref{fig:runningPH-2}, we can use the Parametrization of the IG of \pref{fig:runningBRN-ig} given in \pref{fig:runningBRN-param}.

\begin{figure}[t]
\begin{align*}
K_{a,\{b,c\},\emptyset} &= [2 ; 2] & K_{b,\{a\},\emptyset} &= [0 ; 1] \\
K_{a,\{b\},\{c\}} &= [1 ; 1] & K_{b,\emptyset,\{a\}} &= [0 ; 0] \\
K_{a,\{c\},\{b\}} &= [1 ; 1] &&\\
K_{a,\emptyset,\{b,c\}} &= [0 ; 0] & K_{c,\emptyset,\emptyset} &= [0 ; 1]
\end{align*}
\caption{\label{fig:runningBRN-param}
An possible Parametrization of the IG of \pref{fig:runningBRN-ig}.
This Parametrization ensures the same behavior as the PH given in \pref{fig:runningPH-2}.
}
\end{figure}
\end{example*}

At last, we describe the asynchronous dynamics of a BRN using Thomas' parameters.
From a given state $s$, a transition to another state $s'$ is possible provided that only one component $a$ will evolve of one expression level towards the nearest value of $K_{a,\GRNres{a}{s}}$.

\begin{definition}[Asynchronous dynamics]\label{def:dynamics}
Let $s$ be a state of a BRN using Thomas' parameters $(\IG, K)$ where $\IG = (\Gamma, E_+, E_-)$.
The state that succeeds to $s$ is given by the indeterministic function $f(s)$:
\begin{align*}
  & f(s) = s' \Longrightarrow \exists a \in \Gamma,
    \GRNget{s'}{a} = f^a(s) \wedge
    \forall b \in \Gamma, b \neq a, \GRNget{s}{b} = \GRNget{s'}{b}
    \quad\text{, with}\\
  & f^a(s) =
  \begin{cases}
    \GRNget{s}{a} + 1 & \text{if } \GRNget{s}{a} < K_{a, \GRNres{a}{s}} \\
    \GRNget{s}{a} & \text{if } \GRNget{s}{a} \in K_{a,\GRNres{a}{s}}\\
    \GRNget{s}{a} - 1 & \text{if } \GRNget{s}{a} > K_{a,\GRNres{a}{s}}
  \end{cases}
\end{align*}
\end{definition}

\begin{example*}
In the BRN that consists of the IG in \pref{fig:runningBRN-ig} and the Parametrization in \pref{fig:runningBRN-param}, the following transitions are possible given the dynamics defined in \pref{def:dynamics}:
\[\GRNetat{a_0, b_1, c_1} \rightarrow \GRNetat{a_1, b_1, c_1} \rightarrow \GRNetat{a_2, b_1, c_1} \rightarrow
\GRNetat{a_2, b_0, c_1} \rightarrow \GRNetat{a_1, b_0, c_1}.\]
\end{example*}
