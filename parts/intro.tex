\section{Introduction}
As regulatory phenomena play a crucial role in biological systems, they need to be studied accurately.
Biological Regulatory Networks (BRNs) consist in sets of either positive or negative mutual effects between the components.
With the purpose of analyzing these systems, they are often modeled as graphs which make it possible to determine the possible evolutions of all the interacting components of the system.
Indeed, besides continuous models of physicists, often designed through systems of ordinary
differential equations, a discrete modeling approach was initiated by René Thomas in 1973
\cite{Thomas73}.

In this approach, the different levels of a component, such as concentration or expression levels, are abstractly represented by (positive) integer values and transitions between these levels may be considered as instantaneous.
Hence, qualitative state graphs may be derived from which we are able to formally find out all the possible behaviors expressed as sequences of transitions between these states.
Nevertheless, these dynamics can be precisely established only with regard to some discrete parameters which stand for a kind of ``focal points'', i.e. the evolutionary tendency from each state and depending of the set of resources in this very state, that is, the set of the other currently interacting components.
Hereafter, we refer to these discrete parameters as Thomas' parameters.

Thomas' modeling has motivated numerous works around the link between the Interaction Graph (IG)
(summarizing the global influences between components) and the possible dynamics (e.g.,
\cite{RiCo07,RRT08}),
model reduction (e.g., \cite{Naldi09}), formal checking of dynamics (e.g., \cite{Richard06,Naldi07}), 
and the incorporation of time (e.g., \cite{Siebert06,Ahmad08}) and probability
(e.g., \cite{Twardziok10-CMSB}) dimensions, to name but a few.
While the formal checking of dynamical properties is often limited to small networks because of the
state graph explosion, the main drawback of this framework is the difficulty to specify Thomas'
parameters, especially for large networks.

In order to address the formal checking of dynamical properties within very large BRNs, we recently
introduced in \cite{PMR10-TCSB} a new formalism, named the \emph{``Process Hitting''} (PH), to model
concurrent systems having components with a few qualitative levels.
A PH describes, in an atomic manner, the possible evolutions of a process (representing one
component at one level) triggered by the hit of at most one other process in the system.
This framework can be seen as a special class of formalisms like Petri Nets or Communicating Finite
State Machines, where the causality between actions is restricted.
Thanks to the particular structure of interactions within a PH, very efficient static analysis
methods have been developed to over- and under-approximate reachability properties making tractable
the formal analysis of BRNs with hundreds of components \cite{PMR12-MSCS}.

The PH is suitable to model BRNs with different levels of abstraction in the specification of
cooperations (associated influences) between components.
This allows to model BRNs with a partial knowledge on precise evolution functions for components
by capturing the largest (the most general) dynamics.

The objectives of the work presented in this paper are the following.
Firstly, we show that starting from one PH model, it is possible to find back the underlying IG.
We perform an exhaustive search for the possible interactions on one component from all the
others, consistently with the knowledge of the dynamics that these interactions lead to and that are
expressed in PH.
The second phase of our work concerns the Thomas' parameters inference.
It consists in determining the nesting set (possibly too large) of the parameters which necessarily
lead to the satisfaction of the known cooperating constraints.
The resulting BRN dynamics is ensured to respect the PH dynamics, \ie no spurious transitions are
made possible by the inference.
Answer Set Programming (ASP) \cite{Baral03} turns out to be effective for these enumerative searches.

The outcome of this work is twofold.
The first benefit is that such an approach makes it possible to refine the construction of
BRNs with a partial and progressively brought knowledge in PH, while being able to export such
models in the Thomas' framework.
The second feature of our method is that it can be applied on very large BRNs.

Finally, it must be noticed that we are not interested in this paper in the derivation of one
PH from a BRN (which was previously described in \cite{PMR10-TCSB}) but, on the contrary, to finding out
a set of BRNs from one PH.

Our work is related to the approach of \cite{Khalis09} which relies on temporal logic, and \cite{20646302,DBLP:conf/ipcat/CorblinFTCT12} which also uses constraint programming. Both aim at determining a class of models which are consistent with available partial data on the regulatory structure and dynamical properties.
%These classes are built in order to infer properties common to some studied models.
In our approach, we intend to focus on the Thomas' parameters inference and we claim we are able to deal with larger biological networks.

\paragraph{Outline.}
\pref{sec:frameworks} recalls the PH and Thomas frameworks;
\pref{sec:infer-IG} defines the IG inference from PH;
\pref{sec:infer-K} details the enumeration of Thomas parametrizations compatible with a PH
and discuss its implementation in ASP.
\pref{sec:examples} illustrates the applicability of our method on simple examples
and large biological models.

\paragraph{Notations.}
$[i;j]$ is the set of integers $\{ i, i+1, \dots, j \}$;
we note $[i_1;j_1] \leq_{[]} [i_2;j_2] \EQDEF (i_1\leq j_1\wedge i_2\leq j_2)$
and $[i_1;j_1] <_{[]} [i_2;j_2] \EQDEF 
%	[i_1;j_1] \leq_{[]} [i_2;j_2] \wedge (i_1\neq i_2 \vee j_1\neq j_2)$.
(i_1<i_2 \wedge j_1\leq j_2) \vee (i_1\leq i_2\wedge j_1 < j_2)$.
Given an integer $k$, $k<[i;j] \EQDEF k <i$ and $k>[i;j]\EQDEF k > j$.
